\documentclass[a4paper,11pt,oneside,openany]{scrbook}
\usepackage{monads}
\begin{document}

\begin{titlepage}
	\begin{center}
		\Huge \textbf{Monads~and~their~applications~II}\\
		\vspace{1cm}
		\Large Dr.\ Daniel Schäppi's course lecture notes\\
	\end{center}

	\vspace{1cm}

	\begin{center}
		\Large	by\\
		\vspace{.2cm}
		\Large	Nicola Di Vittorio\\
		\Large	Matteo Durante\\
	\end{center}
\end{titlepage}
\thispagestyle{empty}\doclicenseThis

\frontmatter


\mainmatter

\chapter*{2-Monads and Their 2-Categories of Algebras}

\section{Introduction}

These notes will focus on 2-dimensional monad theory, which can be viewed as the
study of algebraic structures on 2-categories. Like in the one-dimensional case,
after defining a 2-monad we concern ourselves with the categories of algebras it
defines, however the higher dimension allows to relax the definitions and
observe how different coherence conditions lead to different (and generally less
well-behaved) objects.

One may ask why we are keen to better understand 2-monads. One answer is
that, similarly to the 1-dimensional case, this allows us to better understand
other 2-categories, perhaps with additional structure (i.e.\ monoidal, braided,
some kinds of limits, etc) by relating them to 2-categories of algebras.

We now start recalling some relevant definitions and facts which we will need
later on.

In order to carry out our project we shall work with $\cV$-cosmos and
presentability conditions.

\begin{defn}
    A cosmos $\cV$ is a complete, cocomplete symmetric monoidal closed category.
\end{defn}

\begin{defn}
    An object $c$ in a $\cV$-category $\cC$ is $\kappa$-presentable if
    $\cC(c,-)\colon\cC\rightarrow\cV$ preserves $\kappa$-filtered colimits. This
    is equivalent to saying that the functor
    $\cC(c,-)\colon\cC_0\rightarrow\cV_0$ is $\kappa$-accessible, where $\cC_0$
    and $\cV_0$ are the underlying categories.
\end{defn}

\begin{thm}
    Let $\cV$ be a lfp cosmos. Then $\cV$-$\Cat$ is a lfp cosmos and a lfp
    2-category.
\end{thm}

By studying monads in this setting we achieve a great level of generality since
our results will not depend on the underlying enrichment, thus unifying many
contexts.

But what is a 2-monad?

\begin{defn}
    A 2-monad is a monad in the 2-category 2-$\CAT$ of locally small
    2-categories, 2-functors and (strict) 2-natural transformations.
\end{defn}

We will often construct them using presentations, that is via colimit
constructions and free 2-monads on 2-endofunctors. This is achieved through the
following results.

\begin{thm}\label{lfpadj}
    Let $\cV$ be a lfp cosmos, $\cC$ a locally $\kappa$-presentable
    $\cV$-category. Then the forgetful functor
    \[
        \cV-\Mnd_\kappa(\cC)\rightarrow\cV-\CAT_\kappa(\cC,\cC)
    \]
    is monadic. Moreover, it preserves colimits.
\end{thm}

\begin{cor}
    In the above situation, the functor
    \[
        (-)\Alg\colon\cV-\Mnd_\kappa(\cC)\rightarrow\cV-\CAT/\cC
    \]
    sends colimits to limits.
\end{cor}

\begin{rmk}
    In general, $\cV-\Mnd_\kappa(\cC)$ is not a $\cV$-category. This is because
    monads are monoids in a monoidal $\cV$-category of endofunctors, but monoids
    in general do not define a $\cV$-category: for example, consider
    $\Mon(\Ab)=\Ring$, which is not even additive.

    This has to do with the non-existence of a ``diagonal'' $\cV$-functor
    $\cV\rightarrow\cV\otimes\cV$. In particular, if $\cV$ is cartesian then
    this problem does not arise and indeed for $\cV=\Cat$ we expect the
    monadic adjunction~\ref{lfpadj} to be enriched.
\end{rmk}

Unfortunately, we can't apply the theorem above to show the corollary. Instead,
we use it to give a presentation of a 2-monad whose algebras are 2-monads with
rank $\kappa$.

Given a monoidal 2-category $\cM$ (i.e.\ the associator $(A\otimes B)\otimes
C\rightarrow A\otimes(B\otimes C)$ is 2-natural, satisfies the pentagon axioms,
etc), we have a 2-category $\Mon(\cM)$ of monoids $(M,\mu\colon M\otimes
M\rightarrow M,\eta\colon I\rightarrow M)$ in $\cM$ with 1-cells the monoid
morphisms and 2-cells the 2-cells $\alpha\colon f\Rightarrow g\colon M
\rightarrow N$ in $\cM$ s.t.\
\[\begin{tikzcd}
	{M\otimes M} & M & N & {=} & {M\otimes M} && {N\otimes N} & N
	\arrow[""{name=0, anchor=center, inner sep=0}, "f", curve={height=-12pt}, from=1-2, to=1-3]
	\arrow[""{name=1, anchor=center, inner sep=0}, "g"', curve={height=12pt}, from=1-2, to=1-3]
	\arrow["{\mu_M}"', from=1-1, to=1-2]
	\arrow["{\mu_N}"', from=1-7, to=1-8]
	\arrow[""{name=2, anchor=center, inner sep=0}, "{g\otimes g}"', curve={height=12pt}, from=1-5, to=1-7]
	\arrow[""{name=3, anchor=center, inner sep=0}, "{f\otimes f}", curve={height=-12pt}, from=1-5, to=1-7]
	\arrow["\alpha", shorten <=3pt, shorten >=3pt, Rightarrow, from=0, to=1]
	\arrow["\alpha\otimes\alpha", shorten <=3pt, shorten >=3pt, Rightarrow, from=3, to=2]
\end{tikzcd},\]
\[\begin{tikzcd}
	I & M & N & {=} & {\id_{\eta_N}}
	\arrow[""{name=0, anchor=center, inner sep=0}, "f", curve={height=-12pt}, from=1-2, to=1-3]
	\arrow[""{name=1, anchor=center, inner sep=0}, "g"', curve={height=12pt}, from=1-2, to=1-3]
	\arrow["{\eta_M}"', from=1-1, to=1-2]
	\arrow["\alpha", shorten <=3pt, shorten >=3pt, Rightarrow, from=0, to=1]
\end{tikzcd}\]
hold.

If $-\otimes-$ preserves $\kappa$-filtered colimits in each variable, then the
2-functors $FM=M\otimes M$, $GM=(M\otimes M)\otimes M+M+M$ are
$\kappa$-accessible and we have two natural ways to go from $F$-algebras to
$G$-algebras.

The coequalizer of the resulting pair of maps on the free monads
$TG\rightrightarrows TF$ gives us a presentation of a 2-monad $T$ as a
coequalizer. It has $T\Alg\cong\Mon(\cM)$ by construction if $\cM$ is locally
$\kappa$-presentable as a 2-category.

Let $\cK$ be a locally $\kappa$-presentable 2-category, i.e.\ $\cV-\Cat$ and
specifically $\Cat$, and let $\cM=[\cK,\cK]_\kappa$. Then the category of
$\kappa$-accessible endofunctors on $\cK$, that is $\cM$, is itself locally
$\kappa$-presentable.

Notice that the composition preserves $\kappa$-filtered colimits in each
varible. Indeed, for $F^*$ it's clear and for $F_*$ is too since $F$ is
$\kappa$-accessible.

Monoids in $\cM$ are 2-monads!

To show that $2-\Mnd_\kappa(\cK)\rightarrow 2-\Mnd(\cK)$ preserves colimits we
need the following proposition.

\begin{prop}
    Let $F$ be a strong monoidal 2-adjoint $\cM\rightarrow\cM'$. Then the right
    2-adjoint inherits a lax monoidal structure s.t.\ unit and counit are
    monoidal. Both 2-functors lift to the 2-categories of monoids, so
    $\Mon(F)\colon\Mon(\cM)\rightarrow\Mon(\cM')$ is a left 2-adjoint.
\end{prop}
\begin{proof}
    Exercise.
\end{proof}

We can now prove what we stated earlier.

\begin{thm}
    Let $\cK$ be a locally $\kappa$-presentable 2-category. Then the forgetful
    2-functor
    \[
        2-\Mnd_\kappa(\cK)\rightarrow[\cK,\cK]_\kappa
    \]
    is 2-monadic and $\kappa$-accessible. In particular, $2-\Mnd_\kappa(\cK)$ is
    a locally $\kappa$-presentable 2-category.
    Moreover, the inclusion
    \[
        2-\Mnd_\kappa(\cK)\rightarrow 2-\Mnd(\cK)
    \]
    preserves colimits and in fact it is a left adjoint.
\end{thm}
\begin{proof}
    We have $2-\Mnd_\kappa(\cK)=\Mon([\cK,\cK]_\kappa)$, so the above discussion
    shows that there is a $\kappa$-accessible 2-monad on $[\cK,\cK]_\kappa$ with
    $T\Alg\cong 2-\Mnd_\kappa(\cK)$.

    For the second part, recall that left Kan extensions along the inclusion
    $J\colon\cK_\kappa\rightarrow\cK$ of $\kappa$-presentable objects gives an
    equivalence of 2-categories $[\cK_\kappa,\cK]\rightarrow[\cK,\cK]_\kappa$
    (this is true for a general lfp cosmos ----missing bit, it was 11:23----).

    It follows that the inclusion $[\cK,\cK]_\kappa\rightarrow[\cK,\cK]$ is, up
    to equivalence, given by the left Kan extension along $J$. (Check and finish
    this proof)
\end{proof}

This will allows us to write presentations of 2-monads for 2-categories such as
$\bbR$-linear categories, simplicial categories, etc, which has two important
consequences: firstly, when constructing a 2-monad from free monads we may also
use weighted colimits; secondly, since 2-monads with rank $\kappa$ are algebras
for a 2-monad with rank $\kappa$, any general theorem we prove about algebras
gives a corresponding 2-monad with rank $\kappa$.

As we mentioned earlier, we may be interested in less strict definitions
compared to the 1-dimensional case. Here we start considering them by specifying
new classes of morphisms of algebras.

\begin{defn}
    Let $T$ be a 2-monad, $(A,a)$, $(B,b)$ two $T$-algebras.

    A lax $T$-morphism is a pair $(f,\overline{f})$ where $f\colon A\rightarrow
    B$ is a 1-cell and $\overline{f}\colon b\cdot Tf\rightarrow f\cdot a$ is a
    2-cell such that the equations
    \[\begin{tikzcd}
        {T^2A} & TA & A & {=} & {T^2A} & TA & A \\
        {T^2B} & TB & B && {T^2B} & TB & B
        \arrow["a", from=1-2, to=1-3]
        \arrow["f", from=1-3, to=2-3]
        \arrow["b"', from=2-2, to=2-3]
        \arrow["Tf"{description}, from=1-2, to=2-2]
        \arrow["{T^2f}"', from=1-1, to=2-1]
        \arrow["{\mu_B}"', from=2-1, to=2-2]
        \arrow["{\mu_A}", from=1-1, to=1-2]
        \arrow["{\overline{f}}", shorten <=6pt, shorten >=6pt, Rightarrow, from=2-2, to=1-3]
        \arrow[shorten <=8pt, shorten >=8pt, Rightarrow, no head, from=2-1, to=1-2]
        \arrow["b"', from=2-6, to=2-7]
        \arrow["f", from=1-7, to=2-7]
        \arrow["Tf"{description}, from=1-6, to=2-6]
        \arrow["Tb"', from=2-5, to=2-6]
        \arrow["a", from=1-6, to=1-7]
        \arrow["{T^2f}"', from=1-5, to=2-5]
        \arrow["Ta", from=1-5, to=1-6]
        \arrow["{\overline{f}}", shorten <=6pt, shorten >=6pt, Rightarrow, from=2-6, to=1-7]
        \arrow["{T\overline{f}}", shorten <=6pt, shorten >=6pt, Rightarrow, from=2-5, to=1-6]
    \end{tikzcd},\]
    \[\begin{tikzcd}
        A & TA & A & {=} & {\id_f} \\
        B & TB & B
        \arrow["f", from=1-3, to=2-3]
        \arrow["Tf"{description}, from=1-2, to=2-2]
        \arrow["f"', from=1-1, to=2-1]
        \arrow["{\eta_B}"', from=2-1, to=2-2]
        \arrow["b"', from=2-2, to=2-3]
        \arrow["a", from=1-2, to=1-3]
        \arrow["{\eta_A}", from=1-1, to=1-2]
        \arrow["{\overline{f}}", shorten <=6pt, shorten >=6pt, Rightarrow, from=2-2, to=1-3]
        \arrow[shorten <=8pt, shorten >=8pt, Rightarrow, no head, from=2-1, to=1-2]
    \end{tikzcd}\]
    hold.

    A lax $T$-morphism is a pseudo $T$-morphism if $\overline{f}$ is an
    isomorphism and it is strict if $\overline{f}=\id$.

    A colax or oplax $T$-morphism is a lax $T$-morphism with the direction of
    $\overline{f}$ reversed and the equations adapted.

    A 2-cell between lax/pseudo/strict $T$-morphisms $\alpha\colon
    (f,\overline{f})\Rightarrow(g,\overline{g})$ is a 2-cell $\alpha\colon
    f\Rightarrow g$ s.t.\
    \[\begin{tikzcd}
        TA & A & {=} & TA & A \\
        TB & B && TB & B
        \arrow[""{name=0, anchor=center, inner sep=0}, "g", curve={height=-12pt}, from=1-2, to=2-2]
        \arrow[""{name=1, anchor=center, inner sep=0}, "f"', curve={height=12pt}, from=1-2, to=2-2]
        \arrow["b"', from=2-1, to=2-2]
        \arrow["a", from=1-1, to=1-2]
        \arrow[""{name=2, anchor=center, inner sep=0}, "Tf"', curve={height=12pt}, from=1-1, to=2-1]
        \arrow[""{name=3, anchor=center, inner sep=0}, "g", curve={height=-12pt}, from=1-5, to=2-5]
        \arrow["b"', from=2-4, to=2-5]
        \arrow["a", from=1-4, to=1-5]
        \arrow[""{name=4, anchor=center, inner sep=0}, curve={height=-12pt}, from=1-4, to=2-4]
        \arrow[""{name=5, anchor=center, inner sep=0}, curve={height=12pt}, from=1-4, to=2-4]
        \arrow["T\alpha", shorten <=5pt, shorten >=5pt, Rightarrow, from=5, to=4]
        \arrow["\alpha", shorten <=5pt, shorten >=5pt, Rightarrow, from=1, to=0]
        \arrow["{\overline{f}}", shorten <=13pt, shorten >=13pt, Rightarrow, from=2, to=1]
        \arrow["{\overline{g}}", shorten <=13pt, shorten >=13pt, Rightarrow, from=4, to=3]
    \end{tikzcd}\]

    We write $T\Alg_S$, $T\Alg_P$ and $T\Alg_L$ for the 2-categories of
    $T$-algebras, strict/pseudo/lax $T$-morphisms and 2-cells as above.
\end{defn}

(Other missing bit)

\section{Presentations of 2-Monads}

We have defined two 2-categories $T\Alg_P$, $T\Alg_L$ of pseudo and lax
morphisms respectively for a 2-monad $T$. We want to understand how to describe
them when $T$ is given by a presentation.

We remember that in a complete 2-category $\cK$ we have a 2-endofunctor
$<A,B>\colon\cK\rightarrow\cK$ for each pair of objects $A$, $B$ in $\cK$ given
by the right Kan extension of $B\colon *\rightarrow\cK$ along $A\colon
*\rightarrow\cK$. In particular, $<A,B>C=B^{\cK(C,A)}$ and, if $A=B$, this
defines a 2-monad, just like in the 1-dimensional case. Moreover, the 2-monad
morphisms $T\Rightarrow<A,B>$ are in natural bijection with $T$-algebra
structures on $A$.

Now we can form for any pair of 1-cells $f,g\colon A\rightarrow B$ in $\cK$ the
(iso???) comma object
\[\begin{tikzcd}
	{\{f,g\}_{p/l}} & {<A,A>} \\
	{<B,B>} & {<A,B>}
	\arrow["d"', from=1-1, to=2-1]
	\arrow["{<A,f>}", from=1-2, to=2-2]
	\arrow["c", from=1-1, to=1-2]
	\arrow["{<g,B>}"', from=2-1, to=2-2]
	\arrow["\lambda"', shorten <=10pt, shorten >=10pt, Rightarrow, from=1-2, to=2-1]
\end{tikzcd}\]

in $[\cK,\cK]$. If $f=g$, then this is again a 2-monad and 2-monad morphisms
$T\rightarrow\{f,f\}_{p/l}$ correspond to (pseudo) lax $T$-morphism structures
on the 1-cell $f$. More precisely, such a morphism corresponds to a $T$-algebra
structure on $A$ and one on $B$, namely $c\cdot\gamma$ and $d\cdot\gamma$ and a
(invertible) 2-cell $\overline{f}\colon Tf\cdot b\Rightarrow f\cdot a$
corresponding to $\lambda\cdot\gamma$ s.t.\ $(f,\overline{f})$ is a lax (pseudo)
$T$-morphism.
\[\begin{tikzcd}
	{[\rho,\rho]} & {\{f,f\}} \\
	{\{g,g\}_l} & {\{f,g\}_l}
	\arrow["{\{f,\rho\}_l}", from=1-2, to=2-2]
	\arrow[from=1-1, to=1-2]
	\arrow["{\{\rho,g\}_l}"', from=2-1, to=2-2]
	\arrow[from=1-1, to=2-1]
	\arrow["\lrcorner"{anchor=center, pos=0.125}, draw=none, from=1-1, to=2-2]
\end{tikzcd}\]
which inherits a 2-monad structure for which a 2-monad morphism $T\Rightarrow
[\rho,\rho]$ exists if and only if $\rho$ is a $T$-transformation between
$(f,\overline{f})$ and $(g,\overline{g})$.

These facts can be used to identify $T\Alg_P$ and $T\Alg_S$ is $T$ is given as a
(weighted) colimit of free monads.

\begin{exmp}
    Let's consider the 2-monad of monads in a monoidal 2-category $\cM$ as
    above, i.e.\ locally $\kappa$-presentable with $-\otimes-$ preserving
    $\kappa$-filtered colimits in each variable. As we saw, we define
    $FM=M\otimes M+I$, $GM=(M\otimes M)\otimes M+M+M$. Let's write $T(F)$,
    $T(G)$ for the free 2-monads on these 2-endofunctors.

    There is a natural 2-functor $T(F)\Alg_S\rightarrow T(G)\Alg_S$ sending
    $(M,p,u)$ to $(M,p\cdot(p\otimes u),p\cdot(u\otimes M))$ and there is
    another two functor mapping it to $(M,p\cdot(M\otimes p),\id_M,\id_M)$.
    These correspond to 2-monad morphisms and the 2-monad for monoids is exactly
    its coequalizer.
\end{exmp}

A relevant question: what would happen if we considered lax/pseudo
$T$-morphisms in this case? The simple existence of $\{f,f\}_l$ tells us that
this is some kind of equalizer, however there is a problem: what is $T(F)\Alg_l$
and what does the 2-functor $T(F)\Alg_l\rightarrow T(G)\Alg_l$ look like?

From $T(F)\rightsquigarrow\{f,f\}_l$ we get a morphism $T\rightarrow
T(F)\rightarrow\{f,f\}_l$, which is however hard to analyze. This requires a bit
of a detour.

\begin{thm}[doctrinal adjunction]
    Let $(f,\overline{f})\colon(A,a)\rightarrow(B,b)$ be a pseudo $T$-morphism
    s.t.\ $f$ is a left adjoint to $u\colon B\rightarrow A$ with unit $\eta$ and
    counit $\epsilon$. Then there exists a unique lax $T$-morphism structure
    $\overline{u}$ on $u$ s.t.\ $\eta$ and $\epsilon$ are $T$-transformations.
\end{thm}
\begin{proof}
    We shall prove uniqueness. For this, we observe that the equality
    \[\begin{tikzcd}
        TA & A &&& TA && A \\
        && B & {=} && TB && B \\
        TA & A &&& TA && A
        \arrow["a", from=1-1, to=1-2]
        \arrow["f", from=1-2, to=2-3]
        \arrow[""{name=0, anchor=center, inner sep=0}, Rightarrow, no head, from=1-1, to=3-1]
        \arrow[""{name=1, anchor=center, inner sep=0}, Rightarrow, no head, from=1-2, to=3-2]
        \arrow["u", from=2-3, to=3-2]
        \arrow["a", from=3-1, to=3-2]
        \arrow["a"', from=3-5, to=3-7]
        \arrow["a", from=1-5, to=1-7]
        \arrow[""{name=2, anchor=center, inner sep=0}, Rightarrow, no head, from=1-5, to=3-5]
        \arrow[""{name=3, anchor=center, inner sep=0}, "f", from=1-7, to=2-8]
        \arrow[""{name=4, anchor=center, inner sep=0}, "u", from=2-8, to=3-7]
        \arrow["b"{pos=0.3}, from=2-6, to=2-8]
        \arrow[""{name=5, anchor=center, inner sep=0}, "Tf", from=1-5, to=2-6]
        \arrow[""{name=6, anchor=center, inner sep=0}, "Tu", from=2-6, to=3-5]
        \arrow[shorten <=13pt, shorten >=13pt, Rightarrow, no head, from=0, to=1]
        \arrow["T\eta", shorten <=10pt, shorten >=10pt, Rightarrow, from=2, to=2-6]
        \arrow["{\overline{b}}", shorten <=26pt, shorten >=26pt, Rightarrow, from=6, to=4]
        \arrow["{\overline{f}}", shorten <=26pt, shorten >=26pt, Rightarrow, from=5, to=3]
        \arrow["\eta", shorten <=10pt, shorten >=10pt, Rightarrow, from=1, to=2-3]
    \end{tikzcd}\]
    implies that (is the $\overline{f}$ inverted???)
    \[\begin{tikzcd}
        TA & TB & B & {=} & TA & TB \\
        & TA & A && A & B \\
        &&&&& A
        \arrow[""{name=0, anchor=center, inner sep=0}, "u", from=1-3, to=2-3]
        \arrow[""{name=1, anchor=center, inner sep=0}, "Tu"{description}, from=1-2, to=2-2]
        \arrow["Tf", from=1-1, to=1-2]
        \arrow["b", from=1-2, to=1-3]
        \arrow["a"', from=2-2, to=2-3]
        \arrow[""{name=2, anchor=center, inner sep=0}, Rightarrow, no head, from=1-1, to=2-2]
        \arrow[""{name=3, anchor=center, inner sep=0}, Rightarrow, no head, from=2-5, to=3-6]
        \arrow[""{name=4, anchor=center, inner sep=0}, "u", from=2-6, to=3-6]
        \arrow["f", from=2-5, to=2-6]
        \arrow[""{name=5, anchor=center, inner sep=0}, "b", from=1-6, to=2-6]
        \arrow[""{name=6, anchor=center, inner sep=0}, "a", from=1-5, to=2-5]
        \arrow["Tf", from=1-5, to=1-6]
        \arrow["{\overline{u}}", shorten <=13pt, shorten >=13pt, Rightarrow, from=1, to=0]
        \arrow["T\eta", shorten <=6pt, shorten >=6pt, Rightarrow, from=2, to=1]
        \arrow["{\overline{f}^{-1}}", shorten <=13pt, shorten >=13pt, Rightarrow, from=6, to=5]
        \arrow["\eta", shorten <=6pt, shorten >=6pt, Rightarrow, from=3, to=4]
    \end{tikzcd}\]
    and
    \[\begin{tikzcd}
        TB & B & {=} & draw \\
        TA & A & {}
        \arrow["b", from=1-1, to=1-2]
        \arrow["a"', from=2-1, to=2-2]
        \arrow[""{name=0, anchor=center, inner sep=0}, "u", from=1-2, to=2-2]
        \arrow[""{name=1, anchor=center, inner sep=0}, "Tu"', from=1-1, to=2-1]
        \arrow["{\overline{u}}", shorten <=13pt, shorten >=13pt, Rightarrow, from=1, to=0]
    \end{tikzcd}\]
    by the triangle identities for $Tf\dashv Tu$.

    For existance, $(u,\overline{u})$ is a lax $T$-morphism with the desired
    properties by exercise 13.4 from the previous course.
\end{proof}

We now study a kind of limit existing in $T\Alg_l$.

\begin{defn}
    Given a 2-category $\cK$ and an arrow $f\colon A\rightarrow B$ in it, it
    colax limit is the universal 2-cell
    \[\begin{tikzcd}
        & C && {} \\
        A && B
        \arrow["q", from=1-2, to=2-3]
        \arrow["p"', from=1-2, to=2-1]
        \arrow[""{name=0, anchor=center, inner sep=0}, "f"', from=2-1, to=2-3]
        \arrow["\lambda"', shorten <=6pt, shorten >=6pt, Rightarrow, from=1-2, to=0]
    \end{tikzcd}\]
    in $\cK$. This means that for each $a\colon X\rightarrow A$, $b\colon
    X\rightarrow B$ and $\alpha\colon f\cdot a\rightarrow b$ there exists a
    unique 1-cell $t\colon X\rightarrow C$ s.t.\
    \[\begin{tikzcd}
        & X && {=} && X \\
        A && B &&& C \\
        &&&& A && B
        \arrow["b", from=1-2, to=2-3]
        \arrow["a"', from=1-2, to=2-1]
        \arrow[""{name=0, anchor=center, inner sep=0}, "f"', from=2-1, to=2-3]
        \arrow["q", from=2-6, to=3-7]
        \arrow["p"', from=2-6, to=3-5]
        \arrow[""{name=1, anchor=center, inner sep=0}, "f"', from=3-5, to=3-7]
        \arrow["t", from=1-6, to=2-6]
        \arrow["\alpha"', shorten <=6pt, shorten >=6pt, Rightarrow, from=1-2, to=0]
        \arrow["\lambda"', shorten <=6pt, shorten >=6pt, Rightarrow, from=2-6, to=1]
    \end{tikzcd}\]
    holds. The 2-dimensional universal property asserts that for all $a'\colon
    A\rightarrow A$, $b'\colon X\rightarrow B$, $\alpha'\colon b'\rightarrow
    f\cdot a'$ and 2-cells $\gamma\colon a\Rightarrow a'$, $\delta\colon
    b\Rightarrow b'$ with
    \[\begin{tikzcd}
        && {} \\
        X & {} &&& {} && X \\
        && B & {=} & A \\
        A &&&&&& B
        \arrow[""{name=0, anchor=center, inner sep=0}, "{a'}"', curve={height=12pt}, from=2-1, to=4-1]
        \arrow[""{name=1, anchor=center, inner sep=0}, "a", curve={height=-12pt}, from=2-1, to=4-1]
        \arrow[""{name=2, anchor=center, inner sep=0}, "b", curve={height=-6pt}, from=2-1, to=3-3]
        \arrow[""{name=3, anchor=center, inner sep=0}, "f"', curve={height=6pt}, from=4-1, to=3-3]
        \arrow["{a'}"', curve={height=6pt}, from=2-7, to=3-5]
        \arrow["f"', curve={height=6pt}, from=3-5, to=4-7]
        \arrow[""{name=4, anchor=center, inner sep=0}, "b", curve={height=-12pt}, from=2-7, to=4-7]
        \arrow[""{name=5, anchor=center, inner sep=0}, "{b'}"', curve={height=12pt}, from=2-7, to=4-7]
        \arrow["\gamma"', shorten <=4pt, shorten >=4pt, Rightarrow, from=1, to=0]
        \arrow["\alpha", shorten <=12pt, shorten >=12pt, Rightarrow, from=2, to=3]
        \arrow["{\alpha'}"', shorten <=18pt, shorten >=18pt, Rightarrow, from=5, to=3-5]
        \arrow["\delta"', shorten <=4pt, shorten >=4pt, Rightarrow, from=4, to=5]
    \end{tikzcd}\]
    there exists a unique 2-cell $\phi\colon t\Rightarrow t'$ s.t.\
    $p\cdot\phi=\gamma$, $q\cdot\phi=\delta$.

    Notice that this is precisely the comma object
    \[\begin{tikzcd}
        {\id\downarrow f} & B \\
        A & B
        \arrow[from=1-1, to=2-1]
        \arrow["f", from=2-1, to=2-2]
        \arrow[Rightarrow, no head, from=1-2, to=2-2]
        \arrow[from=1-1, to=1-2]
        \arrow[shorten <=9pt, shorten >=9pt, Rightarrow, from=1-2, to=2-1]
    \end{tikzcd}\]
    in $\cK$. This is a weighted limit in the enriched sense, hence defined via
    an isomorphism of categories and not just an equivalence.

    The pseudo limit of $f$ is the analogous construction with $\lambda$ and
    $\alpha$ invertible. The lax limit has the direction of $\lambda$ reversed.
\end{defn}

We can now state the following.

\begin{prop}
    Let $\cK$ be a 2-category with colax limits of arrows and $T$ a 2-monad on
    it. For any 1-cell $(f,\overline{f})\colon (A,a)\rightsquigarrow (B,b)$ in
    $T\Alg_l$ there exists a unique $T$-algebra structure on the colax limit of
    $f$ s.t.\ the projections are strict 2-morphisms. The 2-cell
    \[\begin{tikzcd}
        & C && {} \\
        A && B
        \arrow["q", from=1-2, to=2-3]
        \arrow["p"', from=1-2, to=2-1]
        \arrow[""{name=0, anchor=center, inner sep=0}, "f"', rightsquigarrow, from=2-1, to=2-3]
        \arrow["\lambda"', shorten <=6pt, shorten >=6pt, Rightarrow, from=1-2, to=0]
    \end{tikzcd}\]
    is a $T$-transformation and $(G,\lambda)$ is a colax limit in $T\Alg_l$.
    Moreover, $p$ and $q$ jointly detect strict morphisms, that is a 1-cell
    $t\colon X\rightarrow C$ is strict if and only if $pt$ and $qt$ are strict.
    In particular, the colax limit of $(f,\overline{f})$ exists and it is
    strictly presented by the forgetful 2-functor $U_l\colon
    T\Alg_l\rightarrow\cK$.
\end{prop}
\begin{proof}
    There exists a unique 1-cell $c\colon TC\rightarrow C$ s.t.\ the equation
    \[\begin{tikzcd}
        & TA & A &&&& A \\
        TC &&& {=} & TC & C \\
        & TB & B &&&& B
        \arrow[""{name=0, anchor=center, inner sep=0}, "Tq"', from=2-1, to=3-2]
        \arrow["b"', from=3-2, to=3-3]
        \arrow[""{name=1, anchor=center, inner sep=0}, "f", from=1-3, to=3-3]
        \arrow[""{name=2, anchor=center, inner sep=0}, "Tf"{description}, from=1-2, to=3-2]
        \arrow["a", from=1-2, to=1-3]
        \arrow["Tp", from=2-1, to=1-2]
        \arrow["f", from=1-7, to=3-7]
        \arrow[""{name=3, anchor=center, inner sep=0}, "q"', from=2-6, to=3-7]
        \arrow["p", from=2-6, to=1-7]
        \arrow["c", from=2-5, to=2-6]
        \arrow["T\lambda", shorten <=12pt, shorten >=12pt, Rightarrow, from=0, to=1-2]
        \arrow["{\overline{f}}", shorten <=13pt, shorten >=13pt, Rightarrow, from=2, to=1]
        \arrow["\lambda", shorten <=12pt, shorten >=12pt, Rightarrow, from=3, to=1-7]
    \end{tikzcd}\]
    holds. Note that the direction of $\lambda$ is important! Since $p\cdot
    c=a\cdot Tp$, $q\cdot c=b\cdot Tq$, so if we can show that $(C,c)$ is a
    $T$-algebra then $p$ and $q$ are strict $T$-morphisms. Similarly, the above
    equation then says that $\lambda$ is a $T$-transformation.

    Applying $T$ to the above equation and whiskering the result on the right
    with $\overline{f}$ gives
    \[\begin{tikzcd}
        & {T^2A} & TA & A &&&& TA & A \\
        {T^2C} &&&& {=} & {T^2C} & TC \\
        & {T^2B} & TB & B &&&& TB & B
        \arrow[""{name=0, anchor=center, inner sep=0}, "{T^2q}"', from=2-1, to=3-2]
        \arrow["Tb"', from=3-2, to=3-3]
        \arrow[""{name=1, anchor=center, inner sep=0}, "Tf"{description}, from=1-3, to=3-3]
        \arrow[""{name=2, anchor=center, inner sep=0}, "{T^2f}"{description}, from=1-2, to=3-2]
        \arrow["Ta", from=1-2, to=1-3]
        \arrow["{T^2p}", from=2-1, to=1-2]
        \arrow["b", from=3-3, to=3-4]
        \arrow[""{name=3, anchor=center, inner sep=0}, "f", from=1-4, to=3-4]
        \arrow["a", from=1-3, to=1-4]
        \arrow[""{name=4, anchor=center, inner sep=0}, "f", from=1-9, to=3-9]
        \arrow["b"', from=3-8, to=3-9]
        \arrow[""{name=5, anchor=center, inner sep=0}, "Tf"{description}, from=1-8, to=3-8]
        \arrow["a", from=1-8, to=1-9]
        \arrow["Tp", from=2-7, to=1-8]
        \arrow[""{name=6, anchor=center, inner sep=0}, "Tq"', from=2-7, to=3-8]
        \arrow["Tc", from=2-6, to=2-7]
        \arrow["{T^2\lambda}", shorten <=12pt, shorten >=12pt, Rightarrow, from=0, to=1-2]
        \arrow["{T\overline{f}}", shorten <=13pt, shorten >=13pt, Rightarrow, from=2, to=1]
        \arrow["{\overline{f}}", shorten <=13pt, shorten >=13pt, Rightarrow, from=1, to=3]
        \arrow["{\overline{f}}", shorten <=13pt, shorten >=13pt, Rightarrow, from=5, to=4]
        \arrow["T\lambda", shorten <=12pt, shorten >=12pt, Rightarrow, from=6, to=1-8]
    \end{tikzcd}\]
    Notice that the diagram on the right reduces to
    \[\begin{tikzcd}
        && {} & A \\
        {T^2C} & TC & C \\
        &&& B
        \arrow["Tc", from=2-1, to=2-2]
        \arrow["c", from=2-2, to=2-3]
        \arrow[""{name=0, anchor=center, inner sep=0}, "q"', from=2-3, to=3-4]
        \arrow["p", from=2-3, to=1-4]
        \arrow["f", from=1-4, to=3-4]
        \arrow["\lambda", shorten <=12pt, shorten >=12pt, Rightarrow, from=0, to=1-4]
    \end{tikzcd}\]
    and applying the axioms for a lax T-morphism and the 2-naturality of
    $\mu\colon T^2\Rightarrow T$, we find that the left hand side above is
    \[\begin{tikzcd}
        && {T^2A} & TA & A \\
        & {T^2C} \\
        && {T^2B} & TB & B \\
        &&& TA & A \\
        {=} & {T^2C} & TC \\
        &&& TB & B \\
        &&&& A \\
        {=} & {T^2C} & TC & C \\
        &&&& B
        \arrow[""{name=0, anchor=center, inner sep=0}, "{T^2q}"', from=2-2, to=3-3]
        \arrow["a", from=1-4, to=1-5]
        \arrow[""{name=1, anchor=center, inner sep=0}, "f", from=1-5, to=3-5]
        \arrow["Tb"', from=3-3, to=3-4]
        \arrow[""{name=2, anchor=center, inner sep=0}, "Tf"{description}, from=1-4, to=3-4]
        \arrow["b", from=3-4, to=3-5]
        \arrow[""{name=3, anchor=center, inner sep=0}, "{T^2f}"{description}, from=1-3, to=3-3]
        \arrow["Ta", from=1-3, to=1-4]
        \arrow["{T^2p}", from=2-2, to=1-3]
        \arrow[""{name=4, anchor=center, inner sep=0}, "f", from=4-5, to=6-5]
        \arrow["a", from=4-4, to=4-5]
        \arrow["b"', from=6-4, to=6-5]
        \arrow[""{name=5, anchor=center, inner sep=0}, "Tq"', from=5-3, to=6-4]
        \arrow["Tp", from=5-3, to=4-4]
        \arrow[""{name=6, anchor=center, inner sep=0}, "Tf"{description}, from=4-4, to=6-4]
        \arrow["{\mu_C}", from=5-2, to=5-3]
        \arrow["f", from=7-5, to=9-5]
        \arrow[""{name=7, anchor=center, inner sep=0}, "q"', from=8-4, to=9-5]
        \arrow["p", from=8-4, to=7-5]
        \arrow["c", from=8-3, to=8-4]
        \arrow["{\mu_C}", from=8-2, to=8-3]
        \arrow["{\overline{f}}", shorten <=13pt, shorten >=13pt, Rightarrow, from=2, to=1]
        \arrow["{T^2\lambda}", shorten <=12pt, shorten >=12pt, Rightarrow, from=0, to=1-3]
        \arrow["{T\overline{f}}", shorten <=13pt, shorten >=13pt, Rightarrow, from=3, to=2]
        \arrow["T\lambda", shorten <=12pt, shorten >=12pt, Rightarrow, from=5, to=4-4]
        \arrow["{\overline{f}}", shorten <=13pt, shorten >=13pt, Rightarrow, from=6, to=4]
        \arrow["\lambda", shorten <=12pt, shorten >=12pt, Rightarrow, from=7, to=7-5]
    \end{tikzcd}\]
    so from the 1-dimensional universal property it follows that
    $c\cdot\mu_C=c\cdot Tc$.
The unit axiom is left as an exercise. To show that $(C,c)$ is a $T$-algebra, $p,q$ are strict morphisms and $\lambda$ is a $T$-transformation we have to check the universal properties. Consider a $2$-cell 
 \[\begin{tikzcd}
        & X && {} \\
        A && B
        \arrow["h", rightsquigarrow, from=1-2, to=2-3]
        \arrow["g"', rightsquigarrow, from=1-2, to=2-1]
        \arrow[""{name=0, anchor=center, inner sep=0}, "f"', rightsquigarrow, from=2-1, to=2-3]
        \arrow["\alpha"', shorten <=6pt, shorten >=6pt, Rightarrow, from=1-2, to=0]
    \end{tikzcd}\]
in $T\Alg_l$. This is a $2$-cell $\alpha\colon h\Rightarrow fg$ in $\K$ subject to the axiom for a $T$-transformation. In particular, there exists a unique $1$-cell $t\colon X\to C$ s.t.\ $\alpha=\lambda t$. The composite $\lambda\cdot c\cdot Tt$ corresponds to the $2$-cell  
\[\begin{tikzcd}
	& TA & A \\
	TX \\
	& TB & B
	\arrow["Tg", from=2-1, to=1-2]
	\arrow[""{name=0, anchor=center, inner sep=0}, "Th"', from=2-1, to=3-2]
	\arrow["a", from=1-2, to=1-3]
	\arrow["b"', from=3-2, to=3-3]
	\arrow[""{name=1, anchor=center, inner sep=0}, "f", from=1-3, to=3-3]
	\arrow[""{name=2, anchor=center, inner sep=0}, "Tf"{description}, from=1-2, to=3-2]
	\arrow["{\bar{f}}", shorten <=17pt, shorten >=17pt, Rightarrow, from=2, to=1]
	\arrow["T\alpha", shorten <=17pt, shorten >=17pt, Rightarrow, from=0, to=1-2]
\end{tikzcd}\]
and the composite $\lambda\cdot t\cdot x$ corresponds to the $2$-cell 
\[\begin{tikzcd}
	&& A \\
	TX & X \\
	&& B
	\arrow["x", from=2-1, to=2-2]
	\arrow["g", from=2-2, to=1-3]
	\arrow["h"', from=2-2, to=3-3]
	\arrow["f", from=1-3, to=3-3]
\end{tikzcd}\]
in $\K$. Since $\alpha$ is a $2$-cell in $T\Alg_l$, comparing the first of these with $\bar{g}\colon a\cdot Tg \Rightarrow g\cdot x$, we get the $2$-cell $\alpha\cdot x$ compared with $\bar{h}\colon b\cdot Th\Rightarrow h\cdot x$. In other words, $\bar{g}$ and $\bar{h}$ satisfy the defining equations for $2$-cells in the $2$-dimensional universal property of the colax limit of $f$. Thus there exists a unique $2$-cell $\bar{t}\colon c\cdot Tt \Rightarrow t\cdot x$ s.t.\ $p\cdot\bar{t} = \bar{g}$ and $q\cdot\bar{t}=\bar{h}$. If we can show that $(t,\bar{t})$ is a lax $T$-morphism, then these last equations show $p\cdot (t,\bar{t})=(g,\bar{g})$ and $q\cdot (t,\bar{t})=(h,\bar{h})$ as $1$-cells in $T\Alg_l$. Conversely, the equations also show that $(t,\bar{t})$ is unique. As a diagram, the equation $p\bar{t}=\bar{g}$ looks like
\[\begin{tikzcd}
	& X &&&&& X \\
	TX && C & A & {=} & TX & {} & A \\
	& TC &&&&& TA \\
	& TA
	\arrow["x", from=2-1, to=1-2]
	\arrow["t", from=1-2, to=2-3]
	\arrow[""{name=0, anchor=center, inner sep=0}, "Tt"', from=2-1, to=3-2]
	\arrow[""{name=1, anchor=center, inner sep=0}, "c"', from=3-2, to=2-3]
	\arrow["Tp"', from=3-2, to=4-2]
	\arrow[""{name=2, anchor=center, inner sep=0}, "Tg"', curve={height=12pt}, from=2-1, to=4-2]
	\arrow[""{name=3, anchor=center, inner sep=0}, "g", curve={height=-12pt}, from=1-2, to=2-4]
	\arrow["p"', from=2-3, to=2-4]
	\arrow[""{name=4, anchor=center, inner sep=0}, "a"', curve={height=24pt}, from=4-2, to=2-4]
	\arrow["{\bar{t}}"', shorten <=20pt, shorten >=20pt, Rightarrow, from=3-2, to=1-2]
	\arrow["x", from=2-6, to=1-7]
	\arrow["g", from=1-7, to=2-8]
	\arrow["Tg"', from=2-6, to=3-7]
	\arrow["a"', from=3-7, to=2-8]
	\arrow[shorten <=20pt, shorten >=20pt, Rightarrow, from=3-7, to=1-7]
	\arrow["{\bar{g}}"', shift left=4, Rightarrow, draw=none, from=3-7, to=1-7]
	\arrow[shorten <=15pt, shorten >=15pt, Rightarrow, no head, from=1, to=4]
	\arrow[shift left=3, shorten <=7pt, shorten >=5pt, Rightarrow, no head, from=0, to=2]
	\arrow[shorten <=4pt, shorten >=4pt, Rightarrow, no head, from=3, to=2-3]
\end{tikzcd}\]
in $\K$. Applying $T$ to this equation and composing with $Tg$ we get 
\[\begin{tikzcd}[column sep=5mm, row sep=5mm]
	&& X \\
	& TX && C &&&&& X \\
	{T^2X} && TC && A & {=} && TX && A \\
	& {T^2C} && TA &&& {T^2X} && TA \\
	&& {T^2A} &&&&& {T^2A} & {}&.
	\arrow["Tx", from=3-1, to=2-2]
	\arrow["x", from=2-2, to=1-3]
	\arrow["t", from=1-3, to=2-4]
	\arrow["Tt", from=2-2, to=3-3]
	\arrow["{T^2t}", from=3-1, to=4-2]
	\arrow["Tc"', from=4-2, to=3-3]
	\arrow["c"', from=3-3, to=2-4]
	\arrow["p", from=2-4, to=3-5]
	\arrow["Tp", from=3-3, to=4-4]
	\arrow[""{name=0, anchor=center, inner sep=0}, "g", curve={height=-35pt}, from=1-3, to=3-5]
	\arrow["{T^2p}", from=4-2, to=5-3]
	\arrow["Ta"', from=5-3, to=4-4]
	\arrow["{T\bar{t}}"', shorten <=15pt, shorten >=15pt, Rightarrow, from=4-2, to=2-2]
	\arrow["{\bar{t}}"', shorten <=15pt, shorten >=15pt, Rightarrow, from=3-3, to=1-3]
	\arrow[""{name=1, anchor=center, inner sep=0}, "{T^2g}"', curve={height=35pt}, from=3-1, to=5-3]
	\arrow["Tx", from=4-7, to=3-8]
	\arrow["x", from=3-8, to=2-9]
	\arrow["g", from=2-9, to=3-10]
	\arrow["{T^2g}"', from=4-7, to=5-8]
	\arrow["{T\bar{g}}", shorten <=15pt, shorten >=15pt, Rightarrow, from=5-8, to=3-8]
	\arrow["a"', from=4-4, to=3-5]
	\arrow["Tg"', from=3-8, to=4-9]
	\arrow["Ta"', from=5-8, to=4-9]
	\arrow["a"', from=4-9, to=3-10]
	\arrow["{\bar{g}}", shorten <=15pt, shorten >=15pt, Rightarrow, from=4-9, to=2-9]
	\arrow[shorten <=18pt, shorten >=18pt, Rightarrow, no head, from=4-4, to=2-4]
	\arrow[shorten <=18pt, shorten >=18pt, Rightarrow, no head, from=5-3, to=3-3]
	\arrow[shorten >=6pt, Rightarrow, no head, from=2-4, to=0]
	\arrow[shorten >=8pt, Rightarrow, no head, from=4-2, to=1]
\end{tikzcd}\]
Using the fact that $(g,\bar{g})$ is a lax $T$-morphism and the $2$-naturality of $\mu\colon T^2\Rightarrow T$ we find that the above $2$-cell is equal to
\[\begin{tikzcd}[column sep=5mm, row sep=5mm]
	&& X &&&&& X \\
	& TX && A &&& TX && C & A \\
	{T^2X} & TC & TA && {=^{\text{from}}_{p\bar{t}=\bar{g}}} & {T^2X} && TC \\
	& {T^2C} & {T^2A} &&&& {T^2C} &&&.
	\arrow["g", from=1-3, to=2-4]
	\arrow["x", from=2-2, to=1-3]
	\arrow[""{name=0, anchor=center, inner sep=0}, "Tg", from=2-2, to=3-3]
	\arrow["a"', from=3-3, to=2-4]
	\arrow["Tt"', from=2-2, to=3-2]
	\arrow[""{name=1, anchor=center, inner sep=0}, "Tp"', from=3-2, to=3-3]
	\arrow[""{name=2, anchor=center, inner sep=0}, "{\mu_X}", from=3-1, to=2-2]
	\arrow[""{name=3, anchor=center, inner sep=0}, "{T^2t}"', from=3-1, to=4-2]
	\arrow["{\mu_C}", from=4-2, to=3-2]
	\arrow["{\mu_A}"', from=4-3, to=3-3]
	\arrow["{\mu_X}", from=3-6, to=2-7]
	\arrow["x", from=2-7, to=1-8]
	\arrow["{T^2t}"', from=3-6, to=4-7]
	\arrow["{\mu_C}"', from=4-7, to=3-8]
	\arrow["c"', from=3-8, to=2-9]
	\arrow["t", from=1-8, to=2-9]
	\arrow["Tt"', from=2-7, to=3-8]
	\arrow["p", from=2-9, to=2-10]
	\arrow["{\bar{g}}"', shorten <=15pt, shorten >=15pt, Rightarrow, from=3-3, to=1-3]
	\arrow[""{name=4, anchor=center, inner sep=0}, "{T^2p}"', from=4-2, to=4-3]
	\arrow["{\bar{t}}"', shorten <=15pt, shorten >=15pt, Rightarrow, from=3-8, to=1-8]
	\arrow[shorten <=20pt, shorten >=20pt, Rightarrow, no head, from=4-7, to=2-7]
	\arrow[shorten <=1pt, shorten >=4pt, Rightarrow, no head, from=3-2, to=0]
	\arrow[shift right=2, shorten <=15pt, shorten >=9pt, Rightarrow, no head, from=3, to=2]
	\arrow[shorten <=12pt, shorten >=12pt, Rightarrow, no head, from=4, to=1]
\end{tikzcd}\]
A similar argument shows that the equality
\[\begin{tikzcd}[column sep=5mm, row sep=5mm]
	&& X & B &&&& X & B \\
	& TX && C & {=} && TX && C \\
	{T^2X} && TC && {} & {T^2X} && TC \\
	& {T^2C} &&&&& {T^2C}
	\arrow["Tx", from=3-1, to=2-2]
	\arrow["x", from=2-2, to=1-3]
	\arrow["t", from=1-3, to=2-4]
	\arrow["Tt", from=2-2, to=3-3]
	\arrow["c"', from=3-3, to=2-4]
	\arrow["{T^2t}"', from=3-1, to=4-2]
	\arrow["Tc"', from=4-2, to=3-3]
	\arrow["q"', from=2-4, to=1-4]
	\arrow["{\mu_X}", from=3-6, to=2-7]
	\arrow["x", from=2-7, to=1-8]
	\arrow["t", from=1-8, to=2-9]
	\arrow["Tt"', from=2-7, to=3-8]
	\arrow["c"', from=3-8, to=2-9]
	\arrow["{T^2t}"', from=3-6, to=4-7]
	\arrow["{\mu_C}"', from=4-7, to=3-8]
	\arrow["q"', from=2-9, to=1-9]
	\arrow["{T\bar{t}}"', shorten <=13pt, shorten >=13pt, Rightarrow, from=4-2, to=2-2]
	\arrow["{\bar{t}}"', shorten <=13pt, shorten >=13pt, Rightarrow, from=3-3, to=1-3]
	\arrow["{\bar{t}}"', shorten <=13pt, shorten >=13pt, Rightarrow, from=3-8, to=1-8]
	\arrow[shorten <=13pt, shorten >=13pt, Rightarrow, no head, from=4-7, to=2-7]
\end{tikzcd}\]
holds. From the uniqueness part of the $2$-dimensional universal property it follows that the equation
\[\begin{tikzcd}
	{T^2X} & TX & X & {} & {T^2X} & TX & X \\
	{T^2C} & TC & C & {} & {T^2C} & TC & C
	\arrow["Tx", from=1-1, to=1-2]
	\arrow["x", from=1-2, to=1-3]
	\arrow[""{name=0, anchor=center, inner sep=0}, "t", from=1-3, to=2-3]
	\arrow[""{name=1, anchor=center, inner sep=0}, "Tt"', from=1-2, to=2-2]
	\arrow["c"', from=2-2, to=2-3]
	\arrow[""{name=2, anchor=center, inner sep=0}, "{T^2t}"', from=1-1, to=2-1]
	\arrow["Tc"', from=2-1, to=2-2]
	\arrow["{\mu_X}", from=1-5, to=1-6]
	\arrow["{T^2t}"', from=1-5, to=2-5]
	\arrow["{\mu_C}"', from=2-5, to=2-6]
	\arrow["Tt"', from=1-6, to=2-6]
	\arrow["x", from=1-6, to=1-7]
	\arrow["t", from=1-7, to=2-7]
	\arrow["c"', from=2-6, to=2-7]
	\arrow["{=}"{description}, draw=none, from=1-4, to=2-4]
	\arrow["{T\bar{z}}", shorten <=20pt, shorten >=20pt, Rightarrow, from=2, to=1]
	\arrow["{\bar{z}}", shorten <=18pt, shorten >=18pt, Rightarrow, from=1, to=0]
\end{tikzcd}\]
holds. The unit axiom is again left as an exercise. It remains to check the $2$-dimensional universal property, so consider $\gamma, \delta$ $2$-cells in $T\Alg_l$ s.t.\
\[\begin{tikzcd}
	X &&&& X \\
	&& B & {=} &&& B \\
	A &&&& A
	\arrow[""{name=0, anchor=center, inner sep=0}, "g", curve={height=-12pt}, squiggly, from=1-1, to=3-1]
	\arrow[""{name=1, anchor=center, inner sep=0}, "{g'}"', curve={height=12pt}, squiggly, from=1-1, to=3-1]
	\arrow[""{name=2, anchor=center, inner sep=0}, "h", curve={height=-6pt}, squiggly, from=1-1, to=2-3]
	\arrow[""{name=3, anchor=center, inner sep=0}, "f"', curve={height=6pt}, squiggly, from=3-1, to=2-3]
	\arrow[""{name=4, anchor=center, inner sep=0}, "h", curve={height=-12pt}, squiggly, from=1-5, to=2-7]
	\arrow[""{name=5, anchor=center, inner sep=0}, "{h'}"', curve={height=12pt}, squiggly, from=1-5, to=2-7]
	\arrow["f"', curve={height=6pt}, squiggly, from=3-5, to=2-7]
	\arrow["{g'}"', squiggly, from=1-5, to=3-5]
	\arrow["\alpha", shorten <=12pt, shorten >=12pt, Rightarrow, from=2, to=3]
	\arrow["\gamma", shorten <=7pt, shorten >=7pt, Rightarrow, from=0, to=1]
	\arrow["\delta", shorten <=5pt, shorten >=5pt, Rightarrow, from=4, to=5]
	\arrow["{\alpha'}"', shorten <=18pt, shorten >=10pt, Rightarrow, from=5, to=3-5]
\end{tikzcd}\]
holds. The data of a $T$-transformation is just a $2$-cell in $\K$ which is compared and whiskered as in $\K$. From the universal property of $\varphi$ in $\K$ it follows that there is a unique $2$-cell $\varphi\colon t\Rightarrow t'$ with $p\varphi = \gamma$, $q\varphi=\delta$. It only remains to check that $\varphi$ is a $T$-transformation, i.e.\ that the equation
\[\begin{tikzcd}
	TX & X & {} & TX & X \\
	TC & C & {} & TC & C
	\arrow["x", from=1-1, to=1-2]
	\arrow[""{name=0, anchor=center, inner sep=0}, "Tt"', from=1-1, to=2-1]
	\arrow["c"', from=2-1, to=2-2]
	\arrow[""{name=1, anchor=center, inner sep=0}, "{t'}", curve={height=-12pt}, from=1-2, to=2-2]
	\arrow[""{name=2, anchor=center, inner sep=0}, "t"', curve={height=12pt}, from=1-2, to=2-2]
	\arrow["{=}"{description}, draw=none, from=1-3, to=2-3]
	\arrow["x", from=1-4, to=1-5]
	\arrow[""{name=3, anchor=center, inner sep=0}, "{t'}", from=1-5, to=2-5]
	\arrow[""{name=4, anchor=center, inner sep=0}, "{Tt'}", curve={height=-12pt}, from=1-4, to=2-4]
	\arrow["c"', from=2-4, to=2-5]
	\arrow[""{name=5, anchor=center, inner sep=0}, "Tt"', curve={height=12pt}, from=1-4, to=2-4]
	\arrow["{\bar{t}}", shorten <=12pt, shorten >=12pt, Rightarrow, from=0, to=2]
	\arrow["\varphi", shorten <=4pt, shorten >=4pt, Rightarrow, from=2, to=1]
	\arrow["T\varphi", shorten <=3pt, shorten >=6pt, Rightarrow, from=5, to=4]
	\arrow["{\bar{t'}}", shorten <=12pt, shorten >=12pt, Rightarrow, from=4, to=3]
\end{tikzcd}\]
holds. After whiskering with $p\colon C\to A$, the equation becomes
\[\begin{tikzcd}
	TX & X & {} & TX & X \\
	TA & A & {} & TA & A
	\arrow["x", from=1-1, to=1-2]
	\arrow[""{name=0, anchor=center, inner sep=0}, "Tg"', from=1-1, to=2-1]
	\arrow["a"', from=2-1, to=2-2]
	\arrow[""{name=1, anchor=center, inner sep=0}, "{g'}", curve={height=-12pt}, from=1-2, to=2-2]
	\arrow["{=}"{description}, draw=none, from=1-3, to=2-3]
	\arrow[""{name=2, anchor=center, inner sep=0}, "{Tg'}", curve={height=-12pt}, from=1-4, to=2-4]
	\arrow["x", from=1-4, to=1-5]
	\arrow["a"', from=2-4, to=2-5]
	\arrow[""{name=3, anchor=center, inner sep=0}, "{g'}", from=1-5, to=2-5]
	\arrow[""{name=4, anchor=center, inner sep=0}, "g"', curve={height=12pt}, from=1-2, to=2-2]
	\arrow[""{name=5, anchor=center, inner sep=0}, "Tg"', curve={height=12pt}, from=1-4, to=2-4]
	\arrow["{\bar{g}}", shorten <=12pt, shorten >=12pt, Rightarrow, from=0, to=4]
	\arrow["\gamma", shorten <=4pt, shorten >=4pt, Rightarrow, from=4, to=1]
	\arrow["T\gamma", shorten <=4pt, shorten >=4pt, Rightarrow, from=5, to=2]
	\arrow["{\bar{g'}}", shorten <=12pt, shorten >=12pt, Rightarrow, from=2, to=3]
\end{tikzcd}\]
which holds since $\gamma$ is a $T$-transformation. The equation also holds after whiskering with $q$ since $\delta$ is a $T$-transformation. Therefore $\varphi$ is indeed a $T$-transformation, which concludes the proof of the $2$-dimensional universal property. Finally, if $q$ and $h$ are strict $T$-morphisms, then the equation $p\cdot\bar{t}=\bar{g}$ and $q\cdot\bar{t}=\bar{h}$ implies that $\bar{t}=1$, i.e.\ $(t,\bar{t})$ is a strict $T$-morphism. 
\end{proof}
In any $2$-category $\K$ with colax limits of arrows, we get for each $f\colon A\to B$ with colax limit $(C_f,p_f,q_f,X)$ a unique $1$-cell $r_f\colon A\to C_f$ s.t.
\[\begin{tikzcd}
	& A &&&& A \\
	&&& {=} & {} & {C_f} \\
	A && B && A && B
	\arrow[Rightarrow, no head, from=1-2, to=3-1]
	\arrow[""{name=0, anchor=center, inner sep=0}, "f"', from=3-1, to=3-3]
	\arrow[""{name=1, anchor=center, inner sep=0}, "f", from=1-2, to=3-3]
	\arrow["{p_f}"', from=2-6, to=3-5]
	\arrow["{q_f}", from=2-6, to=3-7]
	\arrow[""{name=2, anchor=center, inner sep=0}, "f"', from=3-5, to=3-7]
	\arrow["{r_f}", from=1-6, to=2-6]
	\arrow[shift left=5, shorten <=8pt, shorten >=10pt, Rightarrow, no head, from=0, to=1]
	\arrow["\lambda"', shorten <=5pt, shorten >=5pt, Rightarrow, from=2-6, to=2]
\end{tikzcd}\]
holds. In particular, $q_fr_f=f$ and $p_fr_f=\id_f$.
\begin{prop}
    In the above situation, there exists a unique $2$-cell $\eta_f\colon\id_{C_f}\Rightarrow r_f\cdot p_f$ s.t.\ $p_f\eta_f=1, q_f\eta_f=\lambda$. This $2$-cell exhibits $r_f$ as right adjoint of $p_f$ with colimit the identity $p_f r_f=\id_A$.
    \end{prop}
    \begin{proof}
        Taking $\gamma=1_{p_f}\colon p_f\Rightarrow p_fr_fp_f$ and $\delta=\lambda\colon q_f\Rightarrow fpf=q_fr_fp_f$ we have
        \[\begin{tikzcd}
	A && {C_f} &&&& {C_f} \\
	&&&& {=} && A \\
	{C_f} & A && B && A && B
	\arrow["{p_f}"', from=3-1, to=3-2]
	\arrow[""{name=0, anchor=center, inner sep=0}, "{p_f}"', from=1-3, to=3-2]
	\arrow[""{name=1, anchor=center, inner sep=0}, "{q_f}", from=1-3, to=3-4]
	\arrow["f"', from=3-2, to=3-4]
	\arrow["{p_f}"', from=1-7, to=2-7]
	\arrow[Rightarrow, no head, from=2-7, to=3-6]
	\arrow[""{name=2, anchor=center, inner sep=0}, "f"', from=3-6, to=3-8]
	\arrow[""{name=3, anchor=center, inner sep=0}, "f", from=2-7, to=3-8]
	\arrow[""{name=4, anchor=center, inner sep=0}, "{q_f}", curve={height=-12pt}, from=1-7, to=3-8]
	\arrow["{p_f}"', from=1-3, to=1-1]
	\arrow[""{name=5, anchor=center, inner sep=0}, "{r_f}"', from=1-1, to=3-1]
	\arrow[shorten <=25pt, shorten >=25pt, Rightarrow, no head, from=5, to=0]
	\arrow["\lambda", shorten <=23pt, shorten >=23pt, Rightarrow, from=1, to=3-2]
	\arrow[shift left=4, shorten <=5pt, shorten >=11pt, Rightarrow, no head, from=2, to=3]
	\arrow["\lambda"', shift right=4, shorten <=5pt, shorten >=8pt, Rightarrow, from=4, to=3]
\end{tikzcd}\]
\[\begin{tikzcd}
	&& {C_f} \\
	&& A \\
	{=} && {C_f} \\
	& A && B
	\arrow["{p_f}"', from=1-3, to=2-3]
	\arrow["{r_f}"', from=2-3, to=3-3]
	\arrow["{p_f}"', from=3-3, to=4-2]
	\arrow[""{name=0, anchor=center, inner sep=0}, "f"', from=4-2, to=4-4]
	\arrow[""{name=1, anchor=center, inner sep=0}, "{q_f}", from=3-3, to=4-4]
	\arrow[""{name=2, anchor=center, inner sep=0}, "f"', curve={height=-12pt}, from=2-3, to=4-4]
	\arrow[""{name=3, anchor=center, inner sep=0}, "{q_f}", curve={height=-18pt}, from=1-3, to=4-4]
	\arrow["\lambda", shift right=2, shorten <=1pt, shorten >=5pt, Rightarrow, from=3, to=2]
	\arrow[shorten <=9pt, Rightarrow, no head, from=2, to=3-3]
	\arrow["\lambda"', shift right=3, shorten <=5pt, shorten >=5pt, Rightarrow, from=1, to=0]
\end{tikzcd}\]
so there exists a unique $2$-cell $\eta_f\colon\id_{C_f}\Rightarrow r_f\cdot p_f$ with $p_f\cdot\eta_f=1, q_f\eta_f=\lambda$ by the $2$-dimensional universal property. It remains to show that the triangle identities hold. Since $\epsilon=1$ these become $p_f\eta_f=1$ and $\eta_f r_f=1$. So one of these we already checked. For the second it suffices to check that it holds after whiskering with $p_f$ and $\eta_f$, where we get $p_f\eta_f r_f=1$ and $q_f\eta_f r_f=\lambda r_f=1$ (by def of $r_f$) and $p_f\eta_f=1$ by definition. 
\end{proof}
A right adjoint $r$ with counit the identity is sometimes called a RARI (Right Adjoint Right Inverse). The corresponding left adjoint is called a LALI (Left Adjoint Left Inverse). The dual concepts (with unit the identity) are called RALI and LARI. For $T\Alg_p$ we can work instead with pseudolimits of arrows, which is the universal  \[\begin{tikzcd}
	& {P_f} \\
	A && B.
	\arrow["{p_f}"', from=1-2, to=2-1]
	\arrow["{q_f}", from=1-2, to=2-3]
	\arrow[""{name=0, anchor=center, inner sep=0}, "f"', from=2-1, to=2-3]
	\arrow["\sim", shorten <=5pt, shorten >=5pt, Rightarrow, from=1-2, to=0]
\end{tikzcd}\]
\begin{prop}
    The forgetful $2$-functor $U_p\colon T\Alg_p\to\K$ creates pseudolimits of arrows.
\end{prop} 
\begin{proof}
    The same construction\footnote{In part $p_f,q_f$ strict!} as in the case of $T\Alg_l$ works, we just have to observe that $\bar{t}$ is an isomorphism, which follows from $p\bar{t}=\bar{g}$ and $q\bar{t}=\bar{h}$ and the fact that those are isomorphisms, since $f$ and $g$ are pseudomorphisms and $p,q$ jointly detect isos.
\end{proof}
\begin{prop}
    If $\K$ has pseudolimits of arrows and $(f,\bar{f})\colon A\rightsquigarrow B$ is a pseudo $T$-morphism, then there exists a unique $r_f\colon A\rightsquigarrow P_f$ such that 
    \[\begin{tikzcd}
	&&&&& A \\
	& A && {} && {P_f} \\
	A && B & {} & A & {} & B
	\arrow[Rightarrow, no head, from=2-2, to=3-1]
	\arrow["f", squiggly, from=2-2, to=3-3]
	\arrow["f"', squiggly, from=3-1, to=3-3]
	\arrow["{=}"{description}, draw=none, from=2-4, to=3-4]
	\arrow["r_f", squiggly, from=1-6, to=2-6]
	\arrow["{p_f}"', from=2-6, to=3-5]
	\arrow["f"', squiggly, from=3-5, to=3-7]
	\arrow["{q_f}", from=2-6, to=3-7]
	\arrow[curve={height=12pt}, Rightarrow, no head, from=1-6, to=3-5]
	\arrow["f", curve={height=-12pt}, squiggly, from=1-6, to=3-7]
	\arrow["\sim", "\lambda"', shorten <=7pt, shorten >=4pt, Rightarrow, from=2-6, to=3-6]
\end{tikzcd}\]
    and an invertible $\eta_f\colon 1\Rightarrow r_f p_f$ s.t.\ $(r_f,p_f,\eta_f,1)$ is an adjoint equivalence.
\end{prop}
\begin{proof}
    Existence of $\eta_f$ and triangle identities follow as before. Moreover, $\eta_f$ is invertible since both $p_f\eta_f=1$ and $q_f\eta_f=\lambda$ are invertible and $p_f, q_f$ jointly detect isos.
\end{proof}
In particular, we can replace (up to equivalence) a pseudo $T$-morphism by a strict $T$-morphism 
\[\begin{tikzcd}
	& {P_f} \\
	A && B
	\arrow["{r_f}", squiggly, from=2-1, to=1-2]
	\arrow["f"', squiggly, from=2-1, to=2-3]
	\arrow[""{name=0, anchor=center, inner sep=0}, "{q_f}", from=1-2, to=2-3]
	\arrow["\simeq"{pos=0.7}, shift right=2, Rightarrow, draw=none, from=2-1, to=0]
\end{tikzcd}\]
of path-spaces.
With this at hand we can prove the following theorem, which is useful for constructing $2$-monads via presentations. Specifically, for identifying the pseudo and lax $T$-morphisms of such $2$-monads.
\begin{thm}\label{unique ext}
    Let $S$ and $T$ be $2$-monads on a $2$-category with colax limits of arrows. Let $F_s\colon T\Alg_s\to S\Alg_s$ be a (strict) $2$-functor such that the triangle \[\begin{tikzcd}
	{T\Alg_s} && {S\Alg_s} \\
	& \K
	\arrow["{U_t}"', from=1-1, to=2-2]
	\arrow["{F_s}", from=1-1, to=1-3]
	\arrow["{U_s}", from=1-3, to=2-2]
\end{tikzcd}\]
commutes. Then there exists a unique $2$-functor $F_l\colon T\Alg_l\to S\Alg_l$ s.t.\ the diagram
\[\begin{tikzcd}
	{T\Alg_s} && {S\Alg_s} \\
	{T\Alg_l} && {S\Alg_l} \\
	& \K
	\arrow["{F_s}", from=1-1, to=1-3]
	\arrow["J"', from=1-1, to=2-1]
	\arrow["J", from=1-3, to=2-3]
	\arrow["{F_l}", from=2-1, to=2-3]
	\arrow["{U_l}"', from=2-1, to=3-2]
	\arrow["{U_l}", from=2-3, to=3-2]
\end{tikzcd}\]
commutes.
\end{thm}
\begin{proof}
    For the existence note that $F_s$ is induced by a (unique) $2$-monad morphism $\varphi\colon S\to T$ s.t.\ the semantics $1$-functor $$(-)\Alg\colon2\mbox{-}\Mnd(\K)^{\op}\to 2\mbox{-}\Cat/\K$$
is full and faithful. This can be used to define $F_l$ as follows. We send
\[\begin{tikzcd}
	TA & A \\
	TB & B
	\arrow["a", from=1-1, to=1-2]
	\arrow[""{name=0, anchor=center, inner sep=0}, "Tf"', from=1-1, to=2-1]
	\arrow["b"', from=2-1, to=2-2]
	\arrow[""{name=1, anchor=center, inner sep=0}, "f", from=1-2, to=2-2]
	\arrow["{\bar{f}}", shorten <=17pt, shorten >=17pt, Rightarrow, from=0, to=1]
\end{tikzcd}\]
to 
\[\begin{tikzcd}
	SA & TA & A \\
	SB & TB & B
	\arrow["{\varphi_A}", from=1-1, to=1-2]
	\arrow["Sf"', from=1-1, to=2-1]
	\arrow["{\varphi_B}"', from=2-1, to=2-2]
	\arrow[""{name=0, anchor=center, inner sep=0}, "Tf"', from=1-2, to=2-2]
	\arrow["a", from=1-2, to=1-3]
	\arrow["b"', from=2-2, to=2-3]
	\arrow[""{name=1, anchor=center, inner sep=0}, "f", from=1-3, to=2-3]
	\arrow["{\bar{f}}", shorten <=17pt, shorten >=17pt, Rightarrow, from=0, to=1]
\end{tikzcd}\]
and we let $F_l$ be the identity on $2$-cells. The interesting part is the converse. Since the inclusions $J$ are bijective on objects, $F_l$ is uniquely determined on $0$-cells. The two $2$-functors $U_l\colon T\Alg\to\K$ and $U_l\colon S\Alg_l\to\K$ are both injective on $2$-cells, so $F_l$ is also uniquely determined on $2$-cells. It remains to show uniqueness on $1$-cells. So let $(f,\bar{f})\colon (A,a)\rightsquigarrow (B,b)$ be a $1$-cell in $T\Alg_l$. Since we have colax limits of arrows in $\K$, we can factor $(f,\bar{f})$ as follows
 \[\begin{tikzcd}
	&&&&& A \\
	& A && {} && {C_f} \\
	A && B & {} & A & {} & B.
	\arrow[Rightarrow, no head, from=2-2, to=3-1]
	\arrow["f", squiggly, from=2-2, to=3-3]
	\arrow["f"', squiggly, from=3-1, to=3-3]
	\arrow["{=}"{description}, draw=none, from=2-4, to=3-4]
	\arrow["r_f", squiggly, from=1-6, to=2-6]
	\arrow["{p_f}"', from=2-6, to=3-5]
	\arrow["f"', squiggly, from=3-5, to=3-7]
	\arrow["{q_f}", from=2-6, to=3-7]
	\arrow[curve={height=12pt}, Rightarrow, no head, from=1-6, to=3-5]
	\arrow["f", curve={height=-12pt}, squiggly, from=1-6, to=3-7]
	\arrow["\lambda"', shorten <=7pt, shorten >=4pt, Rightarrow, from=2-6, to=3-6]
\end{tikzcd}\]
It follows that $F_l(f,\bar{f})=F_l(q_f)\circ F_l(r_f)$. Since the square in the diagram commutes, $F_l(q_f)=F_s(q_f)$, so it only remains to show that $F_l(r_f)$ is uniquely determined. We also know that $(r_f, p_f, \eta_f,1)$ is an adjunction, so since $F_l$ is a $2$-functor it follows that $(F_l(r_f), F_l(p_f), F_l(\eta_f), 1)$ is an adjunction in $S\Alg_s$. Since $p_f$ is also strict, we have $F_l(p_f)=F_s(p_f)$. To summarize: $F_l(r_f)$ is a lax $T$-morphism structure on $U_lF_l(r_f)=U_l(r_f)$ so that $\eta_f$ and $1$ make it a right adjoint of $F_l(p_f)$ in $S\Alg_l$. From the uniqueness part of doctrinal adjunction it follows that $F_l(r_f)$ is uniquely determined by $F_s(p_f), \eta_f, 1$. 
\end{proof}  
\begin{rmk}
    There is an analogous statement for $T\Alg_p$ using the pseudolimit of arrows (assuming they exist in $\K$). Why is this useful? When dealing with monads given by presentations, we will (by construction) have a $2$-functor $F_s\colon T(G)\Alg_s\to T(F)\Alg_s$, so a corresponding monad morphism $T(F)\to T(G)$, whenever $T(F), T(G)$ are free $2$-monads on endofunctors $F, G$. So this corresponds to a $2$-natural $F\to T(G)$, but it is in general hard to describe this explicitly. If we want to figure out what happens on lax morphisms from the definition, we would need to understand this instead. Usually it is easy to guess a $2$-functor $F_l$ that makes everything commute. This assumes that we have a description of $T(F)\Alg_l$ purely in terms of $F$, which is indeed possible as we will see next.
\end{rmk}
\begin{defn}
    Let $F\colon\K\to\K$ be a $2$-functor. An $F$-algebra is a pair $(A,a)$ with $a\colon FA\to A$ a $1$-cell in $\K$ with no axioms.
    Strict morphisms $f\colon(A,a)\to(B,b)$ are $1$-cells $f\colon A\to B$ s.t.\ $b Ff=fa$. A lax $F$-morphism is a pair $(f,\bar{f})$ of a $1$-cell $f\colon A\to B$ and a $2$-cell 
    \[\begin{tikzcd}
	FA & A \\
	FB & B
	\arrow["a", from=1-1, to=1-2]
	\arrow[""{name=0, anchor=center, inner sep=0}, "Ff"', from=1-1, to=2-1]
	\arrow["b"', from=2-1, to=2-2]
	\arrow[""{name=1, anchor=center, inner sep=0}, "f", from=1-2, to=2-2]
	\arrow["{\bar{f}}", shorten <=17pt, shorten >=17pt, Rightarrow, from=0, to=1]
\end{tikzcd}\]
subject to no axioms. An $F$-transformation $\rho\colon(f,\bar{f})\Rightarrow(g,\bar{g})$ is a $2$-cell $\rho\colon f\Rightarrow g$ s.t.\ the equation
\[\begin{tikzcd}
	FA & A & {} & FA & A \\
	FB & B & {} & FB & B
	\arrow["a", from=1-1, to=1-2]
	\arrow[""{name=0, anchor=center, inner sep=0}, "Ff"', from=1-1, to=2-1]
	\arrow["b"', from=2-1, to=2-2]
	\arrow[""{name=1, anchor=center, inner sep=0}, "{g}", curve={height=-12pt}, from=1-2, to=2-2]
	\arrow[""{name=2, anchor=center, inner sep=0}, "f"', curve={height=12pt}, from=1-2, to=2-2]
	\arrow["{=}"{description}, draw=none, from=1-3, to=2-3]
	\arrow["a", from=1-4, to=1-5]
	\arrow[""{name=3, anchor=center, inner sep=0}, "{g}", from=1-5, to=2-5]
	\arrow[""{name=4, anchor=center, inner sep=0}, "{Fg}", curve={height=-12pt}, from=1-4, to=2-4]
	\arrow["b"', from=2-4, to=2-5]
	\arrow[""{name=5, anchor=center, inner sep=0}, "Ff"', curve={height=12pt}, from=1-4, to=2-4]
	\arrow["{\bar{f}}", shorten <=12pt, shorten >=12pt, Rightarrow, from=0, to=2]
	\arrow["\rho", shorten <=4pt, shorten >=4pt, Rightarrow, from=2, to=1]
	\arrow["F\rho", shorten <=3pt, shorten >=6pt, Rightarrow, from=5, to=4]
	\end{tikzcd}\]
	holds. We write $F\Alg$ for the resulting $2$-category. A pseudo $F$-morphism is an $(f,\bar{f})$ s.t.\ $\bar{f}$ is invertible and we write $F\Alg_p$ for the corresponding $2$-category.
\end{defn}

As in the 1-dimensional case, we can relate $F$-algebras and $T(F)$-algebras.

\begin{prop}
    Let $\cK$ be a locally presentable 2-category, $F$ a $\kappa$-accessible
    2-endofunctor on $\cK$, $T(F)$ the free $\kappa$-accessible monad on $F$
    with universal 2-natural transformation $\psi\colon F\rightarrow T(F)$. We
    can then construct isomorphisms of categories
    \[\psi^*\colon T(F)\Alg_L\rightarrow F\Alg_L\]
    \[\psi^*\colon T(F)\Alg_P\rightarrow F\Alg_P\]
    by whiskering with $\psi$.
\end{prop}
\begin{proof}
    It is clear that $FA\xrightarrow{\psi_a}T(F)A\xrightarrow{a}A$ is a
    $F$-algebra for any $T(F)$-algebra $(A,a)$ and, for any lax $T(T)$-morphism
    $(f,\overline{f})$, the 2-cell
    \[\begin{tikzcd}
        FA & {T(F)A} & A \\
        FB & {T(F)B} & B
        \arrow["b"', from=2-2, to=2-3]
        \arrow[""{name=0, anchor=center, inner sep=0}, "f"', from=1-3, to=2-3]
        \arrow["a", from=1-2, to=1-3]
        \arrow[""{name=1, anchor=center, inner sep=0}, "{T(F)f}"{description}, from=1-2, to=2-2]
        \arrow["Ff"', from=1-1, to=2-1]
        \arrow["{\psi_b}"', from=2-1, to=2-2]
        \arrow["{\psi_a}", from=1-1, to=1-2]
        \arrow[shorten <=8pt, shorten >=8pt, Rightarrow, no head, from=2-1, to=1-2]
        \arrow["{\overline{f}}", shorten <=14pt, shorten >=14pt, Rightarrow, from=1, to=0]
    \end{tikzcd}\]
    is a lax $F$-morphism. Since composition of 1-cells in both $F\Alg_L$ and
    $T(F)\Alg_L$ is defined by attaching these 2-cells, this defines a functor
    on the underlying 1-categories.

    Since $\psi$ is 2-natural, the axiom for a $T(F)$-transformation turns into
    the axiom for a $F$-transformation, hence we can extend this to a 2-functor
    by acting as the identity on 2-cells.

    It remains to show that this defines an isomorphism of 2-categories, or
    equivalently that it is a bijection on 0, 1 and 2-cells, which follows from
    the universal property of $\psi$.

    Since $\psi^*$ preserves the underlying 0, 1 and 2-cells we only need to
    check the bijection for a fixed underlying cell. In this case, the claim
    follows from the existence of the 2-monads $<A,A>$, $\{f,f\}_L$ and
    $[\rho,\rho]$. Namely, whiskering with $\psi$ gives a bijection between
    2-monad morphisms $T(F)\rightarrow<A,A>$ and mere 2-natural transformations
    $F\Rightarrow <A,A>$. By adjunction, this corresponds to $a\colon
    FA\rightarrow A$, subject to no axioms. The bijection on 1 and 2-cells
    follows analogously, as proof concerning $T(F)\Alg_P$ and $F\Alg_P$.
\end{proof}

We can use this to identify $T\Alg_L$ when $T$ is given via a presentation
through the following procedure. We start with various (accessible)
2-endofunctors $F,G\ldots$ on $\cK$ and we construct 2-functors
$F\Alg_S\rightarrow G\Alg_S$, etc. These are induced by monad morphisms
$T(G)\rightarrow T(F)$ and if we want to know what happens on lax and pseudo
morphisms we use \ref{unique ext}.

Taking limits, we obtain new categories which are of the form $T\Alg_S$ for the
corresponding category of monads. We can then iterate this by considering
2-functors $T\Alg_L\rightarrow W\Alg_S$ for a 2-endofunctor $W$ on $\cK$.

To do this we need one more ingredient in order to identify the 2-category
$(W\odot D)\Alg_{S/L/P}$ for any small diagram $D\colon\cA^{\op}\rightarrow
2\Mnd_{\kappa}(\cK)$ and any weight $W\colon\cA^{\op}\rightarrow\Cat$.

For $T\Alg_L$, this comes from the corresponding limit of 2-categories
$\{W,D\Alg_L\}$. To show it we first need to turn $(-)\Alg_L$ into a 2-functor.
\[\begin{tikzcd}
	\cC & \cD
	\arrow[""{name=0, anchor=center, inner sep=0}, "G"', curve={height=12pt}, from=1-1, to=1-2]
	\arrow[""{name=1, anchor=center, inner sep=0}, "F", curve={height=-12pt}, from=1-1, to=1-2]
	\arrow["\alpha", shorten <=6pt, shorten >=6pt, Rightarrow, from=1, to=0]
\end{tikzcd}\]
s.t.\
\[\begin{tikzcd}
	\cC & \cD & \cK & {=} & {\id_{U^{\cC}}}
	\arrow[""{name=0, anchor=center, inner sep=0}, "G"', curve={height=12pt}, from=1-1, to=1-2]
	\arrow[""{name=1, anchor=center, inner sep=0}, "F", curve={height=-12pt}, from=1-1, to=1-2]
	\arrow["{U^{\cD}}", from=1-2, to=1-3]
	\arrow["\alpha", shorten <=6pt, shorten >=6pt, Rightarrow, from=1, to=0]
\end{tikzcd}\]

We now need to extend $(-)\Alg_L$ to a 2-functor.

Recall that a monad modification $\alpha\colon\phi\Rrightarrow\psi$ between
monad morphisms is a modification subject to two axioms.

The datum of a modification of 2-monads consists of a 2-cell $\alpha_A$ for each
0-cell $A\in\cK$ and the axioms state that the equations
\[\begin{tikzcd}
	SSA & STA & TTA & A & {=} & SSA & SA & TA
	\arrow[""{name=0, anchor=center, inner sep=0}, "{S\phi_A}", curve={height=-12pt}, from=1-1, to=1-2]
	\arrow[""{name=1, anchor=center, inner sep=0}, "{S\psi_A}"', curve={height=12pt}, from=1-1, to=1-2]
	\arrow[""{name=2, anchor=center, inner sep=0}, "{\phi_{TA}}", curve={height=-12pt}, from=1-2, to=1-3]
	\arrow[""{name=3, anchor=center, inner sep=0}, "{\psi_{TA}}"', curve={height=12pt}, from=1-2, to=1-3]
	\arrow["{\mu^T_A}", from=1-3, to=1-4]
	\arrow["{\mu^S_A}", from=1-6, to=1-7]
	\arrow[""{name=4, anchor=center, inner sep=0}, "{\phi_A}", curve={height=-12pt}, from=1-7, to=1-8]
	\arrow[""{name=5, anchor=center, inner sep=0}, "{\psi_A}"', curve={height=12pt}, from=1-7, to=1-8]
	\arrow["{\alpha_{TA}}", shorten <=6pt, shorten >=6pt, Rightarrow, from=2, to=3]
	\arrow["{S\alpha_A}", shorten <=6pt, shorten >=6pt, Rightarrow, from=0, to=1]
	\arrow["{\alpha_A}"', shorten <=6pt, shorten >=6pt, Rightarrow, from=4, to=5]
\end{tikzcd}\]
\[\begin{tikzcd}
    A & SA & TA & {=} & {1_{\eta_A^T}}
	\arrow["{\eta_A^S}", from=1-1, to=1-2]
	\arrow[""{name=0, anchor=center, inner sep=0}, "{\phi_A}", curve={height=-12pt}, from=1-2, to=1-3]
	\arrow[""{name=1, anchor=center, inner sep=0}, "{\psi_A}"', curve={height=12pt}, from=1-2, to=1-3]
	\arrow["{\alpha_A}", shorten <=6pt, shorten >=6pt, Rightarrow, from=0, to=1]
\end{tikzcd}\]
hold, plus the modification axioms.

We want to finish extending $(-)\Alg_L$ to a 2-functor
$2-\Mnd_\kappa(\cK)^{\text{coop}}\rightarrow 2-\CAT/\cK$, where the target has
the 2-cells specified above, hence we have to define a 2-natural transformation
$\alpha^*\colon\psi^*\Rightarrow\phi^*$ s.t.\ $U_L\alpha^*=1$. Giving a
2-natural transformation means giving a 1-cell in $S\Alg_L$ for each 0-cell in
$T\Alg_L$, i.e.\ for each $T$-algebra we have to specify a lax $S$-morphism.

We do it as follows: given $(A,a)\in T\Alg_L$, we let $(\alpha^*)_{(A,a)}$ be
the lax $S$-morphism
\[\begin{tikzcd}
	SA & TA & A \\
	SA & TA & A
	\arrow[""{name=0, anchor=center, inner sep=0}, Rightarrow, no head, from=1-1, to=2-1]
	\arrow[""{name=1, anchor=center, inner sep=0}, Rightarrow, no head, from=1-2, to=2-2]
	\arrow[Rightarrow, no head, from=1-3, to=2-3]
	\arrow["a", from=1-2, to=1-3]
	\arrow["a"', from=2-2, to=2-3]
	\arrow["{\psi_A}", from=1-1, to=1-2]
	\arrow["{\phi_A}"', from=2-1, to=2-2]
	\arrow[shorten <=8pt, shorten >=8pt, Rightarrow, no head, from=2-2, to=1-3]
	\arrow["{\alpha_A}", shorten <=13pt, shorten >=13pt, Rightarrow, from=0, to=1]
\end{tikzcd}\]
with the identity as underlying 1-cell.

\begin{prop}
    The assignment $\alpha\mapsto\alpha^*$ is well-defined and thus $(-)\Alg_L$
    gives a 2-functor
    \[2-\Mnd_\kappa(\cK)^{\text{coop}}\rightarrow 2-\CAT/\cK\]
\end{prop}
\begin{proof}
    There are a few things to check. We leave some as exercises.

    We start with one of the lax morphism axioms. We want to show that
    \[\begin{tikzcd}
        {(1)} & SSA & STA & SA & TA & A \\
        & SSA & STA & SA & TA & A \\
        &&& {=} \\
        {(2)} & SSA & SA & TA & A \\
        & SSA & SA & TA & A
        \arrow[""{name=0, anchor=center, inner sep=0}, Rightarrow, no head, from=1-4, to=2-4]
        \arrow[""{name=1, anchor=center, inner sep=0}, Rightarrow, no head, from=1-5, to=2-5]
        \arrow[Rightarrow, no head, from=1-6, to=2-6]
        \arrow["a", from=1-5, to=1-6]
        \arrow["a"', from=2-5, to=2-6]
        \arrow["{\psi_A}", from=1-4, to=1-5]
        \arrow["{\phi_A}"', from=2-4, to=2-5]
        \arrow["{S\psi_A}", from=1-2, to=1-3]
        \arrow["{S\phi_A}"', from=2-2, to=2-3]
        \arrow["Sa"', from=2-3, to=2-4]
        \arrow["Sa", from=1-3, to=1-4]
        \arrow[""{name=2, anchor=center, inner sep=0}, Rightarrow, no head, from=1-2, to=2-2]
        \arrow[""{name=3, anchor=center, inner sep=0}, Rightarrow, no head, from=1-3, to=2-3]
        \arrow[Rightarrow, no head, from=4-5, to=5-5]
        \arrow[""{name=4, anchor=center, inner sep=0}, Rightarrow, no head, from=4-4, to=5-4]
        \arrow[""{name=5, anchor=center, inner sep=0}, Rightarrow, no head, from=4-3, to=5-3]
        \arrow[Rightarrow, no head, from=4-2, to=5-2]
        \arrow["{\mu_A^S}", from=4-2, to=4-3]
        \arrow["{\psi_A}", from=4-3, to=4-4]
        \arrow["a", from=4-4, to=4-5]
        \arrow["a"', from=5-4, to=5-5]
        \arrow["{\psi_A}"', from=5-3, to=5-4]
        \arrow["{\mu^S_A}"', from=5-2, to=5-3]
        \arrow["{\alpha_A}", shorten <=13pt, shorten >=13pt, Rightarrow, from=0, to=1]
        \arrow["{S\alpha_A}", shorten <=13pt, shorten >=13pt, Rightarrow, from=2, to=3]
        \arrow["{\alpha_A}", shorten <=13pt, shorten >=13pt, Rightarrow, from=5, to=4]
    \end{tikzcd}\]

    Using a modification axiom,
    \[\begin{tikzcd}
        {(1)} & {=} & SSA & STA & TTA & TA & A \\
        && SSA & STA & TTA & TA & A \\
        &&&&& TA
        \arrow[""{name=0, anchor=center, inner sep=0}, Rightarrow, no head, from=1-5, to=2-5]
        \arrow[Rightarrow, no head, from=1-6, to=2-6]
        \arrow[Rightarrow, no head, from=1-7, to=2-7]
        \arrow["a", from=1-6, to=1-7]
        \arrow["a", from=2-6, to=2-7]
        \arrow["Ta", from=1-5, to=1-6]
        \arrow["Ta", from=2-5, to=2-6]
        \arrow["{S\psi_A}", from=1-3, to=1-4]
        \arrow["{S\phi_A}"', from=2-3, to=2-4]
        \arrow["Sa"', from=2-4, to=2-5]
        \arrow["Sa", from=1-4, to=1-5]
        \arrow[""{name=1, anchor=center, inner sep=0}, Rightarrow, no head, from=1-3, to=2-3]
        \arrow[""{name=2, anchor=center, inner sep=0}, Rightarrow, no head, from=1-4, to=2-4]
        \arrow["{\mu_a}"', from=2-5, to=3-6]
        \arrow["a"', from=3-6, to=2-7]
        \arrow["{S\alpha_A}", shorten <=13pt, shorten >=13pt, Rightarrow, from=1, to=2]
        \arrow["{\alpha_{TA}}", shorten <=13pt, shorten >=13pt, Rightarrow, from=2, to=0]
    \end{tikzcd}\]
    and now we apply a monad modification axiom to find that this is equal to
    \[\begin{tikzcd}
        SSA & SA & TA & A
        \arrow["a", from=1-3, to=1-4]
        \arrow[""{name=0, anchor=center, inner sep=0}, "{\phi_A}", curve={height=-12pt}, from=1-2, to=1-3]
        \arrow[""{name=1, anchor=center, inner sep=0}, "{\psi_A}"', curve={height=12pt}, from=1-2, to=1-3]
        \arrow["{\mu_A^S}", from=1-1, to=1-2]
        \arrow["{\alpha_A}", shorten <=6pt, shorten >=6pt, Rightarrow, from=0, to=1]
    \end{tikzcd},\]
    which we can rewrite as (2). We leave the second axiom as an exercise.

    Next we check the 2-naturality of $\alpha^*$. For the 1-cell axiom, we need
    to consider a 1-cell $(f,\overline{f}\colon(A,a)\rightarrow(B,b)$ in
    $T\Alg_L$. Then we have
    \[\begin{tikzcd}
        \bullet && SA & TA & A && SA & TA & A \\
        \bullet & {=} & SA & TA & A & {=} & SB & TB & B \\
        \bullet && SB & TB & B \\
        && SA & TA & A && \bullet \\
        & {=} & SB & TB & B & {=} & \bullet \\
        && SB & TB & B && \bullet
        \arrow["{\alpha^*_{(A,a)}}", from=1-1, to=2-1]
        \arrow["{\phi^*_{(f,\overline{f})}}", from=2-1, to=3-1]
        \arrow["Sf"', from=2-3, to=3-3]
        \arrow[Rightarrow, no head, from=1-3, to=2-3]
        \arrow[""{name=0, anchor=center, inner sep=0}, "Tf"', from=2-4, to=3-4]
        \arrow[""{name=1, anchor=center, inner sep=0}, "f"', from=2-5, to=3-5]
        \arrow[Rightarrow, no head, from=1-5, to=2-5]
        \arrow[Rightarrow, no head, from=1-4, to=2-4]
        \arrow["a", from=2-4, to=2-5]
        \arrow["a", from=1-4, to=1-5]
        \arrow["b"', from=3-4, to=3-5]
        \arrow["{\phi_B}"', from=3-3, to=3-4]
        \arrow["{\phi_A}", from=2-3, to=2-4]
        \arrow["{\psi_A}", from=1-3, to=1-4]
        \arrow[""{name=2, anchor=center, inner sep=0}, "f", from=1-9, to=2-9]
        \arrow[""{name=3, anchor=center, inner sep=0}, "Tf"{description}, from=1-8, to=2-8]
        \arrow["b"', from=2-8, to=2-9]
        \arrow["a", from=1-8, to=1-9]
        \arrow["{\psi_A}", from=1-7, to=1-8]
        \arrow["Sf"', from=1-7, to=2-7]
        \arrow[""{name=4, anchor=center, inner sep=0}, "{\psi_B}", curve={height=-12pt}, from=2-7, to=2-8]
        \arrow[""{name=5, anchor=center, inner sep=0}, "{\phi_B}"', curve={height=12pt}, from=2-7, to=2-8]
        \arrow["f", from=4-5, to=5-5]
        \arrow[Rightarrow, no head, from=5-5, to=6-5]
        \arrow["b", from=5-4, to=5-5]
        \arrow["b"', from=6-4, to=6-5]
        \arrow["a", from=4-4, to=4-5]
        \arrow["Tf", from=4-4, to=5-4]
        \arrow[Rightarrow, no head, from=5-4, to=6-4]
        \arrow["{\phi_B}"', from=6-3, to=6-4]
        \arrow["{\psi_B}"', from=5-3, to=5-4]
        \arrow["{\psi_A}", from=4-3, to=4-4]
        \arrow["Sf"', from=4-3, to=5-3]
        \arrow[Rightarrow, no head, from=5-3, to=6-3]
        \arrow["{\psi^*_{(f,\overline{f})}}", from=4-7, to=5-7]
        \arrow["{\alpha^*_{(B,b)}}", from=5-7, to=6-7]
        \arrow["{\overline{f}}", shorten <=13pt, shorten >=13pt, Rightarrow, from=0, to=1]
        \arrow["{\overline{f}}", shorten <=13pt, shorten >=13pt, Rightarrow, from=3, to=2]
        \arrow["{\alpha_B}"', shorten <=6pt, shorten >=6pt, Rightarrow, from=5, to=4]
    \end{tikzcd}\]
    which shows the 1-cell part of the 2-naturality condition. We leave the
    2-cell part of 2-naturality as an exercise.

    By construction, we have $U^L\alpha^*_{(A,a)}=1_A$, so this really is a
    2-cell in $2\mbox{-}\CAT/\cK$. This shows that this assignment extends to a
    2-functor if we can prove that composition and whiskering operations for
    monad modifications turn into the corresponding operations in $2\mbox{-}\CAT/\cK$,
    which follows from the definition of composition and whiskering for
    modifications.
\end{proof}

\begin{rmk}
    For $T\Alg_p$ we only have 2-naturality for invertible modifications.
\end{rmk}

Next we want to check that $(-)\Alg_l$ turns weighted colimits into weighted
limits. For this we use the following characterization of $\langle A,A\rangle$, $\{f,f\}_l$
and $[\rho,\rho]$.

\begin{prop}
    Let $\cK$ be complete and $A\in\cK$. Then there is an isomorphism of
    categories
    \[\Mnd(\cK)^{\text{co}}(T,\langle A,A \rangle)\rightarrow
    2\mbox{-}\CAT/\cK(\bb1\xrightarrow{A}\cK,T\Alg_l\xrightarrow{U_l}\cK)\]
    which is 2-natural in $T$.
\end{prop}
\begin{prop}
The $2$-category $2\mbox{-}\CAT/\cK$ is complete as a $\Cat$-enriched category.
\end{prop}
\begin{proof}
    For completeness we need conical limits and powers by $\b2=\{0\to1\}$. We start with the latter. It is given by the pullback in $2\mbox{-}\CAT$ 
    
    \[\begin{tikzcd}
\b2\pitchfork U \arrow[r] \arrow[d, "V"'] \arrow[rd, "\lrcorner", near start, phantom] & \C^{\b2} \arrow[d, "U^{\b2}"] \\
\cK \arrow[r, "\ulcorner\id\urcorner"']                                    & \K^{\b2}                     
\end{tikzcd}\]
where $\ulcorner\id\urcorner$ classifies the identity $2$-cell on $\id_{\cK}$. Note that there is a $2$-dimensional aspect to this, which follows from the $2$-dimensional universal property of $\C^{\b2}$. We also have copowers by $\b2$ given by $\C\times\b2\xrightarrow{\text{pr}}\C\xrightarrow{U}\cK$, so we only need to check the $1$-dimensional universal property for conical limits. Conical limits are classical: products are given by ``wide'' pullbacks 
\[\begin{tikzcd}
                                                     & \prod U_i \arrow[ld] \arrow[rd] &                        \\
\C_i \arrow[rd, "U_i"'] \arrow[rr, "\dots", phantom] &                                 & \C_j \arrow[ld, "U_j"] \\
                                                     & \cK                             &                       
\end{tikzcd}\]
while equalizer are computed as in $2\mbox{-}\CAT$.
\end{proof}
Now we have a $2$-functor between complete $2$-categories and we want to show that it preserves limits. The strategy is as follows. 

Let $\C,\D$ be complete $\V$-categories, $F\colon\C\to\D$ a $\V$-functor, $D\colon\A\to\C$ a diagram and $\W\colon\A\to\V$ a weight. We get the comparison morphism $\bar{F}\colon F\{\W,\D\}\to\{\W, F\D\}$ in $\D$. We want to show that this is an iso. We will construct a new functor $G\colon\D\to\E$ s.t.\ both $G$ and $GF$ preserve weighted limits and $G$ reflects isomorphisms. Then the comparison morphism $\bar{GF}\colon GF\{\W,\D\}\xrightarrow{\cong}\{\W, GF\}$ factors as $GF\{\W,\D\}\xrightarrow{G(\bar{F})} G\{\W, F\D\}\xrightarrow[\cong]{\bar{G}}\{\W'GF\D\}$ so $G(\bar{F})$ is invertible hence also $\bar{F}$ is an isomorphism.

We want to construct such a functor $G$ in our setting. For this we use the constructions $\langle A,A \rangle, \{f,f\}_l$ and $[\rho,\rho]$.
\begin{prop}
Let $\cK$ be complete. Then there is an isomorphism of categories, $2$-natural in $T$.
\[2\mbox{-}\Mnd(\cK)^{\text{co}}(T,\langle A,A \rangle)\rightarrow
    2\mbox{-}\CAT/\cK(\bb1\xrightarrow{A}\cK,T\Alg_l\xrightarrow{U_l}\cK)\]
\end{prop}
\begin{proof}
  From Exercise $1.3$ we know that there is a natural bijection between monad morphisms $T\to\langle A,A\rangle$ and $T\mbox{-}\Alg$ structures $a\colon TA\to A$ on $A$. This gives the bijection on objects. Since this is constructed from the general theory of strict actions of strict monoidal categories, we know from Exercise 1.2 that monad modifications 
\[\begin{tikzcd}
	T & {\langle A,A\rangle}
	\arrow[""{name=0, anchor=center, inner sep=0}, "{a_2}"', curve={height=18pt}, shorten >=4pt, from=1-1, to=1-2]
	\arrow[""{name=1, anchor=center, inner sep=0}, "{a_1}", curve={height=-18pt}, shorten >=4pt, from=1-1, to=1-2]
	\arrow["\varphi"', shorten <=10pt, shorten >=10pt, Rightarrow, from=1, to=0]
\end{tikzcd}\]
correspond to lax $T$-morphisms $(\id_A,\varphi)\colon(A,a_2)\to(A,a_1)$ (note the reversal of direction, omitted in the Exercise). This corresponds precisely to a $2$-cell
\[\begin{tikzcd}
	\bb1 && {T\Alg_l} \\
	\\
	& \cK
	\arrow[""{name=0, anchor=center, inner sep=0}, "{(A,a_2)}", curve={height=-18pt}, from=1-1, to=1-3]
	\arrow[""{name=1, anchor=center, inner sep=0}, "{(A,a_1)}"'{pos=0.55}, curve={height=18pt}, from=1-1, to=1-3]
	\arrow["A"', from=1-1, to=3-2]
	\arrow["{U_l}", from=1-3, to=3-2]
	\arrow["{(\id,\varphi)}"', shorten <=10pt, shorten >=10pt, Rightarrow, from=0, to=1]
\end{tikzcd}\]
in $2\mbox{-}\CAT/\cK$.
\end{proof}
\begin{prop}
Let $\cK$ be a complete $2$-category and $f\colon A\to B$ a $1$-cell in $\cK$. Then there is an isomorphism of categories
\[2\mbox{-}\Mnd(\cK)^{\text{co}}(T,\{f,f\}_l)\rightarrow
    2\mbox{-}\CAT/\cK(\b2\xrightarrow{f}\cK,T\Alg_l\xrightarrow{U_l}\cK)\]
    which is $2$-natural in $T$.
\end{prop}
\begin{proof}
  We already know this bijection on objects. From Exercise $1.4$ we know that this bijection arises from the strict action $[\cK,\cK]\times \Colax[\b2,\cK]\to\Colax[\b2,\cK]$ of $2$-categories. Using Exercise $1.2$ here we find that monad modifications 
  \[\begin{tikzcd}
	T & {\{f,f\}_l}
	\arrow[""{name=0, anchor=center, inner sep=0}, "{\bar{f_2}}"', curve={height=18pt}, shorten >=4pt, from=1-1, to=1-2]
	\arrow[""{name=1, anchor=center, inner sep=0}, "{\bar{f_1}}", curve={height=-18pt}, shorten >=4pt, from=1-1, to=1-2]
	\arrow["\xi"', shorten <=10pt, shorten >=10pt, Rightarrow, from=1, to=0]
\end{tikzcd}\]
correspond to lax $T$-morphisms in $\Colax[\b2,\cK]$, which are the identity on objects, that is to pairs of $2$-cells $\xi_A, \xi_B$ s.t.
\[\begin{tikzcd}
	TA & A & {} & TA & A \\
	TB & B & {} & TB & B
	\arrow[""{name=0, anchor=center, inner sep=0}, "{a_2}"{pos=0.45}, curve={height=-12pt}, from=1-1, to=1-2]
	\arrow[""{name=1, anchor=center, inner sep=0}, "{a_1}"'{pos=0.45}, curve={height=12pt}, from=1-1, to=1-2]
	\arrow["{b_1}"', from=2-1, to=2-2]
	\arrow[""{name=2, anchor=center, inner sep=0}, "f", from=1-2, to=2-2]
	\arrow["{=}"{description}, draw=none, from=1-3, to=2-3]
	\arrow["{a_2}", from=1-4, to=1-5]
	\arrow[""{name=3, anchor=center, inner sep=0}, "{b_2}"{pos=0.49}, curve={height=-12pt}, from=2-4, to=2-5]
	\arrow[""{name=4, anchor=center, inner sep=0}, "{b_1}"', curve={height=12pt}, from=2-4, to=2-5]
	\arrow["f", from=1-5, to=2-5]
	\arrow["Tf"', from=1-1, to=2-1]
	\arrow[""{name=5, anchor=center, inner sep=0}, "Tf"', from=1-4, to=2-4]
	\arrow["{\bar{f_1}}"'{pos=0.4}, shift left=1, shorten <=10pt, shorten >=10pt, Rightarrow, from=2-1, to=2]
	\arrow["{\bar{f_2}}"{pos=0.5}, shift right=1, shorten <=10pt, shorten >=14pt, Rightarrow, from=5, to=1-5]
	\arrow["{\xi_B}"', shorten <=6pt, shorten >=6pt, Rightarrow, from=4, to=3]
	\arrow["{\xi_A}"', shorten <=6pt, shorten >=6pt, Rightarrow, from=1, to=0]
\end{tikzcd}\]
holds and $(\id_A,\xi_A)\colon (A,a_2)\to (A,a_1)$, $(\id_B,\xi_B)\colon (B,b_2)\to (B,b_1)$ are lax $T$-morphisms (exercise). This is precisely a $2$-cell 
\[\begin{tikzcd}
	\b2 && {T\Alg_l} \\
	\\
	& \cK
	\arrow[""{name=0, anchor=center, inner sep=0}, "{(f,\bar{f_2})}", curve={height=-18pt}, from=1-1, to=1-3]
	\arrow[""{name=1, anchor=center, inner sep=0}, "{(f,\bar{f_1})}"'{pos=0.55}, curve={height=18pt}, from=1-1, to=1-3]
	\arrow["f"', from=1-1, to=3-2]
	\arrow["{U_l}", from=1-3, to=3-2]
	\arrow[shorten <=10pt, shorten >=10pt, Rightarrow, from=0, to=1]
\end{tikzcd}\]
in $2\mbox{-}\CAT/\cK$. 
\end{proof}
\begin{prop}
If $\cK$ is complete and \begin{tikzcd}
	A & B
	\arrow[""{name=0, anchor=center, inner sep=0}, "g"', curve={height=12pt}, from=1-1, to=1-2]
	\arrow[""{name=1, anchor=center, inner sep=0}, "f", curve={height=-12pt}, from=1-1, to=1-2]
	\arrow["\rho", shorten <=6pt, shorten >=6pt, Rightarrow, from=1, to=0]
\end{tikzcd} $2$-cell in $\cK$, there is an iso of categories
\[2\mbox{-}\Mnd(\cK)^{\text{co}}(T,[\rho,\rho])\longrightarrow
    2\mbox{-}\CAT/\cK\left(\begin{tikzcd}
	0 & 1
	\arrow[""{name=0, anchor=center, inner sep=0}, curve={height=12pt}, from=1-1, to=1-2]
	\arrow[""{name=1, anchor=center, inner sep=0}, curve={height=-12pt}, from=1-1, to=1-2]
	\arrow[shorten <=6pt, shorten >=6pt, Rightarrow, from=1, to=0]
\end{tikzcd}\xlongrightarrow{\rho}\cK,T\Alg_l\xrightarrow{U_l}\cK\right)\]
that is $2$-natural in $T$.
\end{prop}
\begin{proof}
  One uses the action of $[\cK,\cK]$ on $$\Colax\left[\begin{tikzcd}
	0 & 1
	\arrow[""{name=0, anchor=center, inner sep=0}, curve={height=12pt}, from=1-1, to=1-2]
	\arrow[""{name=1, anchor=center, inner sep=0}, curve={height=-12pt}, from=1-1, to=1-2]
	\arrow[shorten <=6pt, shorten >=6pt, Rightarrow, from=1, to=0]
\end{tikzcd}, \cK\right],$$ which has objects the $2$-cells \begin{tikzcd}
	A & B
	\arrow[""{name=0, anchor=center, inner sep=0}, "g"', curve={height=12pt}, from=1-1, to=1-2]
	\arrow[""{name=1, anchor=center, inner sep=0}, "f", curve={height=-12pt}, from=1-1, to=1-2]
	\arrow["\rho", shorten <=6pt, shorten >=6pt, Rightarrow, from=1, to=0]
\end{tikzcd}, morphisms the quadruples $(a,\phi,\psi,b)$ such that
\[\begin{tikzcd}
	A & {A'} & {} & A & {A'} \\
	B & {B'} & {} & B & {B'}
	\arrow["a", from=1-1, to=1-2]
	\arrow[""{name=0, anchor=center, inner sep=0}, "g", curve={height=-12pt}, from=1-1, to=2-1]
	\arrow[""{name=1, anchor=center, inner sep=0}, "f"', curve={height=12pt}, from=1-1, to=2-1]
	\arrow["b"', from=2-1, to=2-2]
	\arrow[""{name=2, anchor=center, inner sep=0}, "{g'}", from=1-2, to=2-2]
	\arrow["{=}"{description}, draw=none, from=1-3, to=2-3]
	\arrow["a", from=1-4, to=1-5]
	\arrow["b"', from=2-4, to=2-5]
	\arrow[""{name=3, anchor=center, inner sep=0}, "{g'}", curve={height=-12pt}, from=1-5, to=2-5]
	\arrow[""{name=4, anchor=center, inner sep=0}, "{f'}"', curve={height=12pt}, from=1-5, to=2-5]
	\arrow[""{name=5, anchor=center, inner sep=0}, "f"', from=1-4, to=2-4]
	\arrow["\rho", shorten <=5pt, shorten >=5pt, Rightarrow, from=1, to=0]
	\arrow["\psi"', shorten <=8pt, shorten >=8pt, Rightarrow, from=0, to=2]
	\arrow["{\rho'}", shorten <=5pt, shorten >=5pt, Rightarrow, from=4, to=3]
	\arrow["\varphi", shorten <=8pt, shorten >=8pt, Rightarrow, from=5, to=4]
\end{tikzcd}\]
holds. The $2$-cells are pairs of $2$-cells subject to two axioms spelled out in the exercises. The construction is then analogous to the previous two propositions. That is we have to analyze what exactly a $T$-algebra in $$\Colax\left[\begin{tikzcd}
	0 & 1
	\arrow[""{name=0, anchor=center, inner sep=0}, curve={height=12pt}, from=1-1, to=1-2]
	\arrow[""{name=1, anchor=center, inner sep=0}, curve={height=-12pt}, from=1-1, to=1-2]
	\arrow[shorten <=6pt, shorten >=6pt, Rightarrow, from=1, to=0]
\end{tikzcd}, \cK\right]$$ is and what a lax $T$-morphism is, whose $1$-cell part is the identity. 
\end{proof}
The existence of adjoints is due to the completeness assumption. With this at hand we can now prove that $(-)\Alg_l$ turns colimits into limits.
\begin{thm}
    Let $\cK$ be a locally $\kappa$-presentable $2$-category. Then the $2$-functor
    \[(-)\Alg_l\colon2\mbox{-}\Mnd_{\kappa}(\cK)^{\text{coop}}\to 2\mbox{-}\CAT/\cK\] turns weighted colimits into limits.
\end{thm}
\begin{proof}
  We already know that the inclusion $2\mbox{-}\Mnd_{\kappa}(\cK)\to 2\mbox{-}\Mnd(\cK)$ preserves weighted colimits, so it suffices to prove the claim for diagrams in the latter $2$-category, which happen to have a colimit. So let $D\colon\A\to2\mbox{-}\Mnd(\cK)^{\text{co}}$ be a diagram, $\W\colon\A^{\op}\to\CAT$ a weight such that $\W\copw_{\A}D$ exists in $2\mbox{-}\Mnd(\cK)^{\text{co}}$. We have a comparison morphism $L\colon\W\copw_{\A}D\Alg_l\to\{\W,D\Alg_l\}$ in $2\mbox{-}\CAT/\cK$. The represented $2$-functor $2\mbox{-}\CAT/\cK(\bb1\xrightarrow{A}\cK,-)$ preserves weighted limits (as homs do) and the composite $2\mbox{-}\CAT/\cK(\bb1\xrightarrow{A}\cK,-)\circ(-)\Alg_l$ also preserves weighted limits, since it is represented by $\langle A,A \rangle$ by the first Proposition above. So in the commuting diagram 
  \[\begin{tikzcd}[column sep=.1mm, row sep=7mm]
	{2\mbox{-}\CAT/\cK(\bb1\xrightarrow{A}\cK,\W\copw_{\A}D\Alg_l) } && {2\mbox{-}\CAT/\cK(\bb1\xrightarrow{A}\cK,\{\W,D\Alg_l\})} \\
	\\
	& {\{\W,2\mbox{-}\CAT/\cK(\bb1\xrightarrow{A}\cK,D\Alg_l)\}}
	\arrow["{2\mbox{-}\CAT/\cK(\bb1\xrightarrow{A}\cK,L)}", from=1-1, to=1-3]
	\arrow["{\text{comparison}}"',"\cong", from=1-1, to=3-2]
	\arrow["{\text{comparison}}","\cong"', from=1-3, to=3-2]
\end{tikzcd}\]
both arrows labelled "comparison" are isomorphisms (compare with the discussion above for $F=(-)\Alg_l, G=2\mbox{-}\CAT/\cK(\bb1\xrightarrow{A}\cK,-)$).
Upshot: for each $A\in\cK$, $2\mbox{-}\CAT/\cK(\bb1\xrightarrow{A}\cK,L)$ is an isomorphism. Using the same argument applied to $(0\to 1)\xrightarrow{f}\cK$ and 
$$\begin{tikzcd}
	0 & 1
	\arrow[""{name=0, anchor=center, inner sep=0}, curve={height=12pt}, from=1-1, to=1-2]
	\arrow[""{name=1, anchor=center, inner sep=0}, curve={height=-12pt}, from=1-1, to=1-2]
	\arrow[shorten <=6pt, shorten >=6pt, Rightarrow, from=1, to=0]
\end{tikzcd}\xlongrightarrow{\rho}\cK$$ and the propositions about $\{f,f\}_l$ and $[\rho,\rho]$ we find that for all $1$-cells $f$ and all $2$-cells $\rho$ the $2$-functors $2\mbox{-}\CAT/\cK(f,L)$ and $2\mbox{-}\CAT/\cK(\rho,L)$ are isomorphisms. Since the $2$-functors $2\mbox{-}\CAT/\cK(A,-),2\mbox{-}\CAT/\cK(f,-)$ and $2\mbox{-}\CAT/\cK(\rho,-)$ jointly detect isomorphisms, we find that $L$ is an isomorphism. 
\end{proof}
We can do the same construction for $(-)\Alg_p$ and $(-)\Alg_s$. On the other hand, once we know that a $2$-category is of the form $T\Alg_l$ it has subcategories $T\Alg_p$ and $T\Alg_s$. We would like to be able to identify these in terms of the categories $D_i\Alg_p, D_i\Alg_s$ when forming limits. To do this we will use the $2$-monads $\{f,f\}_p$ and $\{f,f\}_s$. We have $2$-monads morphisms $\{f,f\}_s\to\{f,f\}_p\to\{f,f\}_l$ defined by the requirement that the $2$-cell $\bar{f}$ is either an identity or an isomorphism. A factorization of $T\to\{f,f\}_l$ through one of these is unique, if it exists, which it does if and only if the lax morphism corresponding to $\varphi$ is strict resp.\ pseudo.

\begin{lemma}
    Given a diagram $D\colon\cA\rightarrow 2-\Mnd_\kappa(\cK)$ and a weight
    $W\colon\cA^{\op}\rightarrow\Cat$, let $\cK_i\colon D_i\rightarrow W\odot_\cA
    D$ jointly ``codetect'' identities and isomorphisms. A 2-cell $\begin{tikzcd}
	{W\odot_\cA D} & T
	\arrow[""{name=0, anchor=center, inner sep=0}, curve={height=-9pt}, from=1-1, to=1-2]
	\arrow[""{name=1, anchor=center, inner sep=0}, curve={height=9pt}, from=1-1, to=1-2]
	\arrow["\alpha", shorten <=3pt, shorten >=3, Rightarrow, from=0, to=1]
    \end{tikzcd}$ is an identity (an isomorphism) if and only if each $\alpha
    \cK_i$ is. Then a lax $W\odot_\cA D$-morphism $(f,\overline{f})$ is pseudo
    (strict) if and only if $(\cK_i)^*(f,\overline{f})$ is.
\end{lemma}
\begin{proof}
    This follows from the existence of the classifiers $\{f,f\}_S$, $\{f,f\}_P$,
    which are defined by the universal requirement that a certain 2-cell is an
    identity (an isomorphism).
\end{proof}

\begin{lemma}
    The morphisms $\coprod_{i\in \cA}\coprod_{w\in W_i}D_i\rightarrow W\odot_\cA
    D$ jointly codetect isomorphisms and identities.
\end{lemma}
\begin{proof}
    Applying $2-\Mnd_\kappa(-T)$, this translates to a statement about weighted
    limits in $\Cat$, namely that for any $D'\colon\cA^{\op}\rightarrow\Cat$ the
    functor
    \[\{W,D'\}\rightarrow
    \Pi_{i\in\cA}\{W_i,D'_i\}\rightarrow
    \Pi_{i\in\cA}\Pi_{w\in
    W_i}D_i\]
    detects isomorphisms and identities, where the first is the canonical map we
    get from the characterization of weighted limits in terms of powers,
    products and equalizers and the second is a product of functors
    $\{W_i,D'_i\}=\Fun(W_i,D'_i)\rightarrow\Fun(\Ob W_i,D'_i)$.
     The latter functors detect isomorphisms and identities because a natural
     transformation is an isomorphism (an identity) if and only if all of its
     components are.

     The first functor is the equalizer in the standard
     presentation of $\{W,D'\}$ in $\Cat$, hence a (not necessarily full)
     inclusion of subcategories, thus it detects identities (using injectivity
     on objects). It also detects isomorphisms: if $Ff=Gf$ and $f$ is an
     isomorphism then $(Ff)^{-1}=(Gf)^{-1}$, so the inverse of an isomorphism
     lies in the equalizer.
\end{proof}

\begin{rmk}
    In practice once can do much better than the morphism in the above lemma:
    for example, for the cocomma object
    \[\begin{tikzcd}
        A & B \\
        C & D
        \arrow[from=1-2, to=2-2]
        \arrow["f", from=1-1, to=1-2]
        \arrow[from=2-1, to=2-2]
        \arrow[from=1-1, to=2-1]
        \arrow[shorten <=8pt, shorten >=8pt, Rightarrow, from=1-2, to=2-1]
    \end{tikzcd},\]
    the canonical arrow $B+C\rightarrow D$ codetects identities and
    isomorphisms. Identifying such a subset of objects with this property is
    easy once the 2-dimensional universal property is understood.
\end{rmk}

Summarizing, a lax $W\odot_\cA D$-morphism consists of certain 2-cells involving
the categories $D_i\Alg_L$ and it will be pseudo (strict) if and only if all of
the constituents are.

We now have almost all the ingredients necessary to identify $T\Alg_{S/P/L}$
when $T$ is given by a presentation.

\begin{exmp}
    Consider a locally $\kappa$-presentable monoidal 2-category $\cK$ such that
    for all objects $x$ both $x\otimes-$ and $-\otimes x$ preserve
    $\kappa$-filtered colimits. Monoidal here means exactly the 1-categorical
    definition, replacing functors and natural transformations with their
    2-dimensional counterparts. Examples of this are $[\cK,\cK]_\kappa$ with
    $\otimes=\circ$, $\cV-\Cat$ for a lfp cosmos $\cV$.

    We now present a complete characterization of the 2-category of monoids on
    $\cK$.

    Let $F\colon\cK\rightarrow\cK$ be the 2-endofunctor $M\mapsto M\otimes M+I$.
    Then $T\Alg_L$ has as objects the triples $(M,p\colon M\otimes M\rightarrow
    M,u\colon I\rightarrow M)$ subject to no axioms; morphisms
    $(M,p,u)\rightarrow(M',p',u')$ are 1-cells $f\colon M\rightarrow M'$ with
    2-cells
    \[\begin{tikzcd}
        {M\otimes M+I} & M \\
        {M'\otimes M'+I} & {M'}
        \arrow["{p'+u'}"', from=2-1, to=2-2]
        \arrow["{p+u}", from=1-1, to=1-2]
        \arrow[""{name=0, anchor=center, inner sep=0}, "f", from=1-2, to=2-2]
        \arrow[""{name=1, anchor=center, inner sep=0}, "{f\otimes f+I}"', from=1-1, to=2-1]
        \arrow["{\overline{f}}", shorten <=18pt, shorten >=18pt, Rightarrow, from=1, to=0]
    \end{tikzcd}\]
    subject to no axioms. This amounts to a pair of 2-cells
    $\overline{f_2}\colon p'\cdot f\otimes f\Rightarrow f\cdot p$ and
    $\overline{f_0}\colon u'\rightarrow f\cdot u$ by the universal property of
    the coproduct. The 2-cells
    $(f,\overline{f_2},\overline{f_0})\Rightarrow(g,\overline{g_2},\overline{g_0})$
    are 2-cells $\phi\colon f\Rightarrow g$ s.t.\
    \[\begin{tikzcd}
        {M\otimes M} & M & {=} & {M\otimes M} & M \\
        {M'\otimes M'} & {M'} && {M'\otimes M'} & {M'}
        \arrow["p", from=1-1, to=1-2]
        \arrow["{p'}"', from=2-1, to=2-2]
        \arrow[""{name=0, anchor=center, inner sep=0}, "g", from=1-2, to=2-2]
        \arrow[""{name=1, anchor=center, inner sep=0}, "{g\otimes g}", curve={height=-12pt}, from=1-1, to=2-1]
        \arrow[""{name=2, anchor=center, inner sep=0}, "{f\otimes f}"', curve={height=12pt}, from=1-1, to=2-1]
        \arrow[""{name=3, anchor=center, inner sep=0}, "{f\otimes f}"', from=1-4, to=2-4]
        \arrow[""{name=4, anchor=center, inner sep=0}, "f"', curve={height=12pt}, from=1-5, to=2-5]
        \arrow[""{name=5, anchor=center, inner sep=0}, "g", curve={height=-12pt}, from=1-5, to=2-5]
        \arrow["p", from=1-4, to=1-5]
        \arrow["{p'}"', from=2-4, to=2-5]
        \arrow["\phi\otimes\phi", shorten <=6pt, shorten >=6pt, Rightarrow, from=2, to=1]
        \arrow["{\overline{g_2}}", shorten <=11pt, shorten >=11pt, Rightarrow, from=1, to=0]
        \arrow["\phi", shorten <=6pt, shorten >=6pt, Rightarrow, from=4, to=5]
        \arrow["{\overline{f_2}}", shorten <=11pt, shorten >=11pt, Rightarrow, from=3, to=4]
    \end{tikzcd}\]
    and
    \[\begin{tikzcd}
        & M &&& M \\
        I && {=} & I \\
        & {M'} &&& {M'}
        \arrow["u", from=2-1, to=1-2]
        \arrow["{u'}"', from=2-1, to=3-2]
        \arrow[""{name=0, anchor=center, inner sep=0}, "g", curve={height=-12pt}, from=1-2, to=3-2]
        \arrow[""{name=1, anchor=center, inner sep=0}, "f"', curve={height=12pt}, from=1-2, to=3-2]
        \arrow[""{name=2, anchor=center, inner sep=0}, "g", from=1-5, to=3-5]
        \arrow["{u'}"', from=2-4, to=3-5]
        \arrow["u", from=2-4, to=1-5]
        \arrow["\phi", shorten <=5pt, shorten >=5pt, Rightarrow, from=1, to=0]
        \arrow["{\overline{f_0}}", shorten <=2pt, shorten >=7pt, Rightarrow, from=2-1, to=1]
        \arrow["{\overline{g_0}}", shorten <=10pt, shorten >=10pt, Rightarrow, from=2-4, to=2]
    \end{tikzcd}\]
    hold.

    Let $G\colon\cK\rightarrow\cK$ be the 2-endofunctor $GM=M\otimes(M\otimes
    M)+M+M$. The 2-category $G\Alg_L$ has objects $(M,p_\alpha\colon
    M\otimes(M\otimes M)\rightarrow M, p_\lambda\colon M\rightarrow M,
    p_\rho\colon M\rightarrow M)$ and 1-cells are the quadruples
    $(f,\overline{f_\alpha}\colon p'_\alpha\cdot(f\otimes (f\otimes
    f))\Rightarrow f\cdot p_\alpha,\overline{f_\lambda}\colon p'_\lambda\cdot
    f\Rightarrow f\cdot p_\lambda,\overline{f_\rho}\colon p'_\rho\cdot
    f\Rightarrow f\cdot p_\rho)$. The 2-cells are 2-cells $\phi\colon
    f\Rightarrow g$ s.t.
    \[\begin{tikzcd}
        {M\otimes (M\otimes M)} && M & {=} & {M\otimes(M\otimes M)} & M \\
        {M'\otimes(M'\otimes M')} && {M'} && {M'\otimes(M'\otimes M')} & {M'}
        \arrow[""{name=0, anchor=center, inner sep=0}, "{f\otimes (f\otimes f)}"', curve={height=18pt}, from=1-1, to=2-1]
        \arrow[""{name=1, anchor=center, inner sep=0}, "{g\otimes (g\otimes g)}", curve={height=-24pt}, from=1-1, to=2-1]
        \arrow[""{name=2, anchor=center, inner sep=0}, "g", from=1-3, to=2-3]
        \arrow["{p_\alpha}", from=1-1, to=1-3]
        \arrow["{p'_\alpha}"', from=2-1, to=2-3]
        \arrow[""{name=3, anchor=center, inner sep=0}, "{f\otimes(f\otimes f)}"', from=1-5, to=2-5]
        \arrow[""{name=4, anchor=center, inner sep=0}, "f"', curve={height=12pt}, from=1-6, to=2-6]
        \arrow[""{name=5, anchor=center, inner sep=0}, "g", curve={height=-12pt}, from=1-6, to=2-6]
        \arrow["{p_\alpha}", from=1-5, to=1-6]
        \arrow["{p'_\alpha}"', from=2-5, to=2-6]
        \arrow["{\phi\otimes(\phi\otimes\phi)}", shorten <=8pt, shorten >=8pt, Rightarrow, from=0, to=1]
        \arrow["{\overline{g_\alpha}}"{pos=0.7}, shorten <=48pt, shorten >=12pt, Rightarrow, from=1, to=2]
        \arrow["\phi", shorten <=6pt, shorten >=6pt, Rightarrow, from=4, to=5]
        \arrow["{\overline{f_\alpha}}", shorten <=16pt, shorten >=16pt, Rightarrow, from=3, to=4]
    \end{tikzcd}\]
    and the other axioms hold.
    The pseudo/strict versions of these are the ones where $\overline{f_0}$,
    $\overline{f_2}$ (respectively $\overline{f_\alpha}$,
    $\overline{f_\lambda}$, $\overline{f_\rho}$) are isomorphisms/identities.

    Next we construct two 2-functors $\psi_i\colon F\Alg_L\rightarrow G\Alg_L$,
    which send $(M,p,u)$ to $(M,p\cdot M\otimes p, p\cdot
    u\otimes\cdot\lambda_M^{-1},p\cdot M\otimes u\cdot\rho_M^{-1})$ and
    $(M,p\cdot p\otimes M\cdot\alpha_{M,M,M},\id_M,\id_M)$ respectively.

    On 1-cells, $\psi_1$ sends $(f,\overline{f_2},\overline{f_0})$ to
    \[\begin{tikzcd}
        {M\otimes (M\otimes M)} & {M\otimes M} & M \\
        {M'\otimes(M'\otimes M')} & {M'\otimes M'} & {M'}
        \arrow[""{name=0, anchor=center, inner sep=0}, "{f\otimes(f\otimes f)}"', from=1-1, to=2-1]
        \arrow[""{name=1, anchor=center, inner sep=0}, "f", from=1-3, to=2-3]
        \arrow[""{name=2, anchor=center, inner sep=0}, "{f\otimes f}"{description}, from=1-2, to=2-2]
        \arrow["p", from=1-2, to=1-3]
        \arrow["{p'}"', from=2-2, to=2-3]
        \arrow["{M'\otimes p'}"', from=2-1, to=2-2]
        \arrow["{M\otimes p}", from=1-1, to=1-2]
        \arrow["{\overline{f_2}}", shorten <=16pt, shorten >=16pt, Rightarrow, from=2, to=1]
        \arrow["{f\otimes\overline{f_2}}", shorten <=24pt, shorten >=24pt, Rightarrow, from=0, to=2]
    \end{tikzcd},\quad
        \begin{tikzcd}
        M & {I\otimes M} & {M\otimes M} & M \\
        {M'} & {I\otimes M'} & {M'\otimes M'} & {M'}
        \arrow[""{name=0, anchor=center, inner sep=0}, "f", from=1-4, to=2-4]
        \arrow[""{name=1, anchor=center, inner sep=0}, "{f\otimes f}"{description}, from=1-3, to=2-3]
        \arrow["p", from=1-3, to=1-4]
        \arrow["{p'}"', from=2-3, to=2-4]
        \arrow["{u'\otimes M'}"', from=2-2, to=2-3]
        \arrow["{u\otimes M}", from=1-2, to=1-3]
        \arrow[""{name=2, anchor=center, inner sep=0}, "{I\otimes f}"{description}, from=1-2, to=2-2]
        \arrow["{\lambda_{M'}^{-1}}"', from=2-1, to=2-2]
        \arrow["{\lambda_M^{-1}}", from=1-1, to=1-2]
        \arrow["f"', from=1-1, to=2-1]
        \arrow[shorten <=9pt, shorten >=9pt, Rightarrow, no head, from=2-1, to=1-2]
        \arrow["{\overline{f_2}}", shorten <=16pt, shorten >=16pt, Rightarrow, from=1, to=0]
        \arrow["{\overline{f_0}\otimes f}", shorten <=17pt, shorten >=17pt, Rightarrow, from=2, to=1]
    \end{tikzcd}\]
    and
    \[\begin{tikzcd}
        M & {M\otimes I} & {M\otimes M} & M \\
        {M'} & {M'\otimes I} & {M'\otimes M'} & {M'}
        \arrow[""{name=0, anchor=center, inner sep=0}, "f", from=1-4, to=2-4]
        \arrow["{p'}"', from=2-3, to=2-4]
        \arrow[shorten <=9pt, shorten >=9pt, Rightarrow, no head, from=2-1, to=1-2]
        \arrow["{\rho_M^{-1}}", from=1-1, to=1-2]
        \arrow["f"', from=1-1, to=2-1]
        \arrow["{M'\otimes u'}"', from=2-2, to=2-3]
        \arrow[""{name=1, anchor=center, inner sep=0}, "{I\otimes f}"{description}, from=1-2, to=2-2]
        \arrow["{\rho_{M'}^{-1}}"', from=2-1, to=2-2]
        \arrow[""{name=2, anchor=center, inner sep=0}, "{f\otimes f}"{description}, from=1-3, to=2-3]
        \arrow["p", from=1-3, to=1-4]
        \arrow["{M\otimes u}", from=1-2, to=1-3]
        \arrow["{\overline{f_2}}", shorten <=16pt, shorten >=16pt, Rightarrow, from=2, to=0]
        \arrow["{f\otimes\overline{f_0}}", shorten <=17pt, shorten >=17pt, Rightarrow, from=1, to=2]
    \end{tikzcd}\]

    The 2-functor $\psi_2$ sends $(f,\overline{f_2},\overline{f_0})$ to
    \[\begin{tikzcd}
        {M\otimes(M\otimes M)} & {(M\otimes M)\otimes M} & {M\otimes M} & M \\
        {M'\otimes(M'\otimes M')} & {(M'\otimes M')\otimes M'} & {M'\otimes M'} & {M'}
        \arrow["{f\otimes(f\otimes f)}"', from=1-1, to=2-1]
        \arrow[""{name=0, anchor=center, inner sep=0}, "{(f\otimes f)\otimes f}"{description}, from=1-2, to=2-2]
        \arrow[""{name=1, anchor=center, inner sep=0}, "{f\otimes f}"{description}, from=1-3, to=2-3]
        \arrow[""{name=2, anchor=center, inner sep=0}, "f", from=1-4, to=2-4]
        \arrow["{p'\otimes M'}"', from=2-2, to=2-3]
        \arrow["{p\otimes M}", from=1-2, to=1-3]
        \arrow["{p'}"', from=2-3, to=2-4]
        \arrow["p", from=1-3, to=1-4]
        \arrow["{\alpha_{M,M,M}}", from=1-1, to=1-2]
        \arrow[shorten <=14pt, shorten >=14pt, Rightarrow, no head, from=2-1, to=1-2]
        \arrow["{\alpha_{M',M',M'}}"', from=2-1, to=2-2]
        \arrow["{\overline{f_2}}", shorten <=16pt, shorten >=16pt, Rightarrow, from=1, to=2]
        \arrow["{\overline{f_2}\otimes f}", shorten <=24pt, shorten >=24pt, Rightarrow, from=0, to=1]
    \end{tikzcd},\]
    $1_f$ and $1_f$.

    On 2-cells both $\psi_1$ and $\psi_2$ act as the identity. The axioms hold
    because the $\alpha$, $\lambda$, $\rho$ parts are built from
    $\overline{f_0}$ and $\overline{f_2}$.

    From the construction we see that the $\psi_i$ restrict to 2-functors
    $F\Alg_S\rightarrow G\Alg_S$ and these restrictions are induced by 2-monad
    morphisms $\hat{\psi_i}\colon T(G)\rightarrow T(F)$, the free 2-monads on
    $G$ and $F$ respectively, by full faithfullness of the 1-functor
    $(-)\Alg_S$. In other words, $\psi_i=(\hat{\psi_i})^*$ is a strict morphism.
    Since there is a unique extension of $(\hat{\psi_i})^*$ to a 2-functor on
    $T(F)\Alg_L$ compatible with $U_L$, we have $\psi_i=(\hat{\psi_i})^*$ on all
    of $F\Alg_L\cong T(F)\Alg_L$.

    Now let $\Mon$ be the coequalizer of $\hat{\psi_1}$ and $\hat{\psi_2}$ in
    $2-\Mnd_\kappa(\cK)$. Then $\Mon\Alg_L$ is the coequalizer of the $\psi_i$,
    so the objects are precisely the monoids in $\cK$, the 1-cells are the
    triples $(f,\overline{f_2},\overline{f_0})$ subject to three axioms, namely
    that the 2-cells depicted above are equal. The 2-cell axioms remain the
    same: compatibility with $\overline{f_2}$ and $\overline{f_0}$. The
    pseudo/strict morphisms are the ones where $\overline{f_2}$,
    $\overline{f_0}$ are invertible/identities, since $T(F)\rightarrow\Mon$
    codetects isomorphisms/identities.

    We can spell out what this means for $\kappa$-accessible monads.

    Lax morphisms $(T,\mu^T,\eta^T)\rightarrow(S,\mu^S,\eta^S)$ are triples
    $(f,\overline{f_2},\overline{f_0})$ where $f\colon T\rightarrow S$ is
    2-natural and $\overline{f_0}$, $\overline{f_2}$ are modifications
    \[\begin{tikzcd}
        & T & {T^2} \\
        {\id_{\cK}} & {} \\
        & S & {S^2}
        \arrow["{\eta^T}", from=2-1, to=1-2]
        \arrow["{\eta^S}"', from=2-1, to=3-2]
        \arrow["{\mu^S}", from=3-3, to=3-2]
        \arrow["{\mu^T}"', from=1-3, to=1-2]
        \arrow[""{name=0, anchor=center, inner sep=0}, "{f^2}", from=1-3, to=3-3]
        \arrow["f"{description}, from=1-2, to=3-2]
        \arrow["{\overline{f_0}}", shorten <=6pt, shorten >=6pt, Rightarrow, from=2-1, to=2-2]
        \arrow["{\overline{f_2}}"', shorten <=10pt, shorten >=10pt, Rightarrow, from=0, to=2-2]
    \end{tikzcd}\]
    such that --diagrams-- hold.
    \[\begin{tikzcd}
        {T^3} & {T^2} & T && {T^3} & {T^2} & T \\
        {TS^2} & TS && {=} & {T^2S} & TS \\
        {S^3} & {S^2} & S && {S^3} & {S^2} & S
        \arrow[""{name=0, anchor=center, inner sep=0}, "f", from=1-3, to=3-3]
        \arrow["{\mu^S}"', from=3-2, to=3-3]
        \arrow["{S\mu^S}"', from=3-1, to=3-2]
        \arrow["{T\mu^S}", from=2-1, to=2-2]
        \arrow["{T\mu^T}", from=1-1, to=1-2]
        \arrow[""{name=1, anchor=center, inner sep=0}, "Tf", from=1-2, to=2-2]
        \arrow["{\mu^T}", from=1-2, to=1-3]
        \arrow["{f S^2}"', from=2-1, to=3-1]
        \arrow[""{name=2, anchor=center, inner sep=0}, "{Tf^2}"', from=1-1, to=2-1]
        \arrow["fS", from=2-2, to=3-2]
        \arrow[shorten <=8pt, shorten >=8pt, Rightarrow, no head, from=3-1, to=2-2]
        \arrow[""{name=3, anchor=center, inner sep=0}, "f", from=1-7, to=3-7]
        \arrow["{\mu^S}"', from=3-6, to=3-7]
        \arrow["{\mu^SS}"', from=3-5, to=3-6]
        \arrow["{\mu^TS}", from=2-5, to=2-6]
        \arrow["{\mu^TT}", from=1-5, to=1-6]
        \arrow["{\mu^T}", from=1-6, to=1-7]
        \arrow["Tf", from=1-6, to=2-6]
        \arrow[""{name=4, anchor=center, inner sep=0}, "fS", from=2-6, to=3-6]
        \arrow[""{name=5, anchor=center, inner sep=0}, "{f^2S}"', from=2-5, to=3-5]
        \arrow["{T^2f}"', from=1-5, to=2-5]
        \arrow[shorten <=8pt, shorten >=8pt, Rightarrow, no head, from=2-5, to=1-6]
        \arrow["{\overline{f_2}}", shorten <=10pt, shorten >=10pt, Rightarrow, from=2-2, to=0]
        \arrow["{T\overline{f_2}}", shorten <=13pt, shorten >=13pt, Rightarrow, from=2, to=1]
        \arrow["{\overline{f_2}S}", shorten <=13pt, shorten >=13pt, Rightarrow, from=5, to=4]
        \arrow["{\overline{f_2}}", shorten <=10pt, shorten >=10pt, Rightarrow, from=2-6, to=3]
    \end{tikzcd},\]
    \[\begin{tikzcd}
        & {T^2} & T \\
        T & ST && {=} & {1_f} \\
        S & {S^2} & S
        \arrow[""{name=0, anchor=center, inner sep=0}, "fT", from=1-2, to=2-2]
        \arrow[""{name=1, anchor=center, inner sep=0}, "f", from=1-3, to=3-3]
        \arrow["{\mu^S}"', from=3-2, to=3-3]
        \arrow["{\eta^SS}"', from=3-1, to=3-2]
        \arrow["f"', from=2-1, to=3-1]
        \arrow["Sf", from=2-2, to=3-2]
        \arrow["{\mu^T}", from=1-2, to=1-3]
        \arrow["{\eta^TT}", from=2-1, to=1-2]
        \arrow["{\eta^ST}"', from=2-1, to=2-2]
        \arrow[shorten <=8pt, shorten >=8pt, Rightarrow, no head, from=3-1, to=2-2]
        \arrow["{\overline{f_2}}", shorten <=10pt, shorten >=10pt, Rightarrow, from=2-2, to=1]
        \arrow["{\overline{f_0}}"', shorten <=10pt, shorten >=10pt, Rightarrow, from=2-1, to=0]
    \end{tikzcd}\]
    and
    \[\begin{tikzcd}
        T & {T^2} & T \\
        S & ST && {=} & {1_f} \\
        & {S^2} & S
        \arrow["fT", from=1-2, to=2-2]
        \arrow[""{name=0, anchor=center, inner sep=0}, "f", from=1-3, to=3-3]
        \arrow["{\mu^S}"', from=3-2, to=3-3]
        \arrow["Sf", from=2-2, to=3-2]
        \arrow["{\mu^T}", from=1-2, to=1-3]
        \arrow["{T\eta^T}", from=1-1, to=1-2]
        \arrow["f"', from=1-1, to=2-1]
        \arrow["{S\eta^T}", from=2-1, to=2-2]
        \arrow[shorten <=8pt, shorten >=8pt, Rightarrow, no head, from=2-1, to=1-2]
        \arrow[""{name=1, anchor=center, inner sep=0}, "{S\overline{f_0}}"', from=2-1, to=3-2]
        \arrow["{\overline{f_2}}", shorten <=10pt, shorten >=10pt, Rightarrow, from=2-2, to=0]
        \arrow["{S\overline{f_0}}", shorten <=4pt, shorten >=4pt, Rightarrow, from=1, to=2-2]
    \end{tikzcd}\]
    hold.

    Monad modifications between these are required to be compatible with
    $\overline{f_0}$ and $\overline{f_2}$.

    It is somewhat surprising that these are really the lax morphisms if you try
    to recognize them without the machinery we built.
\end{exmp}

Next we want to describe the 2-monad for pseudomonoids in $\cK$, which are
``monoids up to coherent isomorphism'', like monoidal $\cV$-categories. Instead
of forming the equalizer above, we form the iso-inserter and then we use an
equifier to impose the coherence laws. An equifier universally makes two 2-cells
equal.

Since this diagram will involve 2-cells, we need to know that all these 2-cells
in $2-\CAT/\cK$ come from 2-monad modifications. More precisely, we use the
following.

\begin{prop}
    Let $\cK$ be a locally $\kappa$-presentable 2-category. Then the 2-functor
    \[(-)\Alg_L\colon 2-\Mnd_\kappa(\cK)^{\text{coop}}\rightarrow 2-\CAT/\cK\]
    is locally fully faithful: any 2-cell $\alpha\colon\phi^*\Rightarrow\psi^*$
    comes from a unique monad modification $\psi\Rightarrow\phi\colon
    S\rightarrow T$.
\end{prop}
\begin{proof}
    We reduce this to the fact that the semantics-structure adjunction is fully
    faithful in the 1-categorical case.

    By the universal property of powers, $\alpha$ corresponds to a unique
    2-functor
    \[T\Alg_L\xrightarrow{\corners{\alpha}}[2]\pitchfork
    S\Alg_L\cong(S\odot[2])\Alg_L\]
    and $\corners{\alpha}$ sends strict $T$-morphisms to strict
    $(S\odot[2])$-morphisms. The inclusion $\{0,1\}\rightarrow[2]$ induces
    $[2]\pitchfork S\Alg_L\xrightarrow{(\pi_1,\pi_2)}S\Alg_L\times S\Alg_L$ and
    we have $\pi_1\corners{\alpha}=\phi^*$, $\pi_2\corners{\alpha}=\psi^*$, thus
    the composite $(\pi_1,\pi_2)\corners{\alpha}$ sends strict $T$-morphisms to
    strict $S+S$-morphisms.

    Since this inclusion codetects identities it follows that $(\pi_1,\pi_2)$
    detects strict morphisms, so $\corners{\alpha}$ does indeed send strict
    $T$-morphisms to strict $S\odot[2]$-morphisms. The restriction to strict
    morphisms comes from a 2-monad morphism $\gamma$. Moreover, by the
    uniqueness of the extension to lax morphisms we must have
    $\corners{\alpha}=\gamma^*$ on all of $T\Alg_L$. Thus, $\gamma\colon
    S\odot[2]\rightarrow T$ gives the desired 2-cell
    $\beta\colon\psi\Rightarrow\phi\colon S\rightarrow T$ woth $\beta^*=\alpha$
    by construction.

    This shows that $(-)\Alg_L$ is full on 2-cells. Faithfullness again follows
    from the existence of $S\odot[2]$ and faithfulness of $(-)\Alg_L$ on
    1-cells: if $\beta$, $\beta'$ induce the same 2-cell, then the corresponding
    $\corners{\beta},\corners{\beta'}\colon S\odot[2]\rightarrow T$ induce the
    same 1-cell on $T\Alg_L\rightarrow S\odot[2]\Alg_L$, so they are in
    particular equal on $T\Alg_S$, hence $\corners{\beta}=\corners{\beta'}$, so
    $\beta=\beta'$ by universal property of $S\odot[2]$.
\end{proof}

\begin{rmk}
    This argument would be simpler if $(-)\Alg_L$ were fully faithful on
    1-cells, but we don't know if this is true.
\end{rmk}

With this proposition in hand, we can now complete the construction of the
2-monad for pseudomonoids. Namely, instead of forming the coequalizer of
$\hat{\psi_1}$ and $\hat{\psi_2}$ above, we form the co-iso-inserter $T_1$ in
$2-\Mnd_\kappa(\cK)$ instead.

Then $T_1\Alg_L$ has objects $(M,p,u,l)$, where $l$ is an identity-on-objects
isomorphism between $\psi_1(M,p,u)$ and $\psi(M,p,u)$. This amounts to giving
invertible 2-cells
\[\begin{tikzcd}
	{M\otimes(M\otimes M)} && {M\otimes M} \\
	{(M\otimes M)\otimes M} & {M\otimes M} & M
	\arrow["{M\otimes p}", from=1-1, to=1-3]
	\arrow[""{name=0, anchor=center, inner sep=0}, "p", from=1-3, to=2-3]
	\arrow["p"', from=2-2, to=2-3]
	\arrow["{p\otimes M}"', from=2-1, to=2-2]
	\arrow[""{name=1, anchor=center, inner sep=0}, "{\alpha_{M,M,M}}"', from=1-1, to=2-1]
	\arrow["{\alpha^M}", shorten <=39pt, shorten >=39pt, Rightarrow, from=1, to=0]
\end{tikzcd},\]
\[\begin{tikzcd}
	M & {I\otimes M} & {M\otimes M} & M \\
	M &&& M
	\arrow["{\id_M}"', from=2-1, to=2-4]
	\arrow[""{name=0, anchor=center, inner sep=0}, Rightarrow, no head, from=1-4, to=2-4]
	\arrow[""{name=1, anchor=center, inner sep=0}, Rightarrow, no head, from=1-1, to=2-1]
	\arrow["{\lambda_M^{-1}}", from=1-1, to=1-2]
	\arrow["{u\otimes M}", from=1-2, to=1-3]
	\arrow["p", from=1-3, to=1-4]
	\arrow["{\lambda^M}", shorten <=45pt, shorten >=45pt, Rightarrow, from=1, to=0]
\end{tikzcd}\]
and
\[\begin{tikzcd}
	M & {M\otimes I} & {M\otimes M} & M \\
	M &&& M
	\arrow["{\id_M}"', from=2-1, to=2-4]
	\arrow[""{name=0, anchor=center, inner sep=0}, Rightarrow, no head, from=1-4, to=2-4]
	\arrow[""{name=1, anchor=center, inner sep=0}, Rightarrow, no head, from=1-1, to=2-1]
	\arrow["{\rho_M^{-1}}", from=1-1, to=1-2]
	\arrow["{M\otimes u}", from=1-2, to=1-3]
	\arrow["p", from=1-3, to=1-4]
	\arrow["{\rho^M}", shorten <=45pt, shorten >=45pt, Rightarrow, from=1, to=0]
\end{tikzcd}\]
subject to no axioms since $l$ is a 2-cell in $G\Alg_P$.

A 1-cell in $T_1\Alg_L$ is a 1-cell $(f,\overline{f_0},\overline{f_2})$ in
$F\Alg_L$ and that the resulting ``naturality square'' in $G\Alg_L$ coming from
$l$ and $l'$ commute (see the exercises). This means that the equations
\[\begin{tikzcd}[row sep=1cm]
	{M\otimes(M\otimes M)} && {M\otimes M} & M\\
	{M\otimes(M\otimes M)} & {(M\otimes M)\otimes M} & {M\otimes M} & M & {=} \\
	{M'\otimes(M'\otimes M')} & {(M'\otimes M')\otimes M'} & {M'\otimes M'} & {M'}
	\arrow["p", from=1-3, to=1-4]
	\arrow[""{name=0, anchor=center, inner sep=0}, Rightarrow, no head, from=1-4, to=2-4]
	\arrow["p", from=2-3, to=2-4]
	\arrow["{p\otimes M}", from=2-2, to=2-3]
	\arrow["{\alpha_{M,M,M}}", from=2-1, to=2-2]
	\arrow[""{name=1, anchor=center, inner sep=0}, Rightarrow, no head, from=1-1, to=2-1]
	\arrow["{M\otimes p}", from=1-1, to=1-3]
	\arrow["{p'}"', from=3-3, to=3-4]
	\arrow[""{name=2, anchor=center, inner sep=0}, "f", from=2-4, to=3-4]
	\arrow[""{name=3, anchor=center, inner sep=0}, "{(f\otimes f)\otimes f}"{description}, from=2-2, to=3-2]
	\arrow["{p'\otimes M'}"', from=3-2, to=3-3]
	\arrow[""{name=4, anchor=center, inner sep=0}, "{f\otimes f}"{description}, from=2-3, to=3-3]
	\arrow["{\alpha_{M',M',M'}}"', from=3-1, to=3-2]
	\arrow["{f\otimes(f\otimes f)}"', from=2-1, to=3-1]
	\arrow[shorten <=14pt, shorten >=14pt, Rightarrow, no head, from=3-1, to=2-2]
	\arrow["{\alpha_{M,M,M}}", shorten <=69pt, shorten >=69pt, Rightarrow, from=1, to=0]
	\arrow["{\overline{f_2}\otimes 1_f}", shorten <=24pt, shorten >=24pt, Rightarrow, from=3, to=4]
	\arrow["{\overline{f_2}}", shorten <=16pt, shorten >=16pt, Rightarrow, from=4, to=2]
\end{tikzcd}\]
\[\begin{tikzcd}[row sep=1cm]
	& {M\otimes(M\otimes M)} && {M\otimes M} & M \\
	{=} & {M'\otimes (M'\otimes M')} && {M'\otimes M'} & M \\
	& {M'\otimes(M'\otimes M')} & {(M'\otimes M')\otimes M'} & {M'\otimes M'} & {M'}
	\arrow["{M\otimes p}", from=1-2, to=1-4]
	\arrow["p", from=1-4, to=1-5]
	\arrow["{f\otimes(f\otimes f)}"', from=1-2, to=2-2]
	\arrow["{M'\otimes p'}", from=2-2, to=2-4]
	\arrow["p"', from=2-4, to=2-5]
	\arrow[""{name=0, anchor=center, inner sep=0}, "{f\otimes f}"', from=1-4, to=2-4]
	\arrow[""{name=1, anchor=center, inner sep=0}, "f", from=1-5, to=2-5]
	\arrow["{\alpha_{M',M',M'}}"', from=3-2, to=3-3]
	\arrow["{p'\otimes M'}"', from=3-3, to=3-4]
	\arrow["{p'}"', from=3-4, to=3-5]
	\arrow[""{name=2, anchor=center, inner sep=0}, Rightarrow, no head, from=2-5, to=3-5]
	\arrow[shorten <=26pt, shorten >=26pt, Rightarrow, no head, from=2-2, to=1-4]
	\arrow[""{name=3, anchor=center, inner sep=0}, Rightarrow, no head, from=2-2, to=3-2]
	\arrow["{\alpha^{M'}}", shorten <=69pt, shorten >=69pt, Rightarrow, from=3, to=2]
	\arrow["{\overline{f_2}}", shorten <=16pt, shorten >=16pt, Rightarrow, from=0, to=1]
\end{tikzcd}\]
and
\[\begin{tikzcd}
	M & {I\otimes M} & {M\otimes M} & M && M & {I\otimes M} & {M\otimes M} & M \\
	M &&& M & {=} & {M'} & {I\otimes M'} & {M'\otimes M'} & {M'} \\
	{M'} &&& {M'} && {M'} &&& {M'}
	\arrow["{\id_M}"', from=2-1, to=2-4]
	\arrow[""{name=0, anchor=center, inner sep=0}, Rightarrow, no head, from=1-1, to=2-1]
	\arrow[""{name=1, anchor=center, inner sep=0}, Rightarrow, no head, from=1-4, to=2-4]
	\arrow["p", from=1-3, to=1-4]
	\arrow["{u\otimes M}", from=1-2, to=1-3]
	\arrow["{\lambda_M^{-1}}", from=1-1, to=1-2]
	\arrow["{\id_M}"', from=3-1, to=3-4]
	\arrow["f"', from=2-1, to=3-1]
	\arrow["f", from=2-4, to=3-4]
	\arrow["{=}"{description}, draw=none, from=3-1, to=2-4]
	\arrow["{\id_{M'}}"', from=3-6, to=3-9]
	\arrow[""{name=2, anchor=center, inner sep=0}, Rightarrow, no head, from=2-9, to=3-9]
	\arrow[""{name=3, anchor=center, inner sep=0}, Rightarrow, no head, from=2-6, to=3-6]
	\arrow["{p'}"', from=2-8, to=2-9]
	\arrow[""{name=4, anchor=center, inner sep=0}, "f", from=1-9, to=2-9]
	\arrow[""{name=5, anchor=center, inner sep=0}, "{f\otimes f}"{description}, from=1-8, to=2-8]
	\arrow["{u'\otimes M'}"', from=2-7, to=2-8]
	\arrow[""{name=6, anchor=center, inner sep=0}, "{I\otimes f}"{description}, from=1-7, to=2-7]
	\arrow["{\lambda_{M'}^{-1}}"', from=2-6, to=2-7]
	\arrow[""{name=7, anchor=center, inner sep=0}, "f"', from=1-6, to=2-6]
	\arrow["{\lambda_M^{-1}}", from=1-6, to=1-7]
	\arrow["{u\otimes M}", from=1-7, to=1-8]
	\arrow["p", from=1-8, to=1-9]
	\arrow["{\lambda^M}", shorten <=45pt, shorten >=45pt, Rightarrow, from=0, to=1]
	\arrow["{=}"{description}, Rightarrow, draw=none, from=7, to=6]
	\arrow["{\overline{f_0}\otimes 1_f}", shorten <=17pt, shorten >=17pt, Rightarrow, from=6, to=5]
	\arrow["{\overline{f_2}}", shorten <=16pt, shorten >=16pt, Rightarrow, from=5, to=4]
	\arrow["{\lambda^{M'}}"', shorten <=48pt, shorten >=48pt, Rightarrow, from=3, to=2]
\end{tikzcd}\]
hold and the same goes for the one related to $\rho^M$, $\rho^{M'}$.

Note that these equations say precisely that $(f,\overline{f_0},\overline{f_2})$
is a lax monoidal morphism between (pre-)pseudomonoids
$(M,p,u,\alpha^M,\lambda^M,\rho^M)$ and
$(M',p',u',\alpha^{M'},\lambda^{M'},\rho^{M'})$, thus we already have the
correct 1-cells in $T_1\Alg_L$.

The 2-functor $T_1\Alg_L\rightarrow F\Alg_L$ is fully faithful on 2-cells: a
priori we need to impose the equation
\[l'\cdot\psi_1\left(\begin{tikzcd}
	\bullet & \bullet
	\arrow[""{name=0, anchor=center, inner sep=0}, "f", curve={height=-12pt}, from=1-1, to=1-2]
	\arrow[""{name=1, anchor=center, inner sep=0}, "g"', curve={height=12pt}, from=1-1, to=1-2]
	\arrow["\phi", shorten <=3pt, shorten >=3pt, Rightarrow, from=0, to=1]
\end{tikzcd}\right)
=
\psi_2\left(\begin{tikzcd}
	\bullet & \bullet
	\arrow[""{name=0, anchor=center, inner sep=0}, "f", curve={height=-12pt}, from=1-1, to=1-2]
	\arrow[""{name=1, anchor=center, inner sep=0}, "g"', curve={height=12pt}, from=1-1, to=1-2]
	\arrow["\phi", shorten <=3pt, shorten >=3pt, Rightarrow, from=0, to=1]
\end{tikzcd}\right)\cdot l,\]
but both $\psi_1$ and $\psi_2$ act as the identity on 2-cells and whiskering
with $l,l'$ does not affect the 2-cell because $l,l'$ have identities as 1-cell
components.

It follows that we already have the correct 2-cells in $T_1\Alg$ as well. Since
$T(F)\rightarrow T_1$ codetects identities and isomorphisms, the pseudo/strict
$T_1$-morphisms are the $(f,\overline{f_0},\overline{f_2})$ s.t.\
$\overline{f_0},\overline{f_2}$ are invertible/identities.

Our $T_1\Alg_L$ contains the 2-category of pseudomonoids and lax monoidal
morphisms as a full 2-subcategory on those objects, for which the pentagon and
unit triangle laws hold. We can use an equifier to describe this full
2-subcategory.

For this we consider a new 2-endofunctor $H\colon\cK\rightarrow\cK$ which sends
$M$ to $M\otimes(M\otimes(M\otimes M))+M\otimes M$. We construct a 2-functor
$\kappa_1\colon T_1\Alg_L\rightarrow H\Alg_L$ by sending $(M,p,u)$ to
\[M\otimes(M\otimes(M\otimes M))\xrightarrow{M\otimes (M\otimes p)}
M\otimes(M\otimes M)\xrightarrow{M\otimes p}
M\otimes M\xrightarrow{p}M,\]
\[M\otimes M\xrightarrow{M\otimes\lambda_M^{-1}}
M\otimes(I\otimes M)\xrightarrow{M\otimes p}
M\otimes M\xrightarrow{p}M\]
and a 2-functor $\kappa_2\colon T_1\Alg_L\rightarrow H\Alg_L$ by sending
$(M,p,u)$ to
\[\hspace*{-2.5cm}M\otimes(M\otimes(M\otimes M))\xrightarrow{\alpha_{M,M,M\otimes M}}
(M\otimes M)\otimes(M\otimes M)\xrightarrow{\alpha_{M\otimes M,M,M}}
((M\otimes M)\otimes M)\otimes M\xrightarrow{(p\otimes M)\otimes M}
(M\otimes M)\otimes M\xrightarrow{p\otimes M}
M\otimes M\xrightarrow{p}M,\]
\[M\otimes M\xrightarrow{\rho_M^{-1}}
(M\otimes I)\otimes M\xrightarrow{(M\otimes u)\otimes M}
(M\otimes M)\otimes M\xrightarrow{p\otimes M}
M\otimes M\xrightarrow{p}M.\]

We extend this to 1-cells using the evident pastings of $\overline{f_0}$ and
$\overline{f_2}$ and we let both 2-functors act as the identity on 2-cells.

Both restrict to 2-functors on strict morphisms, so by our general results they
are induced by 2-monad morphisms
\[\begin{tikzcd}
	{T(H)} & {T_1} & {}
	\arrow["{\hat{\kappa_1}}", curve={height=-6pt}, from=1-1, to=1-2]
	\arrow["{\hat{\kappa_2}}"', curve={height=6pt}, from=1-1, to=1-2]
\end{tikzcd}\]

There are two ways of changing brackets in a word of four letters and they
correspond to the two composites in MacLane's pentagon law. These and the cells
in the unit triangle induce 2-cells
$\beta_1,\beta_2\colon\kappa_1\Rightarrow\kappa_2$ in $2-\CAT/\cK$. We shall
explain this for the associator and leave the unit law as an exercise. To make
things more readable, we will simply write the tensor product in $\cK$ as a
concatenation, i.e.\ $M\otimes M$ will be $MM$. We construct two $2$-natural transformations $\beta_1, \beta_2\colon \kappa_1\to\kappa_2$ on $2\mbox{-}\Cat/\cK$ with component at $(M,p,u,\alpha,\lambda,\rho)\in T_1\Alg_l$ resp.\ given by 
\[\begin{tikzcd}
	{M(M(MM))} & {M(MM)} && MM && M \\
	& {M(MM)} \\
	{M(M(MM))} && {(MM)M} & MM && M \\
	& {(MM)(MM)} && {M(MM)} \\
	& {((MM)M)M} && {(MM)M} & MM & M
	\arrow["{M(Mp)}", from=1-1, to=1-2]
	\arrow["Mp", from=1-2, to=1-4]
	\arrow["p", from=1-4, to=1-6]
	\arrow[Rightarrow, no head, from=1-1, to=3-1]
	\arrow["{M(Mp)}"', from=3-1, to=2-2]
	\arrow[""{name=0, anchor=center, inner sep=0}, "\alpha"', from=2-2, to=3-3]
	\arrow[""{name=1, anchor=center, inner sep=0}, "\alpha"', from=3-1, to=4-2]
	\arrow["{(MM)p}", from=4-2, to=3-3]
	\arrow["\alpha"', from=4-2, to=5-2]
	\arrow["{(pM)M}"', from=5-2, to=5-4]
	\arrow["pM", from=3-3, to=3-4]
	\arrow[Rightarrow, no head, from=1-2, to=2-2]
	\arrow["p", from=3-4, to=3-6]
	\arrow[""{name=2, anchor=center, inner sep=0}, Rightarrow, no head, from=1-6, to=3-6]
	\arrow["{p(MM)}", from=4-2, to=4-4]
	\arrow["Mp"', from=4-4, to=3-4]
	\arrow["\alpha", from=4-4, to=5-4]
	\arrow["pM"', from=5-4, to=5-5]
	\arrow[""{name=3, anchor=center, inner sep=0}, Rightarrow, no head, from=3-6, to=5-6]
	\arrow[shift left=2, shorten <=61pt, shorten >=61pt, Rightarrow, no head, from=3-1, to=1-2]
	\arrow[shorten <=64pt, shorten >=64pt, Rightarrow, no head, from=5-2, to=4-4]
	\arrow[shift right=2, shorten <=67pt, shorten >=67pt, Rightarrow, no head, from=4-2, to=3-4]
	\arrow["p"', from=5-5, to=5-6]
	\arrow[shorten <=64pt, shorten >=64pt, Rightarrow, no head, from=1, to=0]
	\arrow["{\alpha^M}"', shorten <=120pt, shorten >=120pt, Rightarrow, from=2-2, to=2]
	\arrow["{\alpha^M}"', shorten <=32pt, shorten >=50pt, Rightarrow, from=4-4, to=3]
\end{tikzcd}\]
and 
\[\begin{tikzcd}
	{M(M(MM))} & {M(MM)} && MM && M \\
	{M(M(MM))} & {M((MM)M)} & {M(MM)} & MM && M \\
	& {(M(MM))M} & {(MM)M} && MM & M \\
	& {(M(MM))M} & {((MM)M)M} & {(MM)M} & MM & M
	\arrow["{M(Mp)}", from=1-1, to=1-2]
	\arrow[""{name=0, anchor=center, inner sep=0}, Rightarrow, no head, from=1-1, to=2-1]
	\arrow["M\alpha"', from=2-1, to=2-2]
	\arrow["{M(pM)}"', from=2-2, to=2-3]
	\arrow["Mp"', from=2-3, to=2-4]
	\arrow["Mp", from=1-2, to=1-4]
	\arrow["p", from=1-4, to=1-6]
	\arrow["p"', from=2-4, to=2-6]
	\arrow[Rightarrow, no head, from=1-6, to=2-6]
	\arrow["\alpha"', from=2-2, to=3-2]
	\arrow[""{name=1, anchor=center, inner sep=0}, Rightarrow, no head, from=3-2, to=4-2]
	\arrow["{(Mp)M}", from=3-2, to=3-3]
	\arrow[""{name=2, anchor=center, inner sep=0}, "\alpha", from=2-3, to=3-3]
	\arrow["pM", from=3-3, to=3-5]
	\arrow["p", from=3-5, to=3-6]
	\arrow[""{name=3, anchor=center, inner sep=0}, Rightarrow, no head, from=2-6, to=3-6]
	\arrow["{\alpha M}"', from=4-2, to=4-3]
	\arrow["p"', from=4-5, to=4-6]
	\arrow[Rightarrow, no head, from=3-6, to=4-6]
	\arrow[""{name=4, anchor=center, inner sep=0}, Rightarrow, no head, from=3-5, to=4-5]
	\arrow["{(pM)M}"', from=4-3, to=4-4]
	\arrow["pM"', from=4-4, to=4-5]
	\arrow[""{name=5, anchor=center, inner sep=0}, Rightarrow, no head, from=1-4, to=2-4]
	\arrow["{M\alpha^M}"', shorten <=122pt, shorten >=122pt, Rightarrow, from=0, to=5]
	\arrow["{\alpha^MM}", shorten <=103pt, shorten >=103pt, Rightarrow, from=1, to=4]
	\arrow["{\alpha^M}", shorten <=91pt, shorten >=91pt, Rightarrow, from=2, to=3]
\end{tikzcd}\]
which has the correct codomain since the pentagon law holds in $\cK$. In $\Cat$ these correspond precisely to the two composites in the pentagon law (involving two respectively three instances of the associator). A similar construction allows us to translate the unit axiom into two diagrams involving the second component of $\kappa_1,\kappa_2$ (exercise). These $\beta_i$ are $2$-natural since they are built from $2$-natural transformations in $\cK$ on $2$-cells $\alpha, \lambda, \rho$ which are by definition compatible with all $(f,\overline{f_0},\overline{f_2})$ in $T_1\Alg_l$. Now we use the Proposition ensuring that $(-)\Alg_l$ is fully faithful on $2$-cells: the $\beta_i$ are $(\hat{\beta_i})^*$ for unique monad modifications $\widehat{\beta_i}\colon\widehat{\kappa_2}\Rightarrow\widehat{\kappa_1}$. Let $\mathbf{PsMon}$ be the coequifier 
\[\begin{tikzcd}
	{T(H)} && {T_1} & {\mathbf{PsMon}}
	\arrow[""{name=0, anchor=center, inner sep=0}, "{\widehat{\kappa_2}}"', curve={height=18pt}, from=1-1, to=1-3]
	\arrow[from=1-3, to=1-4]
	\arrow[""{name=1, anchor=center, inner sep=0}, "{\widehat{\kappa_1}}", curve={height=-18pt}, from=1-1, to=1-3]
	\arrow["{\widehat{\beta_2}}", shift left=3, shorten <=5pt, shorten >=5pt, Rightarrow, from=1, to=0]
	\arrow["{\widehat{\beta_1}}", shift right=5, shorten <=5pt, shorten >=5pt, Rightarrow, from=1, to=0]
\end{tikzcd}\]
in $2\mbox{-}\Mnd_{\kappa}(\cK)$. Then $\mathbf{PsMon}\Alg_l$ is the equifier of $\beta_1$ and $\beta_2$, so it is the full sub-$2$-category of $T_1\Alg_l$ consisting of objects where $\beta_1$ and $\beta_2$ agree. Similarly for the unit law. Since an equifier does not affect $1$- and $2$-cells, our previous work shows that $\mathbf{PsMon}\Alg_l$ is isomorphic to the $2$-category of pseudomonoids, lax monoidal morphisms (in the usual sense) and monoidal $2$-cells. We have also shown that $\mathbf{PsMon}\Alg_p$ has as $1$-cells the strong monoidal morphisms and $\mathbf{PsMon}\Alg_l$ has as $1$-cells the strict monoidal morphisms. 
\\
\\
Our next example concerns categories with colimits of a given shape. This construction only works for conical colimits and only if the forgetful functor  $V\colon\V\to\Set$ is conservative (e.g.\ $\Set, \Mod_R$ but not $\sSet, \dgMod_R, \Cat$). We also assume that $\V$ is a lfp cosmos so that $\V\mbox{-}\Cat$ is a lfp $2$-category. 

Let $\D$ be a $\kappa$-presentable (ordinary) category. We will show that the $2$-category of small $\V$-categories with chosen $\D$-colimits and $\V$-functors which preserve $\D$-colimits is $T_{\D}\Alg_p$ for a suitable $\kappa$-accessible $2$-monad $T_{\D}$ on $\V\mbox{-}\Cat$. 

Our assumptions imply that $\C\in\V\mbox{-}\Cat$ has chosen $\D$-colimits iff the diagonal $\V$-functor $\Delta\colon\C\to[\D,\C]$ has a (chosen) left adjoint. 

So we start with the free $2$-monad on the $\kappa$-accessible endo-$2$-functor $F\coloneqq [\D,-]$. The objects of $F\Alg_l$ already have a $1$-cell $l\colon[\D,\C]\to\C$. We need to \emph{insert} a unit and a counit and impose the triangle identities using an equifier. 

There is a slight problem: note that the unit goes from $\id_{[\D,\C]}\Rightarrow\Delta l$, so a priori this is a $2$-cell $FC\rightrightarrows FC$ and doesn't need to live in $H\Alg_l$. But $F$ is a right $2$-adjoint, so we can find a suitable $H$, namely $H=[\D,-]\otimes\D$: to give
\[\begin{tikzcd}
	{[\D,\C]} & {[\D,\C]}
	\arrow[""{name=0, anchor=center, inner sep=0}, "{\Delta l}"', curve={height=12pt}, from=1-1, to=1-2]
	\arrow[""{name=1, anchor=center, inner sep=0}, "\id", curve={height=-12pt}, from=1-1, to=1-2]
	\arrow["\eta", shorten <=6pt, shorten >=6pt, Rightarrow, from=1, to=0]
\end{tikzcd}\]
is equivalent to giving 
\[\begin{tikzcd}
	{[\D,\C]\otimes\D} & \C
	\arrow[""{name=0, anchor=center, inner sep=0}, "{(\Delta l)^\#}"', curve={height=18pt}, from=1-1, to=1-2]
	\arrow[""{name=1, anchor=center, inner sep=0}, "{(\id)^\#}", curve={height=-18pt}, from=1-1, to=1-2]
	\arrow["\eta", shorten <=10pt, shorten >=10pt, Rightarrow, from=1, to=0]
\end{tikzcd}\]
in $\V\mbox{-}\Cat$. Thus our second endo-$2$-functor $G$ sends $\C$ to $\C+[\D,\C]\otimes\D$ (the first term being for the counit). We form the inserter of the two $2$-functors $F\Alg_l\to G\Alg_l$ sending $(\C,l\colon [\D,\C]\to\C)$ to $(l\Delta\colon\C\to\C,\id^\#\colon[\D,\C]\otimes\D\to\C)$ resp.\ $(\id\colon\C\to\C,(\Delta l)^\#\colon[\D,\C]\otimes\D\to\C)$. Here we really need to be able to give in non-invertible $2$-cells. The $1$-cells in $F\Alg_l$ are pairs $(F,\lambda)$ consisting of a $\V$-functor $f\colon\C\to\C'$ and a $2$-cell
\[\begin{tikzcd}
	{[\D,\C]} & \C \\
	{[\D,\C']} & {\C'}
	\arrow["l", from=1-1, to=1-2]
	\arrow[""{name=0, anchor=center, inner sep=0}, "{[\D,f]}"', from=1-1, to=2-1]
	\arrow["{l'}"', from=2-1, to=2-2]
	\arrow[""{name=1, anchor=center, inner sep=0}, "f", from=1-2, to=2-2]
	\arrow["\lambda"', shorten <=16pt, shorten >=16pt, Rightarrow, from=0, to=1]
\end{tikzcd}\]
and the two $2$-functors send this to
\[\left( \begin{tikzcd}
	\C & {[\D,\C]} & \C \\
	\C' & {[\D,\C']} & \C'
	\arrow[""{name=0, anchor=center, inner sep=0}, "f"', from=1-1, to=2-1]
	\arrow["\Delta"', from=2-1, to=2-2]
	\arrow["\Delta", from=1-1, to=1-2]
	\arrow["l", from=1-2, to=1-3]
	\arrow["{l'}"', from=2-2, to=2-3]
	\arrow[""{name=1, anchor=center, inner sep=0}, "f", from=1-3, to=2-3]
	\arrow[""{name=2, anchor=center, inner sep=0}, "{[\D,f]}"{description}, from=1-2, to=2-2]
	\arrow["\lambda", shorten <=15pt, shorten >=15pt, Rightarrow, from=2, to=1]
	\arrow[shorten <=15pt, shorten >=15pt, Rightarrow, no head, from=0, to=2]
\end{tikzcd}, \ 1_{\id^\#}\right)\]
and
\[\left( 1_{\id_{\C}}, \left(\begin{tikzcd}
	{[\D,\C]} & \C & {[\D,\C]} \\
	{[\D,\C']} & {\C'} & {[\D,\C']}
	\arrow["{l'}"', from=2-1, to=2-2]
	\arrow["l", from=1-1, to=1-2]
	\arrow["\Delta", from=1-2, to=1-3]
	\arrow["\Delta"', from=2-2, to=2-3]
	\arrow[""{name=0, anchor=center, inner sep=0}, "{[\D,f]}", from=1-3, to=2-3]
	\arrow[""{name=1, anchor=center, inner sep=0}, "f"', from=1-2, to=2-2]
	\arrow[""{name=2, anchor=center, inner sep=0}, "{[\D,f]}"', from=1-1, to=2-1]
	\arrow[shorten <=16pt, shorten >=16pt, Rightarrow, no head, from=1, to=0]
	\arrow["\lambda", shorten <=16pt, shorten >=16pt, Rightarrow, from=2, to=1]
\end{tikzcd}\right)^\#\right)\]
respectively. Both act as the identity on $2$-cells. Using the adjunction $-\otimes\D\dashv[\D,-]$, we find that the coinserter $T_1$ of the resulting $2$-monad morphism has $T_1\Alg_l$ given by quadruples $(\C,l,\eta,\epsilon)$, where $\eta\colon\id\Rightarrow\Delta l, \epsilon\colon l\Delta\Rightarrow\id$ (subject to no axioms) and $1$-cells are $(f,\lambda)$ s.t.\ 
\[\begin{tikzcd}
	&&&&& \C \\
	{[\D,\C]} & \C & {[\D,\C]} & {} & {[\D,\C]} && {[\D,\C]} \\
	{[\D,\C']} & {\C'} & {[\D,\C']} & {} & {[\D,\C']} && {[\D,\C']} \\
	&&& {}
	\arrow["l", from=2-1, to=2-2]
	\arrow["{c\mapsto\Delta_c}", from=2-2, to=2-3]
	\arrow[""{name=0, anchor=center, inner sep=0}, "{[\D,f]}"', from=2-1, to=3-1]
	\arrow["{l'}"', from=3-1, to=3-2]
	\arrow["{c'\mapsto\Delta_{c'}}"', from=3-2, to=3-3]
	\arrow["{[\D,f]}", from=2-3, to=3-3]
	\arrow[""{name=1, anchor=center, inner sep=0}, "f", from=2-2, to=3-2]
	\arrow["{=}"{description}, draw=none, from=2-4, to=3-4]
	\arrow[""{name=2, anchor=center, inner sep=0}, "{[\D,f]}"', from=2-5, to=3-5]
	\arrow["l", from=2-5, to=1-6]
	\arrow[""{name=3, anchor=center, inner sep=0}, "\Delta", from=1-6, to=2-7]
	\arrow[""{name=4, anchor=center, inner sep=0}, "{\id_{[\D,\C]}}"', from=2-5, to=2-7]
	\arrow["{\id_{[\D,\C']}}"', from=3-5, to=3-7]
	\arrow[""{name=5, anchor=center, inner sep=0}, "{[\D,f]}", from=2-7, to=3-7]
	\arrow[shorten <=9pt, shorten >=9pt, Rightarrow, no head, from=3-2, to=2-3]
	\arrow[shorten <=31pt, shorten >=31pt, Rightarrow, no head, from=2, to=5]
	\arrow["\eta"', shift left=4, shorten <=8pt, shorten >=8pt, Rightarrow, from=4, to=3]
	\arrow["\lambda"', shorten <=16pt, shorten >=16pt, Rightarrow, from=0, to=1]
\end{tikzcd}\]
and
\[\begin{tikzcd}
	&& {} \\
	\C & {[\D,\C]} & \C & {} & \C && \C \\
	{\C'} & {[\D,\C']} & {\C'} & {} & {\C'} && {\C'} \\
	&&&&& {[\D,\C]}
	\arrow[""{name=0, anchor=center, inner sep=0}, "{c\mapsto\Delta_c}", from=2-1, to=2-2]
	\arrow["l", from=2-2, to=2-3]
	\arrow["f"', from=2-1, to=3-1]
	\arrow["{c'\mapsto\Delta_{c'}}"', from=3-1, to=3-2]
	\arrow["{l'}"', from=3-2, to=3-3]
	\arrow[""{name=1, anchor=center, inner sep=0}, "f", from=2-3, to=3-3]
	\arrow[""{name=2, anchor=center, inner sep=0}, "{[\D,f]}"', from=2-2, to=3-2]
	\arrow["{=}"{description}, draw=none, from=2-4, to=3-4]
	\arrow["f"', from=2-5, to=3-5]
	\arrow["{\id_{\C}}", from=2-5, to=2-7]
	\arrow["{\id_{\C'}}", from=3-5, to=3-7]
	\arrow[""{name=3, anchor=center, inner sep=0}, "f", from=2-7, to=3-7]
	\arrow[""{name=4, anchor=center, inner sep=0}, "{c'\mapsto\Delta_{c'}}"', from=3-5, to=4-6]
	\arrow["{l'}"', from=4-6, to=3-7]
	\arrow[shorten <=34pt, shorten >=34pt, Rightarrow, no head, from=3-5, to=2-7]
	\arrow["{\id_{\C}}", curve={height=-30pt}, from=2-1, to=2-3]
	\arrow[shift left=3, shorten <=15pt, shorten >=15pt, Rightarrow, no head, from=3-1, to=2-2]
	\arrow["\lambda"', shorten <=16pt, shorten >=16pt, Rightarrow, from=2, to=1]
	\arrow["\epsilon"'{pos=0.3}, shift right=4, shorten <=22pt, shorten >=45pt, Rightarrow, from=4, to=3]
	\arrow["\epsilon"'{pos=0.3}, shift left=2, shorten <=22pt, shorten >=52pt, Rightarrow, from=0, to=1-3]
\end{tikzcd}\]
We now impose the triangle identities using an equifier in the same manner as before (using the necessary $2$-adjunction for the one, where the target is not $\C$). This is isomorphic to $T_{\D}\Alg_l$ where $T_{\D}$ denotes the corresponding coequifier in $2\mbox{-}\Mnd_{\kappa}(\V\mbox{-}\Cat)$. Since this is a coequifier, the $1$-cells and $2$-cells are the same as in $T_1\Alg_l$. However, now $l\dashv\Delta$ with unit $\eta$ and counit $\epsilon$, so the above coequifier say that $\lambda$ is the \emph{mate} of $1_{\C}$. So each $f$ has a \emph{unique} lax morphism structure. The pseudo $T_1$-morphism are the ones where $\lambda$ is invertible, so the same is true for $T_{\D}$. The components of $\lambda$ are precisely the colimit comparison morphisms, so the pseudo $T_{\D}$-morphisms are exactly the $\D$-colimits preserving $\V$-functors. One can also check that this works for $2$-cells, meaning all $\V$-natural transformations are $T_{\D}$-transformations. In $T_1$ there is a condition which becomes automatic when $\lambda$ is the mate of $1_{\C}$. 
\begin{rmk}
    The free objects for $T_{\D}$ should correspond to the $\D$-colimits closure in the diagram category $[\C,\V]$ of the representables. For this we need to understand ``how free'' $T_{\D}(\C)$ actually is in $T_{\D}\Alg_p$ (as opposed to $T_{\D}\Alg_s$).
\end{rmk}

\begin{rmk}
If we want to get the $2$-monad for categories with colimits of shape $\{\D_i\}_{i\in I}$ for some set of ordinary categories, we simply take the coproduct $\coprod T_{\D_i}$ in $2\mbox{-}\Mnd(\V\mbox{-}\Cat)$ (all $\D_i$ are $\kappa$-presentable). E.g.\ given shapes for binary coproducts, initial object and coequalizers we get finitely cocomplete categories in the case $\V=\Set$. Our final example concerns $2$-categories of $2$-functors. Let $\cK$ be a cocomplete $2$-category and $\A$ a small $2$-category. Then $[\A,\cK]$, the $2$-category of (strict) $2$-functors, (strict) $2$-natural transformations and modifications is the $2$-category of algebras for the $2$-monad 
\begin{align*}
    T\colon[\Ob\A,\cK]&\longrightarrow[\Ob\A,\cK]\\
    (X_a)_{a\in\A}&\mapsto \left(\sum_{a\in\A}\A(a,b)\copw X_a\right)_{b\in\A}
\end{align*}
by definition in our case if $\cK=\Cat$, and in general it follows from the adjunction defining the copower:
$$(\A(a,b)\copw X_a \to X_b) \leftrightsquigarrow (\A(a,b)\to\K(X_a,X_b)).$$
The coproduct of $T^2$ at $c\in\A$ is 
\begin{align*}
    \left(T^2(X_a)_{a\in\A}\right)_c &= \sum_b \A(b,c)\copw\left(T(X_a)_{a\in\A}\right)_b \\
    &= \sum_b \A(b,c)\copw\left(\sum_a \A(a,b)\copw X_a\right)\\
    &\cong \sum_{a,b} (\A(b,c)\times\A(a,b))\copw X_a.
\end{align*}
The unit and multiplication are given by the identities resp.\ composition in $\A$. To give a lax $T$-morphism $(F_a)_{a\in\A}\to(G_a)_{a\in\A}$ amounts to giving a pair $(f,\overline{f})$ whose $f$ is simply a morphism of collections, i.e.\ a $1$-cell $f_a\colon F_a\to G_a$ for each $a\in\A$ and $\overline{f}$ is a $2$-cell
\[\begin{tikzcd}
	{\sum_a\A(a,b)\copw F_a} & {F_b} \\
	{\sum_a\A(a,b)\copw G_a} & {G_b}
	\arrow["{\varphi_b}", from=1-1, to=1-2]
	\arrow[""{name=0, anchor=center, inner sep=0}, "{f_b}", from=1-2, to=2-2]
	\arrow["{\gamma_b}"', from=2-1, to=2-2]
	\arrow[""{name=1, anchor=center, inner sep=0}, "{\sum_a\A(a,b)\copw f_a}"', from=1-1, to=2-1]
	\arrow["{\overline{f_b}}"', shorten <=30pt, shorten >=30pt, Rightarrow, from=1, to=0]
\end{tikzcd}\]
for each $b\in\A$. Here $\varphi$ and $\gamma$ encode the algebraic structure of $F$ and $G$. By the universal property of coproducts, to give $\overline{f_b}$ is equivalent to giving a $2$-cell for each component $a\in\A$, which by universal property of copower corresponds to a $2$-cell 
\[\begin{tikzcd}
	{\A(a,b)} & {\cK(F_a,F_b)} \\
	{\cK(G_a,G_b)} & {\cK(F_a,G_b)}
	\arrow["{F_{a,b}}", from=1-1, to=1-2]
	\arrow[""{name=0, anchor=center, inner sep=0}, "{\cK(F_a,f_b)}", from=1-2, to=2-2]
	\arrow["{\cK(f_a,G_b)}"', from=2-1, to=2-2]
	\arrow[""{name=1, anchor=center, inner sep=0}, "{G_{a,b}}"', from=1-1, to=2-1]
	\arrow["{\overline{f_{a,b}}}", shorten <=33pt, shorten >=33pt, Rightarrow, from=1, to=0]
\end{tikzcd}\]
in $\Cat$. So this is simply a natural transformation in $\Cat$, which has components 
\[\begin{tikzcd}
	{F_a} & {F_b} \\
	{G_a} & {G_b}
	\arrow["F_\psi", from=1-1, to=1-2]
	\arrow[""{name=0, anchor=center, inner sep=0}, "{f_b}", from=1-2, to=2-2]
	\arrow["G_\psi"', from=2-1, to=2-2]
	\arrow[""{name=1, anchor=center, inner sep=0}, "{f_a}"', from=1-1, to=2-1]
	\arrow["f_\psi"', shorten <=13pt, shorten >=13pt, Rightarrow, from=1, to=0]
\end{tikzcd}\]
for each $\psi\colon a\to b$ in $\A$. So the data of a lax $T$-morphism corresponds bijectively to the data of a lax natural transformation $F\Rightarrow G$. In fact, $(f,\overline{f})$ satisfies the axioms of a lax $T$-morphism if and only if $(f_a,f_\psi)$ form a lax natural transformation. The naturality of $\overline{f_{a,b}}$ is precisely the compatibility of $f_\psi$ with $2$-cells and the two axioms for a $T$-morphism correspond to the pasting and identity axioms for a lax natural transformation. This follows since the axioms for $T$-morphisms can be checked componentwise. Similarly, one can check that $T$-transformations are the modifications. Finally, a $2$-cell out of a coproduct is invertible if and only if its components are and 
\[\begin{tikzcd}
	{\A(a,b)\copw X} & Y
	\arrow[""{name=0, anchor=center, inner sep=0}, curve={height=18pt}, from=1-1, to=1-2]
	\arrow[""{name=1, anchor=center, inner sep=0}, curve={height=-18pt}, from=1-1, to=1-2]
	\arrow[shorten <=10pt, shorten >=10pt, Rightarrow, from=1, to=0]
\end{tikzcd}\] 
is an isomorphism if and only if 
\[\begin{tikzcd}
	{\A(a,b)} & {\cK(X,Y)}
	\arrow[""{name=0, anchor=center, inner sep=0}, curve={height=18pt}, from=1-1, to=1-2]
	\arrow[""{name=1, anchor=center, inner sep=0}, curve={height=-18pt}, from=1-1, to=1-2]
	\arrow[shorten <=10pt, shorten >=10pt, Rightarrow, from=1, to=0]
\end{tikzcd}\]
is an isomorphism, so the pseudo $T$-morphism are precisely the $(f_a,f_\psi)$ s.t.\ each $f_\psi$ is an isomorphism. Thus the pseudo $T$-morphisms are precisely the pseudonatural transformations.
\end{rmk}  
\section{Limits and colimits in $T\Alg_p$}
Recall that for a $1$-monad $T$ on a complete category, $T\Alg$ is always complete. The enriched version of this also works. In particular, $T\Alg_s$ is complete if $\cK$ is. What about $T\Alg_p$ and $T\Alg_l$?
We start with some positive results.
\begin{prop}
    If $\cK$ has products and $T\colon\cK\to\cK$ is a $2$-monad, then the products in $T\Alg_s$ are products in $T\Alg_p$. 
\end{prop}
\begin{proof}
    We already know that products exist in $T\Alg_s$, so this amounts to checking the universal property. This is similar to the case we saw involving the colax limit of an arrow. In the next few propositions we will see more examples of this kind, so we have this as an exercise. 
\end{proof}
\begin{rmk}
We actually only proved existence of products in $T\Alg_s$ if $\cK$ is complete. It is true in general if $\cK$ has products (exercise). The same remains true in the following propositions.
\end{rmk}
\begin{prop}
    If $\cK$ has (iso-)inserters, then $T\Alg_p$ has (iso-)inserters. The universal $1$-cell is a strict $T$-morphism and it detects strict $T$-morphisms. 
\end{prop}
\begin{proof}
    We do the inserter case; the iso-inserter is similar. Let $(f,\overline{f}),(g,\overline{g})\colon(A,a)\rightsquigarrow(B,b)$ be two pseudo $T$-morphisms and let 
    \[\begin{tikzcd}
	&& A \\
	I &&&& B \\
	&& A
	\arrow["p", from=2-1, to=1-3]
	\arrow["f", from=1-3, to=2-5]
	\arrow["p"', from=2-1, to=3-3]
	\arrow["g"', from=3-3, to=2-5]
	\arrow["\lambda", shorten <=18pt, shorten >=18pt, Rightarrow, from=1-3, to=3-3]
\end{tikzcd}\]
    be the inserter in $\cK$. We have $a\cdot Tp\colon TI\to A$ and a $2$-cell 
    $$fa\cdot Tp\xRightarrow{\overline{f}^{-1}\cdot Tp}b\cdot Tf\cdot Tp\xRightarrow{b\cdot T\lambda} b\cdot Tg\cdot Tp\xRightarrow{\overline{g}\cdot Tp} ga\cdot Tp$$
    and so from the universal property we get a unique $i\colon TI\to I$ s.t.\ $p\cdot i = a\cdot Tp$ and the equation 
    \[\begin{tikzcd}
	&& TA &&& {} &&& TA \\
	TI &&&& TB && TI &&&& TB \\
	&& A &&&&&& TA \\
	I &&&& B && I &&&& B \\
	&& A &&& {} &&& A
	\arrow["Tp", from=2-1, to=1-3]
	\arrow["a"', from=1-3, to=3-3]
	\arrow["i"', from=2-1, to=4-1]
	\arrow["p", from=4-1, to=3-3]
	\arrow[""{name=0, anchor=center, inner sep=0}, "Tf", from=1-3, to=2-5]
	\arrow["b", from=2-5, to=4-5]
	\arrow[""{name=1, anchor=center, inner sep=0}, "f", from=3-3, to=4-5]
	\arrow["p"', from=4-1, to=5-3]
	\arrow["g"', from=5-3, to=4-5]
	\arrow["\lambda"', shorten <=13pt, shorten >=13pt, Rightarrow, from=3-3, to=5-3]
	\arrow["{=}"{description}, draw=none, from=1-6, to=5-6]
	\arrow["Tp", from=2-7, to=1-9]
	\arrow["Tf", from=1-9, to=2-11]
	\arrow["Tp"', from=2-7, to=3-9]
	\arrow[""{name=2, anchor=center, inner sep=0}, "Tg"', from=3-9, to=2-11]
	\arrow["i"', from=2-7, to=4-7]
	\arrow["p"', from=4-7, to=5-9]
	\arrow["a"', from=3-9, to=5-9]
	\arrow["b", from=2-11, to=4-11]
	\arrow[""{name=3, anchor=center, inner sep=0}, "g"', from=5-9, to=4-11]
	\arrow["\lambda", shorten <=13pt, shorten >=13pt, Rightarrow, from=1-9, to=3-9]
	\arrow["{\overline{f}}"', shorten <=17pt, shorten >=17pt, Rightarrow, from=0, to=1]
	\arrow["{\overline{g}}"', shorten <=17pt, shorten >=17pt, Rightarrow, from=2, to=3]
\end{tikzcd}\]
    holds. As in the proof of the ``colax limit of an arrow'', we use the axioms for $(f,\overline{f})$ and $(g,\overline{g})$ and the $2$-naturality of $\eta$ and $\mu$ to show that $(I,i)$ is a $T$-algebra. By construction, $p\colon(I,i)\to(A,a)$ is a strict $T$-morphism and $\lambda$ a $T$-transformation. It remains to check the universal property, so consider 
    \[\begin{tikzcd}
	&& {(A,a)} \\
	{(X,x)} &&&& {(B,b)} \\
	&& {(A,a)}
	\arrow["{(q,\overline{q})}", squiggly, from=2-1, to=1-3]
	\arrow["{(q,\overline{q})}"', squiggly, from=2-1, to=3-3]
	\arrow[squiggly, from=3-3, to=2-5]
	\arrow[squiggly, from=1-3, to=2-5]
	\arrow["\mu", shorten <=13pt, shorten >=13pt, Rightarrow, from=1-3, to=3-3]
\end{tikzcd}\]
in $T\Alg_p$. This means that the equation 
\[\begin{tikzcd}
	&& TA &&& {} &&& TA \\
	TX &&&& TB && TX &&&& TB \\
	&& A &&&&&& TA &&& {(*)}\\
	X &&&& B && X &&&& B \\
	&& A &&& {} &&& A
	\arrow[""{name=0, anchor=center, inner sep=0}, "Tq", from=2-1, to=1-3]
	\arrow["a"', from=1-3, to=3-3]
	\arrow["x"', from=2-1, to=4-1]
	\arrow[""{name=1, anchor=center, inner sep=0}, "q", from=4-1, to=3-3]
	\arrow[""{name=2, anchor=center, inner sep=0}, "Tf", from=1-3, to=2-5]
	\arrow["b", from=2-5, to=4-5]
	\arrow[""{name=3, anchor=center, inner sep=0}, "f", from=3-3, to=4-5]
	\arrow["q"', from=4-1, to=5-3]
	\arrow["g"', from=5-3, to=4-5]
	\arrow["\mu"', shorten <=20pt, shorten >=20pt, Rightarrow, from=3-3, to=5-3]
	\arrow["{=}"{description}, draw=none, from=1-6, to=5-6]
	\arrow["Tq", from=2-7, to=1-9]
	\arrow["Tf", from=1-9, to=2-11]
	\arrow[""{name=4, anchor=center, inner sep=0}, "Tq"', from=2-7, to=3-9]
	\arrow[""{name=5, anchor=center, inner sep=0}, "Tg"', from=3-9, to=2-11]
	\arrow["x"', from=2-7, to=4-7]
	\arrow[""{name=6, anchor=center, inner sep=0}, "q"', from=4-7, to=5-9]
	\arrow["a"', from=3-9, to=5-9]
	\arrow["b", from=2-11, to=4-11]
	\arrow[""{name=7, anchor=center, inner sep=0}, "g"', from=5-9, to=4-11]
	\arrow["T\mu", shorten <=20pt, shorten >=20pt, Rightarrow, from=1-9, to=3-9]
	\arrow["{\overline{f}}"', shorten <=27pt, shorten >=27pt, Rightarrow, from=2, to=3]
	\arrow["{\overline{g}}"', shorten <=27pt, shorten >=27pt, Rightarrow, from=5, to=7]
	\arrow["{\overline{q}}", shorten <=27pt, shorten >=27pt, Rightarrow, from=0, to=1]
	\arrow["{\overline{q}}"', shorten <=27pt, shorten >=27pt, Rightarrow, from=4, to=6]
\end{tikzcd}\]
holds. We have a unique $1$-cell $h\colon X\to I$ s.t.\ $ph=q$ and $\lambda h=\mu$ from the universal property of $(I,p,\lambda)$ in $\cK$. Thus $\overline{q}$ can be seen as a $2$-cell
$$p\cdot i\cdot Th = a\cdot Tp\cdot Th = a\cdot Tq \xRightarrow{\overline{q}} q\cdot x=p\cdot h\cdot x$$
in $\cK$. Plugging this into $(*)$ and using $ph=q, \lambda h=\mu$, we find that $(\lambda hx)\cdot(f\overline{q})\cdot(\overline{f}\cdot Tp\cdot Th)=(g\overline{q})\cdot (\overline{g}\cdot Tp\cdot Th)\cdot (b\cdot T\lambda\cdot Th)$ holds.
Using the definition of $i$ in terms of $\overline{f}^{-1}$, we find that the equation holds if and only if $(\lambda\cdot h\cdot x)\cdot (f\cdot\overline{q})=(g\cdot\overline{q})(\lambda\cdot i\cdot Th)$ holds. From the $2$-dimensionality of the universal property of $(I,p,\lambda)$ it follows that there exists a unique $\overline{h}\colon i\cdot Th\Rightarrow x\cdot h$ s.t.\ $p\overline{h}=\overline{q}$. Using the uniqueness part of the $2$-dimensional universal property plus the fact that $(q,\overline{q})$ is a pseudo $T$-morphism, it follows that $(h,\overline{h})$ is a pseudo $T$-morphism. This $(h,\overline{h})$ is clearly the unique $1$-cell with $p\cdot (h,\overline{h})=(q,\overline{q})$, so this shows the $1$-dimensional universal property. Checking the $2$-dimensional universal property is left as an exercise.
\end{proof} 

\backmatter
\end{document}
