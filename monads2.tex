\documentclass[a4paper,11pt,oneside,openany]{scrbook}
\usepackage{monads}
\begin{document}

\begin{titlepage}
	\begin{center}
		\Huge \textbf{Monads~and~their~applications~II}\\
		\vspace{1cm}
		\Large Dr.\ Daniel Schäppi's course lecture notes\\
	\end{center}

	\vspace{1cm}

	\begin{center}
		\Large	by\\
		\vspace{.2cm}
		\Large	Nicola Di Vittorio\\
		\Large	Matteo Durante\\

		
	\end{center}
\end{titlepage}
\thispagestyle{empty}\doclicenseThis

\frontmatter


\mainmatter

\chapter*{Introduction}

These notes will focus on 2-dimensional monad theory, which can be viewed as the
study of algebraic structures on 2-categories. Like in the one-dimensional case,
after defining a 2-monad we concern ourselves with the categories of algebras it
defines, however the higher dimension allows to relax the definitions and
observe how different coherence conditions lead to different (and generally less
well-behaved) objects.

One may ask why we are keen to better understand 2-monads. One answer is
that, similarly to the 1-dimensional case, this allows us to better understand
other 2-categories, perhaps with additional structure (i.e.\ monoidal, braided,
some kinds of limits, etc) by relating them to 2-categories of algebras.

We now start recalling some relevant definitions and facts which we will need
later on.

In order to carry out our project we shall work with $\cV$-cosmos and
presentability conditions.

\begin{defn}
    A cosmos $\cV$ is a complete, cocomplete symmetric monoidal closed category.
\end{defn}

\begin{defn}
    An object $c$ in a $\cV$-category $\cC$ is $\kappa$-presentable if
    $\cC(c,-)\colon\cC\rightarrow\cV$ preserves $\kappa$-filtered colimits. This
    is equivalent to saying that the functor
    $\cC(c,-)\colon\cC_0\rightarrow\cV_0$ is $\kappa$-accessible, where $\cC_0$
    and $\cV_0$ are the underlying categories.
\end{defn}

\begin{thm}
    Let $\cV$ be a lfp cosmos. Then $\cV$-$\Cat$ is a lfp cosmos and a lfp
    2-category.
\end{thm}

By studying monads in this setting we achieve a great level of generality since
our results will not depend on the underlying enrichment, thus unifying many
contexts.

But what is a 2-monad?

\begin{defn}
    A 2-monad is a monad in the 2-category 2-$\CAT$ of locally small
    2-categories, 2-functors and (strict) 2-natural transformations.
\end{defn}

We will often construct them using presentations, that is via colimit
constructions and free 2-monads on 2-endofunctors. This is achieved through the
following results.

\begin{thm}\label{lfpadj}
    Let $\cV$ be a lfp cosmos, $\cC$ a locally $\kappa$-presentable
    $\cV$-category. Then the forgetful functor
    \[
        \cV-\Mnd_\kappa(\cC)\rightarrow\cV-\CAT_\kappa(\cC,\cC)
    \]
    is monadic. Moreover, it preserves colimits.
\end{thm}

\begin{cor}
    In the above situation, the functor
    \[
        (-)\Alg\colon\cV-\Mnd_\kappa(\cC)\rightarrow\cV-\CAT/\cC
    \]
    sends colimits to limits.
\end{cor}

\begin{rmk}
    In general, $\cV-\Mnd_\kappa(\cC)$ is not a $\cV$-category. This is because
    monads are monoids in a monoidal $\cV$-category of endofunctors, but monoids
    in general do not define a $\cV$-category: for example, consider
    $\Mon(\Ab)=\Ring$, which is not even additive.

    This has to do with the non-existence of a ``diagonal'' $\cV$-functor
    $\cV\rightarrow\cV\otimes\cV$. In particular, if $\cV$ is cartesian then
    this problem does not arise and indeed for $\cV=\Cat$ we expect the
    monadic adjunction~\ref{lfpadj} to be enriched.
\end{rmk}

Unfortunately, we can't apply the theorem above to show the corollary. Instead,
we use it to give a presentation of a 2-monad whose algebras are 2-monads with
rank $\kappa$.

Given a monoidal 2-category $\cM$ (i.e.\ the associator $(A\otimes B)\otimes
C\rightarrow A\otimes(B\otimes C)$ is 2-natural, satisfies the pentagon axioms,
etc), we have a 2-category $\Mon(\cM)$ of monoids $(M,\mu\colon M\otimes
M\rightarrow M,\eta\colon I\rightarrow M)$ in $\cM$ with 1-cells the monoid
morphisms and 2-cells the 2-cells $\alpha\colon f\Rightarrow g\colon M
\rightarrow N$ in $\cM$ s.t.\
\[\begin{tikzcd}
	{M\otimes M} & M & N & {=} & {M\otimes M} && {N\otimes N} & N
	\arrow[""{name=0, anchor=center, inner sep=0}, "f", curve={height=-12pt}, from=1-2, to=1-3]
	\arrow[""{name=1, anchor=center, inner sep=0}, "g"', curve={height=12pt}, from=1-2, to=1-3]
	\arrow["{\mu_M}"', from=1-1, to=1-2]
	\arrow["{\mu_N}"', from=1-7, to=1-8]
	\arrow[""{name=2, anchor=center, inner sep=0}, "{g\otimes g}"', curve={height=12pt}, from=1-5, to=1-7]
	\arrow[""{name=3, anchor=center, inner sep=0}, "{f\otimes f}", curve={height=-12pt}, from=1-5, to=1-7]
	\arrow["\alpha", shorten <=3pt, shorten >=3pt, Rightarrow, from=0, to=1]
	\arrow["\alpha\otimes\alpha", shorten <=3pt, shorten >=3pt, Rightarrow, from=3, to=2]
\end{tikzcd},\]
\[\begin{tikzcd}
	I & M & N & {=} & {\id_{\eta_N}}
	\arrow[""{name=0, anchor=center, inner sep=0}, "f", curve={height=-12pt}, from=1-2, to=1-3]
	\arrow[""{name=1, anchor=center, inner sep=0}, "g"', curve={height=12pt}, from=1-2, to=1-3]
	\arrow["{\eta_M}"', from=1-1, to=1-2]
	\arrow["\alpha", shorten <=3pt, shorten >=3pt, Rightarrow, from=0, to=1]
\end{tikzcd}\]
hold.

If $-\otimes-$ preserves $\kappa$-filtered colimits in each variable, then the
2-functors $FM=M\otimes M$, $GM=(M\otimes M)\otimes M+M+M$ are
$\kappa$-accessible and we have two natural ways to go from $F$-algebras to
$G$-algebras.

The coequalizer of the resulting pair of maps on the free monads
$TG\rightrightarrows TF$ gives us a presentation of a 2-monad $T$ as a
coequalizer. It has $T\Alg\cong\Mon(\cM)$ by construction if $\cM$ is locally
$\kappa$-presentable as a 2-category.

Let $\cK$ be a locally $\kappa$-presentable 2-category, i.e.\ $\cV-\Cat$ and
specifically $\Cat$, and let $\cM=[\cK,\cK]_\kappa$. Then the category of
$\kappa$-accessible endofunctors on $\cK$, that is $\cM$, is itself locally
$\kappa$-presentable.

Notice that the composition preserves $\kappa$-filtered colimits in each
varible. Indeed, for $F^*$ it's clear and for $F_*$ is too since $F$ is
$\kappa$-accessible.

Monoids in $\cM$ are 2-monads!

To show that $2-\Mnd_\kappa(\cK)\rightarrow 2-\Mnd(\cK)$ preserves colimits we
need the following proposition.

\begin{prop}
    Let $F$ be a strong monoidal 2-adjoint $\cM\rightarrow\cM'$. Then the right
    2-adjoint inherits a lax monoidal structure s.t.\ unit and counit are
    monoidal. Both 2-functors lift to the 2-categories of monoids, so
    $\Mon(F)\colon\Mon(\cM)\rightarrow\Mon(\cM')$ is a left 2-adjoint.
\end{prop}
\begin{proof}
    Exercise.
\end{proof}

We can now prove what we stated earlier.

\begin{thm}
    Let $\cK$ be a locally $\kappa$-presentable 2-category. Then the forgetful
    2-functor
    \[
        2-\Mnd_\kappa(\cK)\rightarrow[\cK,\cK]_\kappa
    \]
    is 2-monadic and $\kappa$-accessible. In particular, $2-\Mnd_\kappa(\cK)$ is
    a locally $\kappa$-presentable 2-category.
    Moreover, the inclusion
    \[
        2-\Mnd_\kappa(\cK)\rightarrow 2-\Mnd(\cK)
    \]
    preserves colimits and in fact it is a left adjoint.
\end{thm}
\begin{proof}
    We have $2-\Mnd_\kappa(\cK)=\Mon([\cK,\cK]_\kappa)$, so the above discussion
    shows that there is a $\kappa$-accessible 2-monad on $[\cK,\cK]_\kappa$ with
    $T\Alg\cong 2-\Mnd_\kappa(\cK)$.

    For the second part, recall that left Kan extensions along the inclusion
    $J\colon\cK_\kappa\rightarrow\cK$ of $\kappa$-presentable objects gives an
    equivalence of 2-categories $[\cK_\kappa,\cK]\rightarrow[\cK,\cK]_\kappa$
    (this is true for a general lfp cosmos ----missing bit, it was 11:23----).

    It follows that the inclusion $[\cK,\cK]_\kappa\rightarrow[\cK,\cK]$ is, up
    to equivalence, given by the left Kan extension along $J$. (Check and finish
    this proof)
\end{proof}

This allows us to write presentations of 2-monads for 2-categories such as
$\bbR$-linear categories, simplicial categories, etc, which has two important
consequences: firstly, when constructing a 2-monad from free monads we may also
use weighted colimits; secondly, since 2-monads with rank $\kappa$ are algebras
for a 2-monad with rank $\kappa$, any general theorem we prove about algebras
gives a corresponding 2-monad with rank $\kappa$.

As we mentioned earlier, we may be interested in less strict definitions
compared to the 1-dimensional case. Here we start considering them by specifying
new classes of morphisms of algebras.

\begin{defn}
    Let $T$ be a 2-monad, $(A,a)$, $(B,b)$ two $T$-algebras.

    A lax $T$-morphism is a pair $(f,\overline{f})$ where $f\colon A\rightarrow
    B$ is a 1-cell and $\overline{f}\colon b\cdot Tf\rightarrow f\cdot a$ is a
    2-cell such that the equations
    \[\begin{tikzcd}
        {T^2A} & TA & A & {=} & {T^2A} & TA & A \\
        {T^2B} & TB & B && {T^2B} & TB & B
        \arrow["a", from=1-2, to=1-3]
        \arrow["f", from=1-3, to=2-3]
        \arrow["b"', from=2-2, to=2-3]
        \arrow["Tf"{description}, from=1-2, to=2-2]
        \arrow["{T^2f}"', from=1-1, to=2-1]
        \arrow["{\mu_B}"', from=2-1, to=2-2]
        \arrow["{\mu_A}", from=1-1, to=1-2]
        \arrow["{\overline{f}}", shorten <=6pt, shorten >=6pt, Rightarrow, from=2-2, to=1-3]
        \arrow[shorten <=8pt, shorten >=8pt, Rightarrow, no head, from=2-1, to=1-2]
        \arrow["b"', from=2-6, to=2-7]
        \arrow["f", from=1-7, to=2-7]
        \arrow["Tf"{description}, from=1-6, to=2-6]
        \arrow["Tb"', from=2-5, to=2-6]
        \arrow["a", from=1-6, to=1-7]
        \arrow["{T^2f}"', from=1-5, to=2-5]
        \arrow["Ta", from=1-5, to=1-6]
        \arrow["{\overline{f}}", shorten <=6pt, shorten >=6pt, Rightarrow, from=2-6, to=1-7]
        \arrow["{T\overline{f}}", shorten <=6pt, shorten >=6pt, Rightarrow, from=2-5, to=1-6]
    \end{tikzcd},\]
    \[\begin{tikzcd}
        A & TA & A & {=} & {\id_f} \\
        B & TB & B
        \arrow["f", from=1-3, to=2-3]
        \arrow["Tf"{description}, from=1-2, to=2-2]
        \arrow["f"', from=1-1, to=2-1]
        \arrow["{\eta_B}"', from=2-1, to=2-2]
        \arrow["b"', from=2-2, to=2-3]
        \arrow["a", from=1-2, to=1-3]
        \arrow["{\eta_A}", from=1-1, to=1-2]
        \arrow["{\overline{f}}", shorten <=6pt, shorten >=6pt, Rightarrow, from=2-2, to=1-3]
        \arrow[shorten <=8pt, shorten >=8pt, Rightarrow, no head, from=2-1, to=1-2]
    \end{tikzcd}\]
    hold.

    A lax $T$-morphism is a pseudo $T$-morphism if $\overline{f}$ is an
    isomorphism and it is strict if $\overline{f}=\id$.

    A colax or oplax $T$-morphism is a lax $T$-morphism with the direction of
    $\overline{f}$ reversed and the equations adapted.

    A 2-cell between lax/pseudo/strict $T$-morphisms $\alpha\colon
    (f,\overline{f})\Rightarrow(g,\overline{g})$ is a 2-cell $\alpha\colon
    f\Rightarrow g$ s.t.\
    \[\begin{tikzcd}
        TA & A & {=} & TA & A \\
        TB & B && TB & B
        \arrow[""{name=0, anchor=center, inner sep=0}, "g", curve={height=-12pt}, from=1-2, to=2-2]
        \arrow[""{name=1, anchor=center, inner sep=0}, "f"', curve={height=12pt}, from=1-2, to=2-2]
        \arrow["b"', from=2-1, to=2-2]
        \arrow["a", from=1-1, to=1-2]
        \arrow[""{name=2, anchor=center, inner sep=0}, "Tf"', curve={height=12pt}, from=1-1, to=2-1]
        \arrow[""{name=3, anchor=center, inner sep=0}, "g", curve={height=-12pt}, from=1-5, to=2-5]
        \arrow["b"', from=2-4, to=2-5]
        \arrow["a", from=1-4, to=1-5]
        \arrow[""{name=4, anchor=center, inner sep=0}, curve={height=-12pt}, from=1-4, to=2-4]
        \arrow[""{name=5, anchor=center, inner sep=0}, curve={height=12pt}, from=1-4, to=2-4]
        \arrow["T\alpha", shorten <=5pt, shorten >=5pt, Rightarrow, from=5, to=4]
        \arrow["\alpha", shorten <=5pt, shorten >=5pt, Rightarrow, from=1, to=0]
        \arrow["{\overline{f}}", shorten <=13pt, shorten >=13pt, Rightarrow, from=2, to=1]
        \arrow["{\overline{g}}", shorten <=13pt, shorten >=13pt, Rightarrow, from=4, to=3]
    \end{tikzcd}\]

    We write $T\Alg_S$, $T\Alg_P$ and $T\Alg_L$ for the 2-categories of
    $T$-algebras, strict/pseudo/lax $T$-morphisms and 2-cells as above.
\end{defn}

(Other missing bit)

\backmatter

\end{document}
